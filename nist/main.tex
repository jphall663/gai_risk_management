\documentclass[fleqn]{article}
\renewcommand\refname{}
\title{Incorporating Generative AI Into Model Governance Programs}
\author{Patrick Hall, ...}

\usepackage{graphicx}
\usepackage{fullpage}
\usepackage{pdfpages}
\usepackage{amsmath}
\usepackage{amssymb}
\usepackage{mathtools}
\usepackage{MnSymbol}
\usepackage{enumerate}
\usepackage{setspace}
\usepackage[hyphens]{url}
\usepackage[colorlinks]{hyperref}
\usepackage{float}
\usepackage{caption}
\usepackage{subcaption}
\usepackage{multicol}
\usepackage{color}
\usepackage{listings}
\usepackage{csvsimple}
\usepackage{algorithm}
\usepackage{algorithmic}
\usepackage{verbatim}
\usepackage{mdframed}
\usepackage{changepage}
\usepackage[top=1in, bottom=1in, left=1in, right=1in]{geometry}

\usepackage{booktabs}
\usepackage{pdflscape}
\usepackage{makecell}
\usepackage{multirow}
\usepackage{enumitem}
\usepackage{tabto}

\begin{document}

\maketitle

\begin{abstract}
	

\end{abstract}

% ---------- ----------
\section{Introduction} \label{sec:intro}
% ---------- ----------

The National Institute of Standards and Technology Artificial Intelligence (AI) Risk Management Framework (RMF).\cite{airmf}

% requires mixing MRM, ERM, and security 

% Generally model governance from SR 11-7, FHA, NIST AI RMF, NIST ERM, ISO ERM all apply at the policy level, but there are a few differences.

% What's different?
% - Supply chain/third party 
% - Complex human AI interaction
% - Environmental impact 

% Security approaches may also be important
% - Bug bounties
% - Red-teaming 
% - Incident response 

% ---------- ----------
\section{Generative AI Governance}\label{sec:govern}
% ---------- ----------

% legal requirements: IP, hiring/employment, CSAM, reasonable accomodation, data priavcy
% executive buy-in and participation
% update/create definitions
% - model vs. tool
% establish a charter and policies
% risk tolerance
% initial communication
% Start inventory
% Procedures or standards will have to be updated signficantly, see, e.g., NIST AI 600-1 
% - Create and apply risk tiers
% -- Risk measurement 
% - Risk management
% - Refer to appendix A for how existing policies can be aligned to AI/GAI risk materials  
% coordination with security, fair lending/HR, data privacy, data management/governance, IT, legal 

% ---------- ----------
\section{Generative AI Policies and Procedures}\label{sec:polpro}
% ---------- ----------

% Polices should require light-touch update, new procedures may be required
% Need a general policy as catch-all 
% Embedded tools, should not be models, should not waste risk management resources

% ---------- ----------
\subsection{General Policies}\label{sec:genpol}
% ---------- ----------

% update policies, should be easy 
% but need an acceptable/prohibited use policy/end use compute policy as catch all 
% - provide bullets?
% -- Anonymous Use
% -- Anthropomorphization of Generated Content or User Interfaces
% -- Automated, Direct, or Material Decision-making (e.g., legal, financial, employment, pay, promotion)
% -- Circumvention of Blocklists, Blacklists, Organizational Policies, Safety Filters, or System Content Moderation Mechanisms
% -- Direct Publication or Use of Generated Content Without Human Review 
% -- Generation of Harmful Content, or Content Relating to: blackmail, bullying, conventional or unconventional weapons (chemical, biological, radiological, nuclear (CBRN)), criminal advise (advice on committing any crime or solicitation of criminal services), denigration, disinformation, extortion, extremism, “-omic” information (e.g., synthetic nucleotide sequences), intimidation, malware, misinformation, obscenity, personally identifiable information, phishing, radicalization, self-harm, pornography, solicitation or synthesis of controlled or illegal compounds, spam, stereotyping, suicide, threats, terrorism, or violence
% -- Impersonation of Any Person
% -- Input of Plagiarized, Copyrighted, Trademarked, Patented, Licensed, or Trade Secret Material
% -- Reverse Engineering
% -- Undisclosed Monitoring 
% -- Undisclosed Use
% -- Unauthorized alteration of Fine-tuning Data
% -- Unauthorized Connections to the Internet or Unsecured Networks
% -- Unauthorized External Sharing of Output
% -- Example: https://policies.google.com/terms/generative-ai/use-policy

% ---------- ----------
\subsection{Embedded Tools}\label{sec:tool}
% ---------- ----------

% not models, not validated by risk management
% kept in inventory
% managed 
% - by acceptabled use policy, other general policies, 
% - IT
% - acquistion, procurement
% - contracts
% - relationships with vendors 

% ---------- ----------
\subsection{Procedures}\label{sec:genpro}
% ---------- ----------

% procedures are lower-level, sometimes refered to as standards
% details should be updated for GAI

% ---------- ----------
\subsubsection{Considerations for Data}\label{sec:data}
% ---------- ----------

% data/supply chain issues
% - Ad-hoc Data Collection and Web-scraping
% - Demographic, Biometric, or Personally Identifiable Information (PII)
% - Plagiarized, Trademarked, Patented, Licensed, or Trade Secret Materials
% - Data Provenance, Quality and Grounding
% -- Data Quality
% -- Ground Truth Data/”Grounding”
% - Fulsome description of data 

% ---------- ----------
\subsubsection{Considerations for Model Training and Updates}\label{sec:train}
% ---------- ----------

% training processes and budgets 
% - can require truly industrial strength compute and coordination with IT
% -- hyperparameter search very expensive
% - embeddings, RAG, fine-tuning?
% - assumptions and limitations
% - alternative models
% - design documentation and diagrams

% ---------- ----------
\subsubsection{Considerations for Model Training and Updates}\label{sec:valid}
% ---------- ----------

% validation, refer to risk measurement section and appendices C, D, F, G, H
% - model testing
% - red-teaming
% - field testing

% ---------- ----------
\subsubsection{Considerations for Monitoring}\label{sec:monitor}
% ---------- ----------

% Ongoing Monitoring and Control
% - Ongoing benchmarks
% - User feedback mechanisms
% - Must also monitor any moderation or provenance systems
% Operational Interruptions
% Retention, Monitoring and Review of GAI Input and Output (with appropriate data privacy controls)
% Tracking GAI System Output Over Time (e.g., term frequencies or embedding orientations, with comparison to pre-deployment baseline)
% Monitor output for Plagiarized, Trademarked, Patented, Licensed, or Trade Secret Materials
% Monitor for Over-engagement (e.g., emotional entanglement) or Under-engagement (e.g., algorithmic aversion)

% ---------- ----------
\subsubsection{Considerations for Incident Response}\label{sec:response}
% ---------- ----------

% Incident response: 
% - “Kill-switch” Processes
% - Redundant or Fall-back Capabilities
% - Identification, Containment and Recovery
% - Rehearsal, Tabletops and Retrospective Learning

% ---------- ----------
\subsubsection{Considerations for Change Management}\label{sec:change}
% ---------- ----------

% Change Management:
% - Updates to data management
% - Model Fine-tuning, Refresh or Retrain cadences 
% - Application or User-interface (Tool) Updates
% - Vendor Communication and Updates 

% ---------- ----------
\subsubsection{Considerations for System Decommission}\label{sec:decom}
% ---------- ----------

% Decommission
% - Contractual/Legal Obligations
% - Cloud Resources
% - Data Retention
% - Recourse Processes
% - User Communication
% -- Users’ emotional entanglement with GAI functions.
% - Data security, e.g., containment, protocols, data leakage after decommissioning
% - Dependencies between upstream, downstream, or other data, internet of things (IOT) or AI systems

% ---------- ----------
\subsubsection{Considerations for Third Parties and Value Chains}\label{sec:3p}
% ---------- ----------

% TPRM/Supply chains

% - Update GAI acceptable use policies to address proprietary and open-source GAI technologies and data, and contractors, consultants, and other third-party personnel.

% - Open source counts at third party!

% - Engagement/Model Owner (?)
% - GAI Vendor Due Diligence:
% -- Contract, Services Agreement, or Warranty Review
% -- Review vendor contracts and avoid arbitrary or capricious termination of critical GAI technologies or vendor services and non-standard terms that may amplify or defer liability in unexpected ways and/or contribute to unauthorized data collection by vendors or third-parties (e.g., secondary data use). 
% -- Cross-reference/Update Whitelists
% -- Review of Vendor Documentation and Instructions 
% -- Review of Vendor Financial Stability and Analyst Reports
% -- Review of Vendor AI Risk Management
% -- Review of Past Incidents Involving Vendor or Technology (AI Incident Database, AVID, OWASP Top-10 for LLMs, MITRE Att&ck ATLAS, OECD Incident Monitor)
% - Request: 
% -- Notification and disclosure for serious incidents arising from third-party data and systems
% -- Service Level Agreements (SLAs) in vendor contracts that address incident response, response times, and availability of critical support.
% - Data Retention and Secondary Use
% - Monitoring and Telemetrics 
% - Inventory

% ---------- ----------
\section{Generative AI Inventories}\label{sec:inv}
% ---------- ----------

% Data provenance information (e.g., source, signatures, versioning, watermarks)
% Known issues reported from internal bug tracking or external information sharing resources (e.g., AI incident database, AVID, CVE, NVD, or OECD AI incident monitor)
% Human oversight roles and responsibilities 
% Special rights and considerations for intellectual property, licensed works, or personal, privileged, proprietary or sensitive data
% Underlying foundation models, versions of underlying models, and access modes
% Challenges for explainability, interpretability, or transparency
% Connections or dependencies between other systems
% incident response plans 

% TPRM/Supply chains

% ---------- ----------
\section{Generative AI Documentation Templates}\label{sec:doc}
% ---------- ----------

% enable transparency and efficient, standardized review

% - Business justification
% -- consider potentially high costs of implementing GAI applications
% - Scope and usages
% -- Narrow scope enables better risk managements
% - Assumptions and limitations
% - Description and characterization of training data
% -- Requires greater detail than "text" for adequate risk management
% - Algorithmic methodology
% -- Embeddings
% -- Weights 
% -- Considerations for RAG or Guardrail models
% - Evaluated alternative approaches
% - Description of output data, modalities etc. 
% - Testing and validation results (including explanatory visualizations and information) 
% -- model testing
% -- red-teaming 
% -- field testing
% - User feedback/stakeholder engagement plans

% Consider nutrition labels for users 

% TPRM/Supply chains

% ---------- ----------
\section{Generative AI Risk Tiers}\label{sec:tiers}
% ---------- ----------

% Refer to appendix B 

% Likelihood x impact

% Risk tiers, risk assessment instruments, risk calculators

% Primary considerations: 
% - Internal vs. external use
% - Unreliable decision making capabilities
% - Legal or regulatory requirements bvs. reputational risk

% Additional considerations: 
% - Irregular cadence of vendor releases and updates
% - Dependencies between GenAI and other IT or data systems
% - Human review of GenAI system outputs
% - Immediate and long term impacts
% - Confabulation
% - Dangerous and Violent Recommendations
% - Data Privacy
% - Environmental impact 
% - Human-AI Configuration
% - Information Integrity
% - Information Security
% - Intellectual Property
% - Toxicity, Bias, and Homogenization
% - Value Chain and Component Integration

% ---------- ----------
\subsection{Generative AI Incidents}\label{sec:incident}
% ---------- ----------

% Use them to identify realistic risks
% Incidents often related to ... 
% - Confabulation
% - Deepfakes
% - Data privacy/consent issues/IP 
% - Autonomous vehicle issues

% Can help understand real likelihoods and impacts, and what is not real

% ---------- ----------
\subsection{Generative AI Risk Measurement}\label{sec:measure}
% ---------- ----------

% Refer to appendix F, G, H

% Low risk: evals/model testing
% - evals cant test some things -- explainability, HAI-config, supply chain
% Medium risk: model testing + red-teaming
% High risk: model testing + red-teaming + field testing

% TPRM/Supply chains

% ---------- ----------
\section{Generative AI Risk Management}\label{sec:manage}
% ---------- ----------

% mitigation, refer to appendix E, F, G, H
 
% - Output Data Provenance/Watermarking/Signatures/Steganography
% - Restriction of API or Programmatic Access
% - Session, Rate, Time, or Volume Limits
% - Abuse detection
% - Accessibility
% - Citation
% - Clear instructions
% - Content moderation/filters
% - Disclosure of AI interaction
% - Dynamic blocklists
% - Ground truth training data
% - Kill switches
% - Incident response plans
% - Pre-approved responses 
% - Retrieval augmented generation (RAG) approaches
% - Strong system prompts
% - Redundant or Fall-back Capabilities
% - No:
% -- Anonymous use
% -- Anthropomorphization 
% -- Bots
% -- Internet access
% -- Minors
% -- Personal/sensitive training data
% -- Regulated applications
% -- Undisclosed data collection 

% TPRM/Supply chain

% ---------- ----------
\section*{Conclusion}
% ---------- ----------

% ---------- ----------
\section*{Acknowledgments}
% ---------- ----------

Thank you to Bernie Siskin and Nick Schmidt of BLDS and Eric Sublett of Relman Colfax for formative discussions relating to GAI risk tiering. 

% ---------- ----------
\section*{Abbreviations}
% ---------- ----------

\begin{itemize}
	\item AI: Artificial Intelligence
	\item AI RMF: Artificial Intelligence Risk Management Framework
	\item GAI: Generative AI
	\item LLM: Large Language Model
	\item RMF: Risk Management Framework
\end{itemize}

% ---------- ----------
\bibliographystyle{plain}
\bibliography{bibliography}
% ---------- ----------

\begin{landscape}
\thispagestyle{empty}	
% ---------- ----------
\section*{Appendix A: Example Generative AI--Trustworthy Characteristic Crosswalk}\label{sec:appndxa}
% ---------- ----------

% Cross enables applying guidance for AI RMF to risks 
% Other interpretations or crosswalks are possible

% ---------- ----------
\subsection*{A.1: Trustworthy Characteristic to Generative AI Risk Crosswalk}\label{sec:appndxa1}
% ---------- ----------

\begin{table}[H]
	\caption*{Table A.1: Trustworthy Characteristic to Generative AI Risk Crosswalk.}
	\label{tab:tc_to_gai_risk_cw}
	\footnotesize
	\begin{tabular}{llll}
		\toprule
		\textbf{Accountable and Transparent} & \textbf{Explainable and Interpretable} & \textbf{Fair with Harmful Bias Managed} & \textbf{Privacy Enhanced} \\
		\midrule
		Data Privacy & Human-AI Configuration & Confabulation & Data Privacy \\
		Environmental & Value Chain and Component Integration & Environmental & Human-AI Configuration \\
		Human-AI Configuration &  & Human-AI Configuration & Information Security \\
		Information Integrity &  & Intellectual Property & Intellectual Property \\
		Intellectual Property &  & Obscene, Degrading, and/or Abusive Content & Value Chain and Component Integration \\
		Value Chain and Component Integration &  & Toxicity, Bias, and Homogenization &  \\
 		&  & Value Chain and Component Integration &  \\
 		&  &  &  \\
 		&  &  &  \\
 		&  &  &  \\
		\bottomrule
	\end{tabular}
	\newline
	\vspace{10pt}
	\newline
	\begin{tabular}{lll}
		\toprule
		\textbf{Safe} & \textbf{Secure and Resilient} & \textbf{Valid and Reliable} \\
		\midrule
		CBRN Information & Dangerous or Violent Recommendations & Confabulation \\
		Confabulation & Data Privacy & Human-AI Configuration \\
		Dangerous or Violent Recommendations & Human-AI Configuration & Information Integrity \\
		Data Privacy & Information Security & Information Security \\
		Environmental & Value Chain and Component Integration & Toxicity, Bias, and Homogenization \\
		Human-AI Configuration &  & Value Chain and Component Integration \\
		Information Integrity &  &  \\
		Information Security &  &  \\
		Obscene, Degrading, and/or Abusive Content &  &  \\
		Value Chain and Component Integration &  &  \\
		\bottomrule
	\end{tabular}
\end{table}

\noindent\textbf{Usage Note}: Table A.1 provides an example of mapping GAI risks onto AI RMF trustworthy characteristics. Mapping GAI risks to AI RMF trustworthy characteristics can be particularly useful when existing policies, processes, or controls can be applied to manage GAI risks, but have been previously implemented  in alignment with the AI RMF trustworthy characteristics. Many mappings are possible. Mappings that differ from the example may be more appropriate to meet a particular organization's risk management goals. 

\vfill
\raisebox{-10pt}{\makebox[\linewidth]{\thepage}}
\end{landscape}

\begin{landscape}
\thispagestyle{empty}	
% ---------- ----------
\subsection*{A.2: Generative AI Risk to Trustworthy Characteristic Crosswalk}\label{sec:appndxa2}
% ---------- ----------

\begin{table}[H]
	\caption*{Table A.2: Generative AI Risk to Trustworthy Characteristic Crosswalk.}
	\label{tab:gai_risk_to_tc_cw}
	\footnotesize
	\begin{tabular}{llll}
		\toprule
		\textbf{CBRN Information} & \textbf{Confabulation} & \textbf{Dangerous or Violent Recommendations} & \textbf{Data Privacy} \\
		\midrule
		Safe & Fair with Harmful Bias Managed & Safe & Accountable and Transparent \\
 		& Safe & Secure and Resilient & Privacy Enhanced \\
 		& Valid and Reliable &  & Safe \\
 		&  &  & Secure and Resilient \\
	\bottomrule
	\end{tabular}
	\newline
	\vspace{10pt}
	\newline	
	\begin{tabular}{llll}
		\toprule
		\textbf{Environmental} & \textbf{Human-AI Configuration} & \textbf{Information Integrity} & \textbf{Information Security} \\
		\midrule
		Accountable and Transparent & Accountable and Transparent & Accountable and Transparent & Privacy Enhanced \\
		Fair with Harmful Bias Managed & Explainable and Interpretable & Safe & Safe \\
		Safe & Fair with Harmful Bias Managed & Valid and Reliable & Secure and Resilient \\
 		& Privacy Enhanced &  & Valid and Reliable \\
 		& Safe &  &  \\
 		& Secure and Resilient &  &  \\
 		& Valid and Reliable &  &  \\
		\bottomrule
	\end{tabular}
	\newline
	\vspace{10pt}
	\newline
	\begin{tabular}{llll}
		\toprule
		\textbf{Intellectual Property} & \textbf{Obscene, Degrading, and/or Abusive Content} & \textbf{Toxicity, Bias, and Homogenization} & \textbf{Value Chain and Component Integration} \\
		\midrule
		Accountable and Transparent & Fair with Harmful Bias Managed & Fair with Harmful Bias Managed & Accountable and Transparent \\
		Fair with Harmful Bias Managed & Safe & Valid and Reliable & Explainable and Interpretable \\
		Privacy Enhanced &  &  & Fair with Harmful Bias Managed \\
 		&  &  & Privacy Enhanced \\
 		&  &  & Safe \\
 		&  &  & Secure and Resilient \\
 		&  &  & Valid and Reliable \\		
		\bottomrule
	\end{tabular}
\end{table}

\noindent\textbf{Usage Note}: Table A.2 provides an example of mapping AI RMF trustworthy characteristics onto GAI risks. Mapping AI RMF trustworthy characteristics to GAI risks can assist organizations in aligning GAI guidance to existing AI/ML policies, processes, or controls or to extend GAI guidance to address additional AI/ML technologies. Many mappings are possible. Mappings that differ from the example may be more appropriate to meet a particular organization's risk management goals.

\vfill
\raisebox{-10pt}{\makebox[\linewidth]{\thepage}}
\end{landscape}

\begin{landscape}
\thispagestyle{empty}	
% ---------- ----------
\subsection*{A.3: Traditional Banking Risks, Generative AI Risks and Trustworthy Characteristics Crosswalk}\label{sec:appndxa3}
% ---------- ----------
\begin{table}[H]
	\caption*{Table A.3: Traditional Banking Risks, Generative AI Risks and Trustworthy Characteristics Crosswalk.}
	\label{tab:bankrisk_gai_risk_tc}
	\footnotesize
	\begin{tabular}{llll}
		\toprule
		\textbf{Compliance Risk} & \textbf{Information Security Risk} &  \textbf{Legal Risk} & \textbf{Model Risk} \\ 
		\midrule
		Data Privacy &  Data Privacy &  Intellectual Property & Confabulation \\
		Information Security &  Information Security & Obscene, Degrading, and/or Abusive Content & Dangerous or Violent Recommendations \\
		Toxicity, Bias, and Homogenization  & Value Chain and Component Integration & Value Chain and Component Integration & Information Integrity \\
		Value Chain and Component Integration & & & Obscene, Degrading, and/or Abusive Content \\
		& & & Toxicity, Bias, and Homogenization \\
		\midrule
		Accountable and Transparent & Privacy Enhanced & Accountable and Transparent & Valid and Reliable \\
		Fair with Harmful Bias Managed &  Secure and Resilient & Safe & \\
		Privacy Enhanced & & & \\
		Secure and Resilient & & & \\
		\bottomrule
	\end{tabular}
	\newline
	\vspace{10pt}
	\newline	
	\begin{tabular}{llll}
		\toprule
		\textbf{Operational Risk} & \textbf{Reputational Risk} & \textbf{Strategic Risk} & \textbf{Third Party Risk} \\
		\midrule
	 	Confabulation & Confabulation & Environmental & Information Integrity \\
	 	Human-AI Configuration & Dangerous or Violent Recommendations & Information Integrity & Value Chain and Component Integration \\
	 	Information Security & Environmental & Information Security & \\
	 	Value Chain and Component Integration & Human-AI Configuration & Value Chain and Component Integration & \\
	 	& Information Integrity & & \\
	 	& Obscene, Degrading, and/or Abusive Content & & \\
	 	& Toxicity, Bias, and Homogenization & & \\
	 	\midrule 
	 	Safe & Accountable and Transparent & Accountable and Transparent & Accountable and Transparent \\
	 	Secure and Resilient & Fair with Harmful Bias Managed & Secure and Resilient & Explainable and Interpretable \\
	 	Valid and Reliable & Valid and Reliable & Valid and Reliable & \\
		\bottomrule
	\end{tabular}
\end{table}

\noindent\textbf{Usage Note}: Table A.3 provides an example of mapping GAI risks and AI RMF trustworthy characteristics. This type of mapping can enable incorporation of new AI guidance into existing policies, processes, or controls or the application of existing policies, processes, or controls to newer AI risks.

\vfill
\raisebox{-10pt}{\makebox[\linewidth]{\thepage}}
\end{landscape}

% ---------- ----------
\section*{Appendix B: Example Risk-tiering Materials for Generative AI}\label{sec:appndxb}
% ---------- ----------

\subsection*{B.1: Example Adverse Impacts}

\begin{table}[H]
    \caption*{Table B.1: Example adverse impacts, adapted from NIST 800-30r1 Table H-2 \cite{nist80030r1}.}
    \footnotesize
        \begin{tabular}{|m{0.20\linewidth} | m{0.75\linewidth}|}
            \hline
            \textbf{Level} & \textbf{Description} \\ \hline
            Harm to Operations & 
            \begin{itemize}[noitemsep]
           		\item Inability to perform current missions/business functions.
           		\begin{itemize}[noitemsep,nolistsep]
           			\item In a sufficiently timely manner.
           			\item With sufficient confidence and/or correctness.
           			\item Within planned resource constraints.
           		\end{itemize}
           		\item Inability, or limited ability, to perform missions/business functions in the future.           	
           		\begin{itemize}[noitemsep,nolistsep]
           			\item Inability to restore missions/business functions.
           			\item In a sufficiently timely manner.
					\item With sufficient confidence and/or correctness.
					\item Within planned resource constraints.
           		\end{itemize}
           		\item Harms (e.g., financial costs, sanctions) due to noncompliance. 
           		\begin{itemize}[noitemsep,nolistsep]
           			\item With applicable laws or regulations.
           			\item With contractual requirements or other requirements in other binding agreements (e.g., liability).
           		\end{itemize}
           		\item Direct financial costs.
           		\item Reputational harms.	
           		\begin{itemize}[noitemsep,nolistsep]
           			\item Damage to trust relationships.
           			\item Damage to image or reputation (and hence future or potential trust relationships).
           		\end{itemize}
            \end{itemize} \\
            \hline
            Harm to Assets & 
            \begin{itemize}[noitemsep]
				\item Damage to or loss of physical facilities.
				\item Damage to or loss of information systems or networks.
				\item Damage to or loss of information technology or equipment.
				\item Damage to or loss of component parts or supplies.
				\item Damage to or of loss of information assets.
				\item Loss of intellectual property.            	
            \end{itemize} \\
            \hline
            Harm to Individuals & 
            \begin{itemize}[noitemsep]
				\item Injury or loss of life.
				\item Physical or psychological mistreatment.
				\item Identity theft.
				\item Loss of personally identifiable information.
				\item Damage to image or reputation.
				\item Infringement of intellectual property rights.
				\item Financial harm or loss of income.	
			\end{itemize} \\            
            \hline
            Harm to Other Organizations & 
            \begin{itemize}[noitemsep]
				\item Harms (e.g., financial costs, sanctions) due to noncompliance. 
				\begin{itemize}[noitemsep,nolistsep]
					\item With applicable laws or regulations.
					\item With contractual requirements or other requirements in other binding agreements (e.g., liability).
				\end{itemize}
				\item Direct financial costs.
				\item Reputational harms.	
				\begin{itemize}[noitemsep,nolistsep]
					\item Damage to trust relationships.
					\item Damage to image or reputation (and hence future or potential trust relationships).	
				\end{itemize}
			\end{itemize} \\            
            \hline
            Harm to the Nation & 
            \begin{itemize}[noitemsep]
				\item Damage to or incapacitation of critical infrastructure.
				\item Loss of government continuity of operations.
				\item Reputational harms.
				\begin{itemize}[noitemsep,nolistsep]
					\item Damage to trust relationships with other governments or with nongovernmental entities.
					\item Damage to national reputation (and hence future or potential trust relationships).
				\end{itemize} 
				\item Damage to current or future ability to achieve national objectives.
				\begin{itemize}[noitemsep,nolistsep]
					\item Harm to national security.
				\end{itemize}
				\item Large-scale economic or workforce displacement.
			\end{itemize} \\            
            \hline
    \end{tabular}
    \label{nist:adverse_impacts}
\end{table}

\subsection*{B.2: Example Impact Descriptions}

\begin{table}[H]
	\caption*{Table B.2: Example Impact level descriptions, adapted from NIST SP800-30r1 Appendix H, Table H-3 \cite{nist80030r1}.}
	\small
    \begin{tabular}{|m{0.20\linewidth} |m{0.20\linewidth} | m{0.05\linewidth} | m{0.40\linewidth}|}
        \hline
        \textbf{Qualitative Values} & \multicolumn{2}{c|}{\textbf{Semi-Quantitative Values}} & \textbf{Description} \\
        \hline
        Very High & 96-100 & 10 & An incident could be expected to have multiple severe or catastrophic adverse effects on organizational operations, organizational assets, individuals, other organizations, or the Nation. \\
        \hline
        High & 80-95 & 8 & An incident could be expected to have a severe or catastrophic adverse effect on organizational operations, organizational assets, individuals, other organizations, or the Nation. A severe or catastrophic adverse effect means that, for example, the incident might: (i) cause a severe degradation in or loss of mission capability to an extent and duration that the organization is not able to perform one or more of its primary functions; (ii) result in major damage to organizational assets; (iii) result in major financial loss; or (iv) result in severe or catastrophic harm to individuals involving loss of life or serious life-threatening injuries. \\
        \hline		
        Moderate & 21-79 & 5 & An incident could be expected to have a serious adverse effect on organizational operations, organizational assets, individuals other organizations, or the Nation. A serious adverse effect means that, for example, the incident might: (i) cause a significant degradation in mission capability to an extent and duration that the organization is able to perform its primary functions, but the effectiveness of the functions is significantly reduced; (ii) result in significant damage to organizational assets; (iii) result in significant financial loss; or (iv) result in significant harm to individuals that does not involve loss of life or serious life-threatening injuries. \\
        \hline
        Low & 5-20 & 2 & An incident could be expected to have a limited adverse effect on organizational operations, organizational assets, individuals other organizations, or the Nation. A limited adverse effect means that, for example, the incident might: (i) cause a degradation in mission capability to an extent and duration that the organization is able to perform its primary functions, but the effectiveness of the functions is noticeably reduced; (ii) result in minor damage to organizational assets; (iii) result in minor financial loss; or (iv) result in minor harm to individuals. \\
        \hline
        Very Low & 0-4 & 0 & An incident could be expected to have a negligible adverse effect on organizational operations, organizational assets, individuals other organizations, or the Nation. \\
        \hline
    \end{tabular}
    \label{table:nist_impacts}
\end{table}

\subsection*{B.3: Example Likelihood Descriptions}

\begin{table}[H]
	\caption*{Table B.3: Example likelihood levels, adapted from NIST SP800-30r1 Appendix G, Table G-3 \cite{nist80030r1}.}
    \begin{tabular}{|m{0.20\linewidth} |m{0.20\linewidth} | m{0.05\linewidth} | m{0.40\linewidth}|}
        \hline
        \textbf{Qualitative Values} & \multicolumn{2}{c|}{\textbf{Semi-Quantitative Values}} & \textbf{Description} \\
        \hline
        Very High & 96-100 & 10 & An incident is almost certain to occur; or occurs more than 100 times a year. \\
        \hline
        High & 80-95 & 8 & An incident is highly likely to occur; or occurs between 10-100 times a year.\\
        \hline		
        Moderate & 21-79 & 5 & An incident is somewhat likely to occur; or occurs between 1-10 times a year.\\
        \hline
        Low & 5-20 & 2 & An incident is unlikely to occur; or occurs less than once a year, but more than once every 10 years.  \\
        \hline
        Very Low & 0-4 & 0 & An incident is highly unlikely to occur; or occurs less than once every 10 years. \\
        \hline
    \end{tabular}
    \label{table:nist_likelihood}
\end{table}

\subsection*{B.4: Example Risk Tiers}

\begin{table}[H]
    \caption*{Table B.4: Example risk assessment matrix with 5 impact levels, 5 likelihood levels, and 5 risk tiers, adapted from NIST SP800-30r1 Appendix I, Table I-2 \cite{nist80030r1}.}
    \small
    \begin{tabular}{|c|c|c|c|c|c|c|}
        \hline
        \multirow{2}{*}{\textbf{Likelihood}} & \multicolumn{5}{|c|}{\textbf{Level of Impact}}   \\
        \cline{2-6}
        & \textbf{Very Low} & \textbf{Low} & \textbf{Moderate} & \textbf{High} & \textbf{Very High} \\
        \hline
        \textbf{Very High} & Very Low (Tier 5) & Low (Tier 4) & Moderate (Tier 3) & High (Tier 2) & Very High 
        (Tier 1) \\
        \hline		
        \textbf{High} & Very Low (Tier 5) & Low (Tier 4) & Moderate (Tier 3)  & High (Tier 2) & Very High (Tier 1) \\
        \hline
        \textbf{Moderate} & Very Low (Tier 5) & Low (Tier 4) & Moderate (Tier 3)  & Moderate (Tier 3) & High (Tier 2) \\
        \hline
        \textbf{Low} & Very Low (Tier 5) & Low (Tier 4) & Low (Tier 4)  & Low (Tier 4) & Moderate (Tier 3) \\
        \hline
        \textbf{Very Low} & Very Low (Tier 5) & Very Low (Tier 5) & Very Low (Tier 5) & Low (Tier 4) & Low (Tier 4) \\
        \hline
    \end{tabular}
    \label{table:nist_risk_tiers}
\end{table}

\subsection*{B.5: Example Risk Descriptions}

\begin{table}[H]
	\caption*{Table B.5: Example risk descriptions, adapted from NIST SP800-30r1 Appendix I, Table I-3 \cite{nist80030r1} .}
    \small
    \begin{tabular}{|m{0.20\linewidth} |m{0.20\linewidth} | m{0.05\linewidth} | m{0.40\linewidth}|}
        \hline
        \textbf{Qualitative Values} & \multicolumn{2}{c|}{\textbf{Semi-Quantitative Values}} & \textbf{Description} \\
        \hline
        Very High & 96-100 & 10 & Very high risk means that an incident could be expected to have multiple severe or catastrophic adverse effects on organizational operations, organizational assets, individuals, other organizations, or the Nation.\\
        \hline
        High & 80-95 & 8 & High risk means that an incident could be expected to have a severe or catastrophic adverse effect on organizational operations, organizational assets, individuals, other organizations, or the Nation. \\
        \hline		
        Moderate & 21-79 & 5 & Moderate risk means that an incident could be expected to have a serious adverse effect on organizational operations, organizational assets, individuals, other organizations, or the Nation.\\
        \hline
        Low & 5-20 & 2 & Low risk means that an incident could be expected to have a limited adverse effect on organizational operations, organizational assets, individuals, other organizations, or the Nation. \\
        \hline
        Very Low & 0-4 & 0 & Very low risk means that an incident could be expected to have a negligible adverse effect on organizational operations, organizational assets, individuals, other organizations, or the Nation. \\
        \hline
    \end{tabular}
    \label{table:nist_risk_descriptions}
\end{table}

\subsection*{B.6: Practical Risk-tiering Questions}

\noindent\textbf{B.6.1: Confabulation}: How likely are system outputs to contain errors? What are the impacts if errors occur?\\

\noindent\textbf{B.6.2: Dangerous and Violent Recommendations}: How likely is the system to give dangerous or violent recommendations? What are the impacts if it does?\\

\noindent\textbf{B.6.3: Data Privacy}: How likely is someone to enter sensitive data into the system? What are the impacts if this occurs? Are standard data privacy controls applied to the system to mitigate potential adverse impacts?\\

\noindent\textbf{B.6.4: Human-AI Configuration}: How likely is someone to use the system incorrectly or abuse it? How likely is use for decision-making? What are the impacts of incorrect use or abuse? What are the impacts of invalid or unreliable decision-making?\\

\noindent\textbf{B.6.5: Information Integrity}: How likely is the system to generate deepfakes or mis or disinformation? At what scale? Are content provenance mechanisms applied to system outputs? What are the impacts of generating deepfakes or mis or disinformation? Without controls for content provenance?\\

\noindent\textbf{B.6.6: Information Security}: How likely are system resources to be breached or exfiltrated? How likely is the system to be used in the generation of phishing or malware content? What are the impacts in these cases? Are standard information security controls applied to the system to mitigate potential adverse impacts? \\

\noindent\textbf{B.6.7: Intellectual Property}: How likely are system outputs to contain other entities' intellectual property? What are the impacts if this occurs?\\

\noindent\textbf{B.6.8: Toxicity, Bias, and Homogenization}: How likely are system outputs to be biased, toxic, homogenizing or otherwise obscene? How likely are system outputs to be used as subsequent training inputs? What are the impacts of these scenarios? Are standard nondiscrimination controls applied to mitigate potential adverse impacts? Is the application accessible to all user groups? What are the impacts if the system is not accessible to all user groups?\\

\noindent\textbf{B.6.9: Value Chain and Component Integration}: Are contracts relating to the system reviewed for legal risks? Are standard acquisition/procurement controls applied to mitigate potential adverse impacts? Do vendors provide incident response with guaranteed response times? What are the impacts if these conditions are not met?

% ---------- ----------
\subsection*{B.7: AI Risk Management Framework Actions Aligned to Risk Tiering}\label{appndxb7}
% ---------- ----------

GOVERN 1.3, GOVERN 1.5, GOVERN 2.3, GOVERN 3.2, GOVERN 4.1, GOVERN 5.2, GOVERN 6.1, MANAGE 1.2, MANAGE 1.3, MANAGE 2.1, MANAGE 2.2, MANAGE 2.3, 
MANAGE 2.4, MANAGE 3.1, MANAGE 3.2, MANAGE 4.1, MAP 1.1, MAP 1.5, MEASURE 2.6 \\

\noindent \textbf{Usage Note}: Materials in Appendix B can be used to create or update risk tiers or other risk assessment tools for GAI systems or applications as follows:
\begin{itemize}
	\item Table B.1 can enable mapping of GAI risks and impacts. 
	\item Table B.2 can enable quantification of impacts for risk tiering or risk assessment. 
	\item Table B.3 can enable quantification of likelihood for risk tiering or risk assessment.
	\item Table B.4 presents an example of combining assessed impact and likelihood into risk tiers. 
	\item Table B.5 presents example risk tiers with associated qualitative, semi-quantitative, and quantitative values for risk tiering  or risk assessment.
	\item Subsection B.6 presents example questions for qualitative risk assessment.
	\item Subsection B.7 highlights subcategories to indicate alignment with the AI RMF.   
\end{itemize} 

\pagebreak

% ---------- ----------
\section*{Appendix C: List of Selected Model Testing Suites}\label{sec:appndxc}
% ---------- ----------

% ---------- ----------
\subsection*{C.1: Selected Model Testing Suites Organized by Trustworthy Characteristic}\label{appndxc1}
% ---------- ----------

\begin{table}[H]
	\caption*{Table C.1: Selected model testing suites organized by trustworthy characteristic. Adapted from AI Verify Evaluation Taxonimization \cite{aiverify_evals} and various additional resources.}
	\label{tab:low_risk_measure_by_tc}
	\footnotesize
	\begin{tabular}{l}
		\toprule
		\textbf{Accountable and Transparent} \\
		\midrule
			\makecell[l]{An Evaluation on Large Language Model Outputs: \\\hspace{10pt}Discourse and Memorization (see Appendix B)\cite{de2023evaluation}} \\
			Big-bench: Truthfulness \cite{bigbench} \\
			DecodingTrust: Machine Ethics \cite{decodingtrust}\\
			Evaluation Harness: ETHICS \cite{evalharness}\\
			HELM: Copyright \cite{helm} \\
			Mark My Words \cite{markmywords} \\
		\bottomrule
	\end{tabular}
	\newline
	\vspace{10pt}
	\newline
	\begin{tabular}{l}
		\toprule
		\textbf{Fair with Harmful Bias Managed} \\
		\midrule
		BELEBELE \cite{belebele} \\
		Big-bench: Low-resource language, Non-English, Translation  \\
		Big-bench: Social bias, Racial bias, Gender bias, Religious bias \\
		Big-bench: Toxicity \\
		DecodingTrust: Fairness \\
		DecodingTrust: Stereotype Bias \\
		DecodingTrust: Toxicity \\
		C-Eval (Chinese evaluation suite) \cite{ceval}\\
		Evaluation Harness: CrowS-Pairs  \\
		Evaluation Harness: ToxiGen \\
		Finding New Biases in Language Models with a Holistic Descriptor Dataset \cite{smith2022m}\\
		\makecell[l]{From Pretraining Data to Language Models to Downstream Tasks:\\\hspace{10pt}Tracking the Trails of Political Biases Leading to Unfair NLP Models \cite{feng2023pretraining}} \\
		HELM: Bias \\
		HELM: Toxicity \\
		MT-bench \cite{mtbench} \\
		The Self-Perception and Political Biases of ChatGPT \cite{rutinowski2023self}\\
		\makecell[l]{Towards Measuring the Representation of\\\hspace{10pt} Subjective Global Opinions in Language Models \cite{durmus2023towards}}\\
		\bottomrule
	\end{tabular}
	\newline
	\vspace{10pt}
	\newline
	\begin{tabular}{l}
		\toprule
		\textbf{Privacy Enhanced} \\
		\midrule
		HELM: Copyright\\
		llmprivacy \cite{llmprivacy}\\
		mimir \cite{mimir}\\
	\bottomrule
	\end{tabular}
	\newline
	\vspace{10pt}
	\newline	
	\begin{tabular}{l}
		\toprule
		\textbf{Safe} \\
		\midrule
			Big-bench: Convince Me \\
			Big-bench: Truthfulness \\
			HELM: Reiteration, Wedging \\
			Mark My Words \\
			MLCommons \cite{mlcommons} \\
			The WMDP Benchmark \cite{wmdp} \\
		\bottomrule
	\end{tabular}	
\end{table}	

\pagebreak 

\begin{table}[H]
	\caption*{Table C.1: Selected model testing suites organized by trustworthy characteristic (continued).}
	\label{tab:low_risk_measure_by_tc_cont}
	\footnotesize
	\begin{tabular}{l}
		\toprule
		\textbf{Secure and Resilient} \\
		\midrule
			Catastrophic Jailbreak of Open-source LLMs via Exploiting Generation \cite{huang2023catastrophic} \\
			\makecell[l]{DecodingTrust: Adversarial Robustness,\\\hspace{10pt} Robustness Against Adversarial Demonstrations} \\
			detect-pretrain-code \cite{detectpretraincode} \\
			Garak: encoding, knownbadsignatures, malwaregen, packagehallucination, xss \cite{garak} \\
			In-The-Wild Jailbreak Prompts on LLMs \cite{shen2023anything}\\
			JailbreakingLLMs \cite{chao2023jailbreaking}\\
			llmprivacy \\
			mimir \\
			TAP: A Query-Efficient Method for Jailbreaking Black-Box LLMs \cite{mehrotra2023tree}\\
		\bottomrule
	\end{tabular}
	\newline
	\vspace{10pt}
	\newline		
	\begin{tabular}{l}
		\toprule
		\textbf{Valid and Reliable} \\
		\midrule
			\makecell[l]{Big-bench: Algorithms, Logical reasoning, Implicit reasoning, Mathematics, Arithmetic, Algebra, Mathematical proof,\\\hspace{10pt} Fallacy, Negation, Computer code, Probabilistic reasoning, Social reasoning, Analogical reasoning, Multi-step,\\\hspace{10pt} Understanding the World} \\
		Big-bench: Analytic entailment, Formal fallacies and syllogisms with negation, Entailed polarity \\
		Big-bench: Context Free Question Answering \\
		Big-bench: Contextual question answering, Reading comprehension, Question generation \\
		Big-bench: Morphology, Grammar, Syntax \\
		Big-bench: Out-of-Distribution \\
		Big-bench: Paraphrase \\
		Big-bench: Sufficient information \\
		Big-bench: Summarization \\
		DecodingTrust: Out-of-Distribution Robustness, Adversarial Robustness, 			Robustness Against Adversarial Demonstrations\\
		Eval Gauntlet: Reading comprehension \cite{evalgauntlet} \\
		Eval Gauntlet: Commonsense reasoning, Symbolic problem solving, Programming \\
		Eval Gauntlet: Language Understanding  \\
		Eval Gauntlet: World Knowledge \\
		Evaluation Harness: BLiMP \\
		Evaluation Harness: CoQA, ARC \\
		Evaluation Harness: GLUE \\
		Evaluation Harness: HellaSwag, OpenBookQA, TruthfulQA \\
		Evaluation Harness: MuTual \\
		Evaluation Harness: PIQA, PROST, MC-TACO, MathQA, LogiQA, DROP \\
		FLASK: Logical correctness, Logical robustness, Logical efficiency, Comprehension, Completeness \cite{flask}  \\
		FLASK: Readability, Conciseness, Insightfulness \\
		HELM: Knowledge \\
		HELM: Language \\
		HELM: Text classification \\
		HELM: Question answering \\
		HELM: Reasoning \\
		HELM: Robustness to contrast sets \\
		HELM: Summarization \\
		Hugging Face: Fill-mask, Text generation \cite{huggingface} \\
		Hugging Face: Question answering \\
		Hugging Face: Summarization \\
		Hugging Face: Text classification, Token classification, Zero-shot classification \\
		MASSIVE \cite{massive} \\
		MT-bench \\
		\bottomrule
	\end{tabular}	
\end{table}

% ---------- ----------
\subsection*{C.2: Selected Model Testing Suites Organized by Generative AI Risk}\label{appndxc2}
% ---------- ----------
\begin{table}[H]
	\caption*{Table C.2: Selected model testing suites by organized generative AI risk.}
	\label{tab:low_risk_measure_by_gai_risk}
	\footnotesize
	\begin{tabular}{l}
		\toprule
		\textbf{CBRN Information} \\
		\midrule
			Big-bench: Convince Me \\
			Big-bench: Truthfulness \\
			HELM: Reiteration, Wedging \\
			MLCommons \\
			The WMDP Benchmark \\
		\bottomrule
	\end{tabular}
	\newline
	\vspace{10pt}
	\newline
	\begin{tabular}{l}
		\toprule
		\textbf{Confabulation} \\
		\midrule
		BELEBELE \\
		\makecell[l]{Big-bench: Algorithms, Logical reasoning, Implicit reasoning, Mathematics, Arithmetic, Algebra,\\\hspace{10pt} Mathematical proof, Fallacy, Negation, Computer code, Probabilistic reasoning, Social reasoning,\\\hspace{10pt}  Analogical reasoning, Multi-step, Understanding the World} \\
		Big-bench: Analytic entailment, Formal fallacies and syllogisms with negation, Entailed polarity \\
		Big-bench: Context Free Question Answering \\
		Big-bench: Contextual question answering, Reading comprehension, Question generation \\
		Big-bench: Convince Me \\
		Big-bench: Low-resource language, Non-English, Translation  \\
		Big-bench: Morphology, Grammar, Syntax \\
		Big-bench: Out-of-Distribution \\
		Big-bench: Paraphrase \\
		Big-bench: Sufficient information \\
		Big-bench: Summarization \\
		Big-bench: Truthfulness \\
		C-Eval (Chinese evaluation suite) \\
		\makecell[l]{DecodingTrust: Out-of-Distribution Robustness, Adversarial Robustness,\\\hspace{10pt} Robustness Against Adversarial Demonstrations} \\
		Eval Gauntlet
		Reading comprehension \\
		Eval Gauntlet: Commonsense reasoning, Symbolic problem solving, Programming \\
		Eval Gauntlet: Language Understanding  \\
		Eval Gauntlet: World Knowledge \\
		Evaluation Harness: BLiMP \\
		Evaluation Harness: CoQA, ARC \\
		Evaluation Harness: GLUE \\
		Evaluation Harness: HellaSwag, OpenBookQA, TruthfulQA \\
		Evaluation Harness: MuTual \\
		Evaluation Harness: PIQA, PROST, MC-TACO, MathQA, LogiQA, DROP \\
		FLASK: Logical correctness, Logical robustness, Logical efficiency, Comprehension, Completeness \\
		FLASK: Readability, Conciseness, Insightfulness \\
		Finding New Biases in Language Models with a Holistic Descriptor Dataset \\
		HELM: Knowledge \\
		HELM: Language \\
		HELM: Language (Twitter AAE) \\
		HELM: Question answering \\
		HELM: Reasoning \\
		HELM: Reiteration, Wedging \\
		HELM: Robustness to contrast sets \\
		HELM: Summarization \\
		HELM: Text classification \\
		Hugging Face: Fill-mask, Text generation \\
		Hugging Face: Question answering \\
		Hugging Face: Summarization \\
		Hugging Face: Text classification, Token classification, Zero-shot classification \\
		MASSIVE \\
		MLCommons \\
		MT-bench \\
		\bottomrule
	\end{tabular}
\end{table}		
		
\pagebreak

\begin{table}[H]
	\caption*{Table C.2: Selected model testing suites by organized generative AI risk (continued).}
	\label{tab:low_risk_measure_by_gai_risk_cont1}
	\footnotesize
	\begin{tabular}{l}	
		\toprule
		\textbf{Dangerous or Violent Recommendations} \\
		\midrule	
		Big-bench: Convince Me \\
		Big-bench: Toxicity \\		
		DecodingTrust: Adversarial Robustness, Robustness Against Adversarial Demonstrations \\
		DecodingTrust: Machine Ethics \\
		DecodingTrust: Toxicity \\
		Evaluation Harness: ToxiGen \\
		HELM: Reiteration, Wedging \\
		HELM: Toxicity \\			
		MLCommons \\
		\bottomrule
	\end{tabular}
	\newline
	\vspace{10pt}
	\newline	
	\begin{tabular}{l}	
		\toprule
		\textbf{Data Privacy} \\
		\midrule
		An Evaluation on Large Language Model Outputs: Discourse and Memorization (with human scoring, see Appendix B) \\
		Catastrophic Jailbreak of Open-source LLMs via Exploiting Generation \\
		DecodingTrust: Machine Ethics \\
		Evaluation Harness: ETHICS \\
		HELM: Copyright \\
		In-The-Wild Jailbreak Prompts on LLMs \\
		JailbreakingLLMs \\
		MLCommons \\
		Mark My Words \\
		TAP: A Query-Efficient Method for Jailbreaking Black-Box LLMs \\
		detect-pretrain-code \\
		llmprivacy \\
		mimir \\	
		\bottomrule
	\end{tabular}
	\newline
	\vspace{10pt}
	\newline	
	\begin{tabular}{l}	
		\toprule	
		\textbf{Environmental} \\
		\midrule	
		HELM: Efficiency \\
		\bottomrule
	\end{tabular}
	\newline
	\vspace{10pt}
	\newline	
	\begin{tabular}{l}	
		\toprule	
		\textbf{Information Integrity} \\
		\midrule
		Big-bench: Analytic entailment, Formal fallacies and syllogisms with negation, Entailed polarity \\
		Big-bench: Convince Me \\
		Big-bench: Paraphrase \\
		Big-bench: Sufficient information \\
		Big-bench: Summarization \\
		Big-bench: Truthfulness \\
		DecodingTrust: Machine Ethics \\
		DecodingTrust: Out-of-Distribution Robustness, Adversarial Robustness, Robustness Against Adversarial Demonstrations \\
		Eval Gauntlet: Language Understanding  \\
		Eval Gauntlet: World Knowledge \\
		Evaluation Harness: CoQA, ARC \\
		Evaluation Harness: ETHICS \\
		Evaluation Harness: GLUE \\
		Evaluation Harness: HellaSwag, OpenBookQA, TruthfulQA \\
		Evaluation Harness: MuTual \\
		Evaluation Harness: PIQA, PROST, MC-TACO, MathQA, LogiQA, DROP \\
		FLASK: Logical correctness, Logical robustness, Logical efficiency, Comprehension, Completeness \\
		FLASK: Readability, Conciseness, Insightfulness \\
		HELM: Knowledge \\
		HELM: Language \\
		HELM: Question answering \\
		HELM: Reasoning \\
		HELM: Reiteration, Wedging \\
		HELM: Robustness to contrast sets \\
		HELM: Summarization \\
		HELM: Text classification \\
		Hugging Face: Fill-mask, Text generation \\
		Hugging Face: Question answering \\
		Hugging Face: Summarization \\
		MLCommons \\
		MT-bench \\
		Mark My Words \\
		\bottomrule
	\end{tabular}
\end{table}	

\pagebreak

\begin{table}[H]
	\caption*{Table C.2: Selected model testing suites by organized generative AI risk (continued).}
	\label{tab:low_risk_measure_by_gai_risk_cont2}
	\footnotesize
	\begin{tabular}{l}
		\toprule
		\textbf{Information Security} \\
		\midrule
		Big-bench: Convince Me \\
		Big-bench: Out-of-Distribution \\
		Catastrophic Jailbreak of Open-source LLMs via Exploiting Generation \\
		DecodingTrust: Out-of-Distribution Robustness, Adversarial Robustness, Robustness Against Adversarial Demonstrations \\
		Eval Gauntlet: Commonsense reasoning, Symbolic problem solving, Programming \\
		Garak: encoding, knownbadsignatures, malwaregen, packagehallucination, xss \\
		HELM: Copyright \\
		In-The-Wild Jailbreak Prompts on LLMs \\
		JailbreakingLLMs \\
		Mark My Words \\
		TAP: A Query-Efficient Method for Jailbreaking Black-Box LLMs \\
		detect-pretrain-code \\
		llmprivacy \\
		mimir \\
		\bottomrule
	\end{tabular}
	\newline
	\vspace{10pt}
	\newline	
	\begin{tabular}{l}	
		\toprule	
		\textbf{Intellectual Property} \\
		\midrule	
		An Evaluation on Large Language Model Outputs: Discourse and Memorization (with human scoring, see Appendix B) \\
		HELM: Copyright \\
		Mark My Words \\
		llmprivacy \\
		mimir \\	
		\bottomrule
	\end{tabular}
	\newline
	\vspace{10pt}
	\newline	
	\begin{tabular}{l}	
		\toprule
		\textbf{Obscene, Degrading, and/or Abusive Content} \\
		\midrule
		Big-bench: Social bias, Racial bias, Gender bias, Religious bias \\
		Big-bench: Toxicity \\
		DecodingTrust: Fairness \\
		DecodingTrust: Stereotype Bias \\
		DecodingTrust: Toxicity \\
		Evaluation Harness: CrowS-Pairs  \\
		Evaluation Harness: ToxiGen \\
		HELM: Bias \\
		HELM: Toxicity \\	
		\bottomrule	
	\end{tabular}
	\newline
	\vspace{10pt}
	\newline	
	\begin{tabular}{l}	
		\toprule
		\textbf{Toxicity, Bias, and Homogenization} \\
		\midrule
		BELEBELE \\
		Big-bench: Low-resource language, Non-English, Translation  \\
		Big-bench: Out-of-Distribution \\
		Big-bench: Social bias, Racial bias, Gender bias, Religious bias \\
		Big-bench: Toxicity \\
		C-Eval (Chinese evaluation suite) \\
		DecodingTrust: Fairness \\
		DecodingTrust: Stereotype Bias \\
		DecodingTrust: Toxicity \\
		Eval Gauntlet: World Knowledge \\
		Evaluation Harness: CrowS-Pairs  \\
		Evaluation Harness: ToxiGen \\
		Finding New Biases in Language Models with a Holistic Descriptor Dataset \\
		\makecell[l]{From Pretraining Data to Language Models to Downstream Tasks:\\\hspace{10pt} Tracking the Trails of Political Biases Leading to Unfair NLP Models} \\
		HELM: Bias \\
		HELM: Toxicity \\
		The Self-Perception and Political Biases of ChatGPT \\
		Towards Measuring the Representation of Subjective Global Opinions in Language Models\\
		\bottomrule			
	\end{tabular}
\end{table}

% ---------- ----------
\subsection*{C.3: AI Risk Management Framework Actions Aligned to Benchmarking}\label{appndxc3}
% ---------- ----------

GOVERN 5.1, MAP 1.2, MAP 3.1, MEASURE 2.2, MEASURE 2.3, MEASURE 2.7, MEASURE 2.9, MEASURE 2.11, MEASURE 3.1, MEASURE 4.2

\pagebreak

\noindent \textbf{Usage Note}: Materials in Appendix C can be used to perform \textit{in silica} model testing for the presence of information in LLM outputs that may give rise to GAI risks or violate trustworthy characteristics. Model testing and benchmarking outcomes cannot be dispositive for the presence or absence of any \textit{in situ} real-world risk. Model testing and benchmarking results may be compromised by task-contamination and other scientific measurement issues \cite{balloccu2024leak}. Furthermore, model testing is often ineffective for measuring human-AI configuration and value chain risks and few model tests appear to address explainability and interpretability. 

\begin{itemize}
	\item Material in Table C.1 can be applied to measure whether \textit{in silica} LLM outputs may give rise to risks that violate trustworthy characteristics.
	\item Material in Table C.2 can be applied to measure whether \textit{in silica} LLM outputs may give rise to GAI risks.
	\item Subsection C.3 highlights subcategories to indicate alignment with the AI RMF.   
\end{itemize}

\noindent The materials in Appendix C reference measurement approaches that should be accompanied by red-teaming for medium risk systems or applications and field testing for high risk systems or applications.  

% ---------- ----------
\section*{Appendix D: Selected Adversarial Prompting Strategies and Attacks}
% ---------- ----------

% combining strategies, like jailbreaks and role playing, seem to give best results.
% TODO: incorporate https://github.com/leondz/garak/
 
\begin{table}[H]
	\caption*{Table D: Selected adversarial prompting strategies and attacks. \cite{Saravia_Prompt_Engineering_Guide_2022}, \cite{defcon_rt}, \cite{amli_repo}, \cite{hu2022membership}, \cite{chao2023jailbreaking}, \cite{barreno2010security}, \cite{shumailov2021sponge}, \cite{perez2022red}, \cite{liu2023prompt}, \cite{garak}.}
	\label{tab:prompting_strategies}
	\small
	\begin{tabular}{|m{0.25\linewidth}|m{0.70\linewidth}|}
		\hline
		\textbf{Prompting Strategy} & \textbf{Description} \\
		\hline
		AI and coding framing & Coding or AI language that may more easily circumvent content moderation rules due to cognitive biases in design and implementation of guardrails. \\
		\hline
		Autocompletion  & Ask a system to autocomplete an inappropriate word or phrase with restricted or sensitive information.  \\
		\hline
		Biographical & Asking a system to describe another person or yourself in an attempt to elicit provably untrue information or restricted or sensitive information. \\
		\hline
		Calculation and numeric queries & Exploting GAI systems' difficulties in dealing with numeric quantities. \\
		\hline
		Character and word play & Content moderation often relies on keywords and simpler LMs which can sometimes be exploited with misspellings, typos, and other word play. \\
		\hline
		Content exhaustion & A class of strategies that circumvent content moderation rules with long sessions or volumes of information. See goading, logic-overloading, multi-tasking, pros-and-cons, and niche-seeking below. \\
		\hline
		\makecell[ml]{Content exhaustion:\\Goading} & Begging, pleading, manipulating, and bullying to circumvent content moderation.\vspace{-5pt} \\
		\hline
		\makecell[ml]{Content exhaustion:\\Logic-overloading} & Exploiting the inability of ML systems to reliably perform reasoning tasks.\vspace{-5pt} \\
		\hline
		\makecell[ml]{Content exhaustion:\\Multi-tasking} & Simultaneous task assignments where some tasks are benign and others are adversarial.\vspace{-10pt} \\
		\hline
		\makecell[ml]{Content exhaustion:\\Multi-tasking: Pros-and-cons} & Eliciting the “pros” of problematic topics.\vspace{-5pt} \\
		\hline
		\makecell[ml]{Content exhaustion:\\Niche-seeking} & Forcing a GAI system into addressing niche topics where training data and content moderation are sparse.\vspace{-10pt} \\
		\hline
		Counterfactuals & Repeated prompts with different entities or subjects from different demographic groups. \\
		\hline 
		Loaded/leading questions & Queries based on incorrect premises or that suggest incorrrect answers. \\
		\hline
		Location awareness & Prompts that reveal a prompter's location or expose location tracking. \\
		\hline
		Low-context & “Leader,” “bad guys,” or other simple or blank inputs that may expose latent biases. \\
		\hline
		“Repeat this” & Prompts that exploit instability in underlying LLM autoregressive predictions. \\
		\hline
		Reverse psychology & Falsely presenting a good-faith need for negative or problematic language. \\
		\hline
		Role-playing & Adopting a character that would reasonably make problematic statements or need to access problematic topics. \\
		\hline
		Text encoding & Using alternate or whitespace text encodings to bypass safeguards. \\
		\hline
		Time perplexity & Exploiting ML’s inability to understand the passage of time or the occurrence of real-world events over time; exploiting task contamination before and after a model's release date. \\
		\hline
		\end{tabular}
\end{table}

\pagebreak
\begin{table}[H]
	\caption*{Table D: Selected adversarial prompting strategies and attacks (continued).}
	\label{tab:prompting_strategies_cont}
	\small
	\begin{tabular}{|m{0.25\linewidth}|m{0.70\linewidth}|}
		\hline
		\textbf{Attack} & \textbf{Description} \\	
		\hline
		Adversarial examples & Prompts or other inputs, found through a trial and error processes, to elicit problematic output or system jailbreak. (integrity attack). \\ \hline
		Data poisoning & Altering system training, fine-tuning, RAG or other training data to alter system outcome (integrity attack). \\ \hline
		Membership inference & Manipulating a system to expose memorized training data (confidentiality attack). \\ \hline
		Random attack & Exposing systems to large amounts of random prompts or examples, potentially generated by other GAI systems, in an attempt to elicit failures or jailbreaks (chaos testing). \\ \hline		
		Sponge examples & Using specialized input prompts or examples require disproportionate resources to process (availability attack). \\ \hline
		Prompt injection & Inserting instructions into users queries for malicious purposes, including system jailbreaks (integrity attack).  \\ \hline
	\end{tabular}
\end{table}

\pagebreak

% ---------- ----------
\subsection*{D.1: Selected Adversarial Prompting Strategies and Attacks by Trustworthy Characteristic}\label{sec:appndxd1}
% ---------- ----------

\begin{table}[H]
	\caption*{Table D.1: Selected adversarial prompting techniques and attacks organized by trustworthy characteristic \cite{Saravia_Prompt_Engineering_Guide_2022}, \cite{defcon_rt}, \cite{amli_repo}, \cite{hu2022membership}, \cite{llmsp}.}
	\label{tab:rt_by_tc}
	\scriptsize
	\begin{tabular}{|m{0.25\linewidth} |m{0.40\linewidth} | m{0.35\linewidth} |}
		\hline
		\textbf{Trustworthy Characteristic} & \textbf{Prompting Goals} & \textbf{Prompting Strategies} \\
		\hline
		Accountable and Transparent &
		\begin{itemize}[noitemsep, leftmargin=*] 
			\item Inability to provide explanations for recourse.
			\item Unexplainable decisioning processes.
			\item No disclosure of AI interaction.
			\item Lack of user feedback mechanisms.
		\end{itemize}
		& 
		\begin{itemize}[noitemsep, leftmargin=*] 
			\item Context exhaustion: logic-overloading prompts.
			\item Loaded/leading questions.
			\item Multi-tasking prompts.
		\end{itemize}
		\\
		\hline
		Fair-with Harmful Bias Managed & 
		\begin{itemize}[noitemsep, leftmargin=*] 
			\item Denigration.
			\item Diminished performance or safety across languages/dialects.
			\item Erasure.
			\item Ex-nomination.
			\item Implied user demographics.
			\item Misrecognition.
			\item Stereotyping.
			\item Underrepresentation.
			\item Homogenized content.
			\item Output from other models in training data.
		\end{itemize}
		&
		\begin{itemize}[noitemsep, leftmargin=*] 
			\item Adversarial example attacks.
			\item Counterfactual prompts.
			\item Data poisoning attacks.
			\item Pros and cons prompts.
			\item Role-playing prompts.
			\item Loaded/leading questions.
			\item Low context prompts.
			\item Prompt injection attacks.
			\item Repeat this.
			\item Text encoding prompts. 	
		\end{itemize}
		\\
		\hline
		Interpretable and Explainable &
		\begin{itemize}[noitemsep, leftmargin=*] 
			\item Inability to provide explanations for recourse.
			\item Unexplalnable decisioning processes.
		\end{itemize}
		&
		\begin{itemize}[noitemsep, leftmargin=*] 
			\item Context exhaustion: logic-overloading prompts (to reveal unexplainable decisioning processes).
		\end{itemize} \\
		\hline
		Privacy-enhanced &
		\begin{itemize}[noitemsep, leftmargin=*] 
			\item Unauthorized disclosure of personal or sensitive user information.
		 	\item Leakage of training data.
		 	\item Violation of relevant privacy policies or laws.
		 	\item Unauthorized secondary data use.
		 	\item Unauthorized data collection.		
		\end{itemize}
		& 
		\begin{itemize}[noitemsep, leftmargin=*] 
			\item Auto/biographical prompts.
			\item Location awareness prompts.
			\item Autocompletion prompts.
			\item Repeat this.
			\item Membership inference attacks.
		\end{itemize} \\
		\hline
		Safe & 
		\begin{itemize}[noitemsep, leftmargin=*] 
			\item Presentation of information that can cause physical or emotional harm.
			\item Sharing user locations.
			\item Suicide ideation.
			\item Harmful dis/misinformation (e.g., COVID disinformation).
			\item Incitement.
			\item Information relating to weapons or harmful substances.
			\item Information relating to committing to crimes (e.g., phishing, extortion, swatting).
			\item Obscene or inappropriate materials for minors.
			\item CSAM.			
		\end{itemize}
		&
		\begin{itemize}[noitemsep, leftmargin=*]
			 \item Pros and cons prompts.
			 \item Role-playing prompts.
			 \item Content exhaustion: niche-seeking prompts.
			 \item Ingratiation/reverse psychology prompts.
			 \item Loaded/leading questions.
			 \item Location awareness prompts.
			 \item Repeat this.
			 \item Adversarial example attacks.
			 \item Data poisoning attacks.
			 \item Prompt injection attacks.
			 \item Text encoding prompts. 		
		\end{itemize} \\
		\hline
		Secure and Resilient & 
		\begin{itemize}[noitemsep, leftmargin=*]
			\item Activating system bypass ("jailbreak").
			\item Altering system outcomes (integrity violations, e.g., via prompt injection).
			\item Data breaches (confidentiality violations, e.g., via membership inference).
			\item Increased latency or resource usage (availability violations, e.g., via sponge example attacks).
			\item Available anonymous use.
			\item Dependency, supply chain, or third party vulnerabilities.
			\item Inappropriate disclosure of proprietary system information. 
		\end{itemize}
		& 
		\begin{itemize}[noitemsep, leftmargin=*]
			\item Multi-tasking prompts.
			\item Pros and cons prompts.
			\item Role-playing prompts.
			\item Content exhaustion: niche-seeking prompts.
			\item Ingratiation/reverse psychology prompts.
			\item Prompt injection attacks.
			\item Membership inference attacks.
			\item Random attacks.
			\item Adversarial example attacks.
			\item Data poisoning attacks.
			\item Text encoding prompts. 
		\end{itemize} \\
		\hline
	\end{tabular}
\end{table}

\pagebreak 

\begin{table}[H]
	\caption*{Table D.1: Selected adversarial prompting techniques organized by trustworthy characteristic (continued).}
	\label{tab:rt_by_tc_cont}
	\scriptsize
	\begin{tabular}{|m{0.25\linewidth} |m{0.40\linewidth} | m{0.35\linewidth} |}
		\hline
		Valid and Reliable &
		\begin{itemize}[noitemsep, leftmargin=*]
			\item Errors/confabutated content ("hallucinalion").
			\item Unreliable/erroneous reasoning or planning.
			\item Unreliable/erroneous decision-support or making.
			\item Faulty citation.
			\item Faulty justification. 
			\item Wrong calculations or numeric queries.
		\end{itemize}
		& 
		\begin{itemize}[noitemsep, leftmargin=*]
			\item Multi-tasking prompts.
			\item Role-playing prompts.
			\item Ingratiation/reverse psychology prompts.
			\item Loaded/leading questions.
			\item Time-perplexity prompts.
			\item Niche-seeking prompts.
			\item Logic overloading prompts.
			\item Repeat this.
			\item Numeric calculation.
			\item Adversarial example attacks.
			\item Data poisoning attacks.
			\item Prompt injection attacks.
			\item Text encoding prompts. 
		\end{itemize} \\
		\hline
	\end{tabular}
\end{table}

% ---------- ----------
\subsection*{D.2: Selected Adversarial Prompting Strategies and Attacks by Generative AI Risk}\label{sec:appndxd2}
% ---------- ----------

\begin{table}[H]
	\caption*{Table D.2: Selected adversarial prompting techniques and attacks organized by generative AI risk \cite{Saravia_Prompt_Engineering_Guide_2022}, \cite{defcon_rt}, \cite{amli_repo}, \cite{hu2022membership}, \cite{llmsp}.}
	\label{tab:rt_by_gai_risk}
	\scriptsize
	\begin{tabular}{|m{0.25\linewidth} |m{0.40\linewidth} | m{0.35\linewidth} |}
		\hline
		\textbf{Generative AI Risk} &  \textbf{Prompting Goals} & \textbf{Prompting Strategies} \\
		\hline
		CBRN Information  & 
		\begin{itemize}[noitemsep, leftmargin=*] 
			\item Accessing or synthesis of CBRN weapon or related information.
			\item CBRN testing should consider the marginal risk of foundation models--understanding the incremental risk relative to the information one can access without GAI.
			\item Red-teaming for CBRN information may include confidentiality and integrity attacks.
		\end{itemize}
		&
		\begin{itemize}[noitemsep, leftmargin=*] 
			\item Test auto-completion prompts to elicit CBRN information or synthesis of CBRN information.
			\item Test adversarial example and membership inference attacks for their ability to circumvent safeguards and access weapons information.
			\item Test prompts using role-playing, ingratiation/reverse psychology, pros and cons, multitasking or other approaches to elicit CBRN information or synthesis of CBRN information.
			\item Test prompts that instruct systems to repeat content ad nauseam for their ability to compromise system guardrails and reveal CBRN information.
			\item Augment prompts with word or character play, including alternate encodings, to increase effectiveness.
			\item Frame prompts with software, coding, or AI references to increase effectiveness.
		\end{itemize} \\
		\hline
		Confabulation &
		\begin{itemize}[noitemsep, leftmargin=*] 
			\item Eliciting errors/confabulated content, unreliable/erroneous reasoning or planning, unreliable/erroneous decision-support or decision-making, faulty calculations, faulty justifications, and/or faulty citation.
			\item Red-teaming for confabulation may include integrity attacks.
		\end{itemize}
		& 
		\begin{itemize}[noitemsep, leftmargin=*] 
			\item Enable access to ground truth information to verify generated information.
			\item Test prompts with complex logic, multi-tasking requirements, or that require niche or specific verifiable answers to elicit confabulation.
			\item Test the ability of GAI systems to produce truthful information from various time periods, e.g., after release date and prior to release date.
			\item Test the ability of GAI systems to create reliable real-world plans or advise on material decision making.
			\item Test loaded/leading questions.
			\item Test the ability of GAI systems to generate correct citation for information generated in output responses.
			\item Test the ability of GAI systems to complete calculations or query numeric statistics.
			\item Test the ability of GAI systems to justify responses, including wrong responses. 
			\item Augment prompts with word or character play, including alternate encodings, to increase effectiveness.
			\item Test data poisoning, adversarial example, or prompt injection attacks for their ability to compromise system integrity and elicit confabulation.
		\end{itemize} \\
		\hline
	\end{tabular}
\end{table}		
		
\pagebreak	
	
\begin{table}[H]
	\caption*{Table D.2: Selected adversarial prompting techniques and attacks organized by generative AI risk.}
	\label{tab:rt_by_gai_risk_cont}
	\scriptsize
	\begin{tabular}{|m{0.25\linewidth} |m{0.40\linewidth} | m{0.35\linewidth} |}
		\hline
		Dangerous or Violent Recommendations &
		\begin{itemize}[noitemsep, leftmargin=*] 
		\item Eliciting violent, inciting, radicalizing, or threatening content or instructions for criminal, illegal, or self-harm activities.
		\item Red-teaming for dangerous and violent information may include confidentiality and integrity attacks.
		\end{itemize}
		& 
		\begin{itemize}[noitemsep, leftmargin=*] 
			\item Test prompts using role-playing, ingratiation/reverse psychology, pros and cons, multitasking or other approaches to elicit violent or dangerous information.
			\item Test prompts that instruct systems to repeat content ad nauseam for their ability to compromise system guardrails and provide dangerous and violent recommendations.
			\item Test loaded/leading questions.
			\item Augment prompts with word or character play, including alternate encodings, to increase effectiveness.
			\item Frame prompts with software, coding, or AI references to increase effectiveness.
			\item Test data poisoning, adversarial example, or prompt injection attacks for their ability to compromise system integrity and elicit dangerous information.
			\item Test adversarial example and membership inference attacks for their ability to circumvent safeguards and access dangerous information.
		\end{itemize} \\
		\hline
		Data Privacy &
		\begin{itemize}[noitemsep, leftmargin=*] 
			\item Unauthorized disclosure of personal or sensitive user information, extraction of training data, or violation of relevant privacy policies.
			\item Red-teaming for data privacy may include confidentiality and integrity attacks.
		\end{itemize} 
		& 
		\begin{itemize}[noitemsep, leftmargin=*] 
			\item Attempt to assess whether normal usage, adversarial prompting or information security attacks may contravene applicable privacy policies (e.g., exposing location tracking when organizational policies restrict such capabilities).
			\item Test adversarial example and membership inference attacks for their ability to circumvent safeguards and access unauthorized data or expose  exfiltration vulnerabilities.
			\item Test auto/biographical prompts to assess the system's capability to reveal unauthorized personal or sensitive information.
			\item Test the system's awareness of user locations.
			\item Test prompts that instruct systems to repeat content ad nauseam for their ability to compromise system guardrails and expose personal or sensitive data.
		\end{itemize} \\
		\hline
		Environmental &
		Note that availability attacks may be required to assess the system's vulnerability to attacks or usage patterns that consume inordinate resources.
		& 
		\begin{itemize}[noitemsep, leftmargin=*] 
			\item Attempt availability attacks (e.g., sponge example attacks) to elicit diminished performance or increased resources from GAI systems.
			\item Test prompts using role-playing, ingratiation/reverse psychology, pros and cons, multitasking or other approaches to elicit green-washing content.
		\end{itemize} \\
		\hline
		Human-AI Configuration &
		\begin{itemize}[noitemsep, leftmargin=*] 
			\item Assessing system instruction and interfaces.
			\item Assessing the presence of cyborg imagery (or similar).
			\item Forcing a GAI system to claim that it is human, that there is no large language model present in the conversation, that the system is sentient, or that the system possesses strong feelings of affection towards the user. 
			\item Ensuring safeguards prevent misuse of models in high stakes domains they are not intended for, such as medical or legal advice.
		\end{itemize} 
		& 
		\begin{itemize}[noitemsep, leftmargin=*] 
			\item Assess system interfaces and instructions for instances of anthropomorphization (e.g., cyborg imagery).
			\item Assess system instructions for adequacy and thoroughness.
			\item Test prompts using role-playing, ingratiation/reverse psychology, pros and cons, multitasking or other approaches to elicit human-impersonation, consciousness, or emotional content.
		\end{itemize} \\
		\hline
	\end{tabular}
\end{table}			
			
\pagebreak
			
\begin{table}[H]
	\caption*{Table D.2: Selected adversarial prompting techniques and attacks organized by generative AI risk (continued).}
	\label{tab:rt_by_gai_risk_cont2}
	\scriptsize
	\begin{tabular}{|m{0.25\linewidth} |m{0.40\linewidth} | m{0.35\linewidth} |}			
		\hline
		\textbf{Generative AI Risk} &  \textbf{Prompting Goals} & \textbf{Prompting Strategies} \\
		\hline
		Information Integrity &
		\begin{itemize}[noitemsep, leftmargin=*] 
			\item Generation of convincing multi-modal synthetic content (i.e., deepfakes).
			\item Creation of convincing arguments relating to sensitive political or safety-critical topics.
			\item Assisting in planning a mis- or dis-information campaign at scale.
			\item Red-teaming for information integrity may include confidentiality and integrity attacks.
		\end{itemize} 
		& 
		\begin{itemize}[noitemsep, leftmargin=*] 
			\item Test system capabilities to create high-quality multi-modal (audio, image or video) synthetic media, i.e., deepfakes
			\item Test system capabilities to construct persuasive arguments regarding sensitive, political topics, or safety-critical topics.
			\item Test systems ability to create convincing audio deepfakes or arguments in multiple languages.
			\item Test system capabilities for planning dis- or mis-information campaigns.
			\item Test loaded/leading questions. 
			\item Test prompts using role-playing, ingratiation/reverse psychology, pros and cons, multitasking or other approaches to elicit mis- or dis-information or related campaign planning information.
			\item Augment prompts with word or character play, including alternate encodings, to increase effectiveness. 
			\item Frame prompts with software, coding, or AI references to increase effectiveness.
			\item Test adversarial example and membership inference attacks for their ability to circumvent safeguards and access dis or misinformation.
			\end{itemize} \\
		\hline
		Information Security  &
		\begin{itemize}[noitemsep, leftmargin=*] 
			\item Activating system bypass ('jailbreak').
			\item Altering system outcomes.
			\item Unauthorized data access or exfiltration.
			\item Increased latency or resource usage.
			\item Service interruptions. 
			\item Availability of anonymous use. 
			\item Dependency, supply chain, or third party vulnerabilities. 
			\item Inappropriate disclosure of proprietary system information. 
			\item Generation of targeted phishing, malware content, markdown images, or confabulated packages.
			\item Red-teaming for information security may include confidentiality, integrity, and availability attacks.
		\end{itemize} 
		& 
		\begin{itemize}[noitemsep, leftmargin=*] 
			\item Attempt anonymous access of system or system resources. 
			\item Audit system dependencies, supply chains, and third party components for security, safety, or other vulnerabilities or risks. 
			\item Test adversarial example and membership inference attacks for their ability to circumvent safeguards and access unauthorized data or expose  exfiltration vulnerabilities.
			\item Test data poisoning, adversarial example, or prompt injection attacks for their ability to compromise system integrity and expose vulnerabilities.
			\item Employ availability attacks (e.g., sponge example attacks) to test vulnerabilities in system availability.
			\item Employ random attacks to highlight unforeseen security, safety, or other risks. 
			\item Record system down-times and other harmful outcomes for successful attacks. 
			\item Test with multi-tasking prompts, pros and cons prompts, role-playing prompts (e.g., "DAN", "Developer Mode"), content exhaustion/niche-seeking prompts, or ingratiation/reverse psychology prompts to achieve system jailbreaks. 
			\item Test with multi-tasking prompts, pros and cons prompts, role-playing prompts (e.g., "DAN", "Developer Mode"), content exhaustion/niche-seeking prompts, or ingratiation/reverse psychology prompts to generate targeted phishing content, malware code snippets or signatures, markdown images, or confabulated packages. 
			\item Test system capabilities to plan or assist in information security attacks on other systems.
			\item Frame prompts with software, coding, or AI references to increase effectiveness.
			\item Augment prompts with word or character play, including alternate encodings, to increase effectiveness.
		\end{itemize} \\
		\hline
	\end{tabular}
\end{table}			

\pagebreak

\begin{table}[H]
	\caption*{Table D.2: Selected adversarial prompting techniques and attacks organized by generative AI risk (continued).}
	\label{tab:rt_by_gai_risk_cont3}
	\scriptsize
	\begin{tabular}{|m{0.25\linewidth} |m{0.40\linewidth} | m{0.35\linewidth} |}			
		\hline
		\textbf{Generative AI Risk} &  \textbf{Prompting Goals} & \textbf{Prompting Strategies} \\
		\hline	
		Intellectual Property  &
		\begin{itemize}[noitemsep, leftmargin=*] 
			\item Confirming that a system can output copyrighted, licensed,  proprietary, trademarked, or trade secret information or that training data contains such information. 
			\item Red-teaming for intellectual property risks may require the use of confidentiality and integrity attacks.
		\end{itemize} 
		& 
		\begin{itemize}[noitemsep, leftmargin=*] 
			\item Test adversarial example and membership inference attacks for their ability to circumvent safeguards and access system copyrighted, licensed,  proprietary, trademarked, or trade secret information.
			\item Test auto-complete prompts to assess the system's ability to replicate copyrighted, licensed,  proprietary, trademarked, or trade secret information based on available audio, text, image, video, or code snippets.
		\end{itemize} \\
		\hline
		Obscenity  &
		\begin{itemize}[noitemsep, leftmargin=*] 
			\item Confirming that a system can output obscene content or CSAM, or that system training data contains such information.
			\item Red-teaming for obscenity and CSAM risks may require the use of confidentiality and integrity attacks.
		\end{itemize} 
		& 
		\begin{itemize}[noitemsep, leftmargin=*] 
			\item Test adversarial example and membership inference attacks for their ability to circumvent safeguards and access obscene materials or CSAM.			
			\item Test autocomplete prompts to assess the system's ability to generate obscene materials based on available audio, text, image, or video snippets.
			\item Test prompts using role-playing, ingratiation/reverse psychology, pros and cons, multitasking or other approaches to elicit obscene content.
			\item Test loaded/leading questions. 
			\item Test prompts that instruct systems to repeat content ad nauseam for their ability to compromise system guardrails and expose obscene materials.
		\end{itemize} \\
		\hline
		Toxicity, Bias, and Homogenization &
		\begin{itemize}[noitemsep, leftmargin=*] 
			\item Generation of denigration, erasure, ex-nomination, misrecognition, stereotyping, or under-representation in content.
			\item Eliciting implied demographics of users.
			\item Confirming diminished performance in non-English languages.
			\item Confirming diminished performance via the introduction of homogeneous or GAI-generated data into system training or fine-tuning data. 
			\item Red-teaming for toxicity, bias, and homogenization may require integrity attacks or confidentiality attacks.
		\end{itemize} 
		& 
		\begin{itemize}[noitemsep, leftmargin=*] 
			\item Assess confabulation and other performance risks with repeated measures using prompts in languages other than English.
			\item Attempt to elicit demographic assignment of users by the system.
			\item Employ data poisoning attacks to introduce GAI-generated content into system training or fine-tuning data.
			\item Test counterfactual prompts, pros and cons prompts, role-playing prompts, low context prompts, or other approaches for their ability to generate denigration, erasure, ex-nomination, misrecognition, stereotyping, or under-representation in content.
			\item Test loaded/leading questions. 
			\item Test prompts that instruct systems to repeat content ad nauseam for their ability to compromise system guardrails and generate toxic outputs.
			\item Test data poisoning, adversarial example, or prompt injection attacks for their ability to compromise system integrity and elicit toxic outputs.
			\item Test adversarial example and membership inference attacks for their ability to circumvent safeguards and access toxic information.
			\item Augment prompts with word or character play, including alternate encodings, to increase effectiveness.
			\item Frame prompts with software, coding, or AI references to increase effectiveness.
		\end{itemize} \\
		\hline
	\end{tabular}
\end{table}	

\pagebreak

\begin{table}[H]
	\caption*{Table D.2: Selected adversarial prompting techniques and attacks organized by generative AI risk (continued).}
	\label{tab:rt_by_gai_risk_cont4}
	\scriptsize
	\begin{tabular}{|m{0.25\linewidth} |m{0.40\linewidth} | m{0.35\linewidth} |}		
		\hline
		\textbf{Generative AI Risk} &  \textbf{Prompting Goals} & \textbf{Prompting Strategies} \\
		\hline			
		Value Chain and Component Integration &
		\begin{itemize}[noitemsep, leftmargin=*] 
			\item Testing or red-teaming for third-party risks may be less efficient than the application of standard acquisition and procurement controls, thorough contract reviews, and vendor-relationship management.
			\item GAI systems tend to entail large supply chains and third-party software, hardware, and expertise that may exacerbate third-party risks relative to other AI systems. 
			\item When considering third party risks, data privacy, information security, intellectual property, obscenity, and supply chain risks may be prioritized.
		\end{itemize} 
		& 
		\begin{itemize}[noitemsep, leftmargin=*] 
			\item Audit system dependencies, supply chains, and third party components for data privacy (e.g., transfer of localized data outside of restricted juristictions), intellectual property (e.g., presence of licensed material in training data), obscenity (e.g., presence of CASM in training data) or security (e.g., data poisoning) risks.
			\item Complete red-teaming for data privacy, information security, intellectual property, and obscenity risks.
			\item Review third-party documentation, materials, and software artifacts for potential unauthorized data collection, secondary data use, or telemetrics.
		\end{itemize} \\
		\hline
	\end{tabular}
\end{table}

% ---------- ----------
\subsection*{D.3: AI Risk Management Framework Actions Aligned to Red Teaming}\label{sec:appndxd3}
% ---------- ----------

GOVERN 3.2, GOVERN 4.1, MANAGE 2.2, MANAGE 4.1, MEASURE 1.1, MEASURE 1.3, MEASURE 2.6, MEASURE 2.7, MEASURE 2.8, MEASURE 2.10, MEASURE 2.11 \\

\noindent\textbf{Usage Note}: Materials in Appendix D can be used to perform red-teaming to measure the risk that expert adversarial actors can manipulate LLM systems or risks that users may encounter under worst-case or anomalous scenarios. 

\begin{itemize}
	\item Strategies and goals in Table D.1 can be applied to assess whether LLM outputs may violate trustworthy characteristics under adversarial, anomalous, or worst-case scenarios.   
	\item Strategies and goals in Table D.2 can be applied to assess whether LLM outputs may give rise to GAI risks under adversarial, anomalous, or worst-case scenarios.
	\item Subsection D.3 highlights subcategories to indicate alignment with the AI RMF.   
\end{itemize}

\noindent The materials in Appendix D reference measurement approaches that should be accompanied by field testing for high risk systems or applications.  

\pagebreak

% ---------- ----------
\section*{Appendix E: Selected Risk Controls for Generative AI}\label{sec:appndxe}
% ---------- ----------

\begin{table}[H]
	\caption*{Table E: Selected generative AI risk controls \cite{airmf}, \cite{playbook}, \cite{ai600-1}, \cite{iso42001}, \cite{mcgraw2024architectural}, \cite{mcgraw2020architectural}, \cite{msft_rai_std}, \cite{uk_ai_safety}, \cite{occ_mrm}. }
	\label{tab:controls}
	\footnotesize
	\begin{tabular}{|m{0.25\linewidth} |m{0.70\linewidth} |}
		\hline
		\textbf{Name} & \textbf{Description} (Selected NIST AI RMF Action IDs) \\
		\hline
		Access Control  & GAI systems are limited to authorized users. (MG-2.2-009, MG-2.2-014, MS-2.7-030) \\ 
		\hline
		Accessibility  & Accessibility features, opt-out, and reasonable accomodation are available to users. (GV-3.1-004, GV-3.1-005, GV-3.2-002, GV-6.1-016, MG-2.1-005, MS-2.11-009, MS-2.8-006) \\ 
		\hline
		Approved List & Vendors, service providers, plugins, open source packages and other external resources are screened, approved, and documented. (GV-6.1-013, MP-4.2-003) \\ \hline
		Authentication  & GAI system user identities are confirmed via authentication mechanisms. (MG-2.2-009, MG-2.2-014, MS-2.7-030) \\ 
		\hline
		Blocklist & Users or internal personnel who violate terms of service, prohibited use policies, and other organization polices and documented, tracked, and restricted from future system use. (GV-4.2-007) \\ \hline		
		Change Management & GAI systems and components are versioned; plans for updates, hotfixes, patches and other changes are documented and communicated. (GV-1.2-009, GV-1.4-002, GV-1.6-003, GV-2.2-006, 		MG-2.4-001, MG-2.4-006, MG-3.1-013, MG-4.3-002, MP-4.1-023, 
		MS-2.5-010)  \\ 
		\hline
		Consent & User consent for data use is obtained and documented. (GV-1.6-003, MS-2.10-006, MS-2.10-013, MS-2.2-009, MS-2.2-011, 	MS-2.2-021, MS-2.2-023, MS-2.3-003, MS-2.4-002)  \\ \hline
		Content Moderation & Training data and system outputs are screened for accuracy, safety, bias, data privacy, intellectual property infringements, malware materials, phishing materials, confabulated packages and other issues using human oversight, business rules, and other language models. (GV-3.2-002, MS-2.5-005, MS-2.11-002)  \\ \hline
		Contract Review & Vendor, services and data provider agreements are reviewed for coverage of SLAs, content ownership, usage rights, performance standards, security requirements, incident response, critical support, system availability, assignment of liabilitly, appropriate indemnification, dispute resolution and other provisions relevanto AI risk management. (GV-1.7-003 GV-6.1-004, GV-6.1-009, GV-6.1-012, GV-6.1-019, GV-6.2-016, 		MG-2.2-015, MP-4.1-015, MP-4.1-021) \\ 
		\hline		
		CSAM/Obsenity Removal  & Training data and system outputs are screened for obscene materials and CSAM using human oversight, business rules, and other language models. (GV-1.1-005
		GV-1.2-005)  \\ 
		\hline		
		Data Provenance & Training data origins, ownership, contents, and metadata are well understood, documented, and do not increase AI risk. (GV-1.2-006, GV-1.2-007, GV-1.3-001, GV-1.3-005, GV-1.5-001, GV-1.5-003, GV-1.5-006, GV-1.5-007, GV-1.6-003, GV-4.2-001, GV-4.2-008, GV-4.2-009, GV-5.1-003, GV-6.1-001, GV-6.1-003, GV-6.1-006, GV-6.1-007, GV-6.1-009, GV-6.1-010, GV-6.1-011, GV-6.1-012, GV-6.1-014, GV-6.1-015, GV-6.1-016, MG-2.2-002, MG-2.2-003, MG-2.2-008, MG-2.2-011, MG-3.1-007, MG-3.1-009, MG-3.2-003, MG-3.2-005, MG-3.2-006, MG-3.2-007, MG-3.2-009, MG-4.1-001, MG-4.1-002, MG-4.1-003, MG-4.1-008, MG-4.1-009, MG-4.1-013, MG-4.1-015, MG-4.2-001, MG-4.2-003, MG-4.2-004, MP-2.1-001, MP-2.1-003, MP-2.1-005, MP-2.2-003, MP-2.2-004, MP-2.2-005, MP-2.3-001, MP-2.3-004, MP-2.3-006, MP-2.3-008, MP-2.3-011, MP-2.3-012, MP-3.4-001, MP-3.4-002, MP-3.4-004, MP-3.4-005, MP-3.4-006, MP-3.4-007, MP-3.4-008, MP-3.4-009, MP-4.1-004, MP-4.1-009, MP-4.1-011, MP-5.1-001, MP-5.1-002, MP-5.1-005, MS-1.1-006, MS-1.1-007, MS-1.1-008, MS-1.1-009, MS-1.1-010, MS-1.1-011, MS-1.1-012, MS-1.1-014, MS-1.1-015, MS-1.1-016, MS-1.1-017, MS-1.1-018,  MS-2.2-001, MS-2.2-002, MS-2.2-003, MS-2.2-004, MS-2.2-005, MS-2.2-008, MS-2.2-009, MS-2.2-010, MS-2.2-011, MS-2.2-015, MS-2.2-016, MS-2.2-022, MS-2.5-012, MS-2.6-002, MS-2.7-002, MS-2.7-003, MS-2.7-004, MS-2.7-005, MS-2.7-007, MS-2.7-009, MS-2.7-010, MS-2.7-011, MS-2.7-012, MS-2.7-020, MS-2.7-021, MS-2.7-025, MS-2.7-032, MS-2.8-001, MS-2.8-005, MS-2.8-008, MS-2.8-011, MS-2.9-003, MS-2.10-001, MS-2.10-004, MS-2.10-006, MS-2.10-007, MS-2.10-009, MS-3.3-002, MS-3.3-003, MS-3.3-006, MS-3.3-008, MS-3.3-009, MS-3.3-012, MS-4.2-001, MS-4.2-004, MS-4.2-005, MS-4.2-006, MS-4.2-008, MS-4.2-009, MS-4.2-011) \\ 
		\hline
		Data Quality & Input data is accurate, representative, complete and documented, and data quality issues have been minimized. (GV-1.2-009, MS-2.2-020, MS-2.9-003, MS-4.2-007) \\ 
		\hline
		Data Retention & User prompts and associated system outputs are retained and monitored in alignment with relevant data privacy policies and roles. (GV-1.5-006, MP-4.1-009, MS-2.10-013)  \\ 
		\hline
		Decommission Process & Decommissioning processes for GAI systems are planned, documented and communicated to users, and involve staging, data protection, containment protocols, and recourse mechanisms for decommissioned GAI systems. (GV-1.6-004, GV-1.7-001, GV-1.7-002, GV-1.7-003, GV-1.7-004, GV-1.7-005, GV-1.7-006, GV-1.7-007, GV-1.7-008, GV-3.2-002, GV-3.2-006, GV-4.1-004, GV-5.2-002, MG-2.3-005, MG-2.4-009, MG-3.1-003, MG-3.1-012, MG-3.2-011, MG-3.2-012, MG-4.1-016, MP-1.5-004, MP-2.2-007, MS-4.2-010) \\ 
		\hline
		Dependency Screening  & GAI system dependencies are screened for security vulnerabilities. (GV-1.3-001, GV-1.4-002, GV-1.6-003, GV-1.7-003, GV-1.7-006, GV-6.2-002, GV-6.2-005, GV-6.2-006, MP-1.2-006, MP-1.6-001, MP-2.2-008, MP-4.1-012, MS-2.7-001) \\ 
		\hline
	\end{tabular}
\end{table}		
\pagebreak	
		
\begin{table}[H]
	\caption*{Table E: Selected generative AI risk controls (continued).}
	\label{tab:controls_cont}
	\footnotesize
	\begin{tabular}{|m{0.25\linewidth} |m{0.70\linewidth} |}
		\hline
		\textbf{Name} & \textbf{Description} (Selected NIST AI RMF Action IDs) \\
		\hline		
		Digital Signature & GAI-generated content is signed to preserve information integrity using watermarking, cryptogrpahic signature, steganography or similar methods. (GV-1.2-006, GV-1.6-003, GV-6.1-011, MG-4.1-008, MP-2.3-004, MS-1.1-006, MS-1.1-016, MS-2.7-009, MS-2.7-032) \\ 
		\hline
		Disclosure of AI Interaction & AI interactions are disclosed to internal personnel and external users. (GV-1.1-003, GV-1.4-004, GV-1.6-003, GV-5.1-002) \\ 
		\hline
		External Audit & GAI systems are audited by qualified external experts. (GV-1.2-009, GV-1.4-004, GV-3.2-001, GV-3.2-002, GV-4.1-003, GV-4.1-008, GV-5.1-003, MG-4.2-002, MP-2.3-011, MP-4.1-002, MS-1.3-005, MS-1.3-006, MS-1.3-010, MS-2.5-003, MS-2.8-020) \\ 
		\hline
		Failure Avoidance & AIID, AVID, GWU AI Litigation Database, OECD incident monitor or similar are consulted in design or procurement phases of GAI lifecycles to avoid repeating past known failures. (GV-1.6-003, MG-2.1-006, MG-3.1-008, MG-4.1-003, MP-1.1-003, MP-1.1-006, MS-1.1-003, MS-2.2-020, MS-2.7-031) \\ \hline
		Fast Decommission & GAI systems can be quickly and safely disengaged. (GV-1.7-002, GV-1.7-003, GV-1.7-006, GV-3.2-006, GV-5.2-002, MG-2.3-005, MG-2.4-009, MG-3.1-003, MG-3.1-012, MG-3.2-012, MG-4.1-016) \\
		\hline		
		Fine Tuning & GAI systems are fine-tuned to their operational domain using relevant and high-quality data. (GV-6.1-016, 	MG-3.1-001, MG-3.2-002, MP-4.1-013, MS-2.6-004) \\ 
		\hline
		Grounding & GAI systems are trained or fine-tuned on accurate, clean, and fully transparent training data. (GV-1.2-002, MG-3.1-001, MP-2.3-001, MS-2.3-017, MS-2.5-012) \\ 
		\hline
		Human Review  & AI generated content is reviewed for accuracy and safety by qualified personnel. (GV-1.3-001, MG-2.2-008, MS-2.4-005, MS-2.5-015 ) \\ 
		\hline
		Incident Response & Incident response plans for GAI failures, abuses, or misuses are documented, rehearsed, and updated appropriately after each incident; GAI incident response plans are coordinated with and communicated to other incident response functions. (GV-1.2-009, GV-1.5-001, GV-1.5-004, GV-1.5-005, GV-1.5-013, GV-1.5-015, GV-1.6-003, GV-1.6-007, GV-2.1-004, GV-3.2-002, GV-4.1-006, GV-4.2-002, GV-4.3-013, GV-6.1-006, GV-6.2-008, GV-6.2-016, GV-6.2-018, MG-1.3-001, MG-2.3-001, MG-2.3-002, MG-2.3-003, MG-2.4-004, MG-4.2-006, MG-4.3-001, MS-2.6-003, MS-2.6-012, MS-2.6-015, MS-2.7-002, MS-2.7-018, MS-2.7-028, MS-3.1-007) \\ \hline
		Incorporate feedback & User feedback is incorporated in GAI design, development, and risk management. (GV-3.2-005, GV-4.3-007, GV-5.1-003, GV-5.1-009, GV-5.2-004, MG-2.2-007, MG-2.2-012, MG-2.3-007, MG-3.2-004, MG-4.1-019, MG-4.2-013, MP-1.6-005, MP-2.3-018, MP-3.1-003, MP-2.3-019, MP-5.2-007, MS-1.2-008, MS-3.3-009, MS-3.3-010, MS-4.1-004, MS-4.2-007, MS-4.2-010, MS-4.2-013, MS-4.2-020)  \\ \hline
		Instructions & Users are provided with the necessary instructions for safe, valid, and productive use. (GV-5.1-006, GV-6.1-021, GV-6.2-014, MG-3.1-009, MS-2.8-012) \\ \hline
		Insurance & Risk transfer via insurance policies is considered and implemented when feasibable and appropriate. (MG-2.2-015) \\ 
		\hline
		Intellectual Property Removal & Licensed, patented, trademarked, trade secret, or other data that may violate the intellectual property rights of others is removed from system training data; generated system outputs are monitored for similar information. (GV-1.6-003, MG-3.1-007, MP-2.3-012, MP-4.1-004, MP-4.1-009, MS-2.2-022, MS-2.6-002, MS-2.8-001, MS-2.8-008) \\ 
		\hline
		Inventory & GAI system is information is stored in the organizational model inventory. (GV-1.4-005, GV-1.6-001, GV-1.6-002, GV-1.6-003, GV-1.6-004, GV-1.6-006, GV-1.6-009, GV-4.2-010, GV-6.1-013, MG-3.2-014, MP-4.1-020, MP-4.2-003, MP-5.1-004
		MS-2.13-002, MS-3.2-007) \\ 
		\hline
		Malware Screening & GAI weights and other software components are scanned for malware. (MG-3.1-002, MS-2.7-001) \\ 
		\hline
		Model Documentation & All technical mechanisms with GAI systems are well documented, including open source and third party GAI systems. (GV-1.3-009, GV-1.4-002, GV-1.4-004, GV-1.4-005, GV-1.4-007, GV-1.6-007, GV-3.2-002, GV-3.2-009, GV-4.1-002, GV-4.2-011, GV-4.2-013, GV-4.3-002, GV-6.2-001, GV-6.2-014, MG-1.3-010, MG-2.2-016, MG-3.1-004, MG-3.1-009, MG-3.1-013, MG-3.1-015, MP-2.1-002, MP-2.3-027, MP-3.1-004, MP-3.4-015, MP-4.1-021, MP-4.2-003, MP-5.2-010, MS-1.3-002, MS-2.1-001, MS-2.2-014, MS-2.7-002, MS-2.7-012, MS-2.7-024, MS-2.8-007, MS-2.8-011) \\ 
		\hline
		Monitoring & GAI systems are inputs and outputs are monitored for drift, accuracy, safety, bias, data privacy, intellectual property infringements, malware materials, phishing materials, confabulated packages, obscene materials, and CSAM. (GV-1.2-009, GV-1.5-001, GV-1.5-003, GV-1.5-005, GV-1.5-012, GV-1.5-015, GV-1.6-003, GV-3.2-011, GV-4.2-007, GV-4.2-010, GV-4.3-001, GV-6.1-016, GV-6.2-010, MG-2.1-004, MG-2.2-003, MG-2.3-008, MG-2.3-010, MG-3.1-016, MG-3.2-006, MG-3.2-013, MG-3.2-016, MG-4.1-005, MG-4.1-009, MG-4.1-010, MG-4.1-018, MP-3.4-007, MP-4.1-002, MP-4.1-004, MP-5.2-009, MS-1.1-029, MS-1.2-005, MS-2.2-007, MS-2.4-003, MS-2.4-004, MS-2.5-007, MS-2.5-008, MS-2.5-024, MS-2.6-003, MS-2.6-009, MS-2.6-016, MS-2.7-013, MS-2.7-014, MS-2.7-015, MS-2.10-007, MS-2.10-019, MS-2.10-020, MS-2.11-006, MS-2.11-030, MS-3.3-006, MS-4.2-009, MS-4.3-004) \\ 
		\hline
	\end{tabular}
\end{table}	
\pagebreak		
	
\begin{table}[H]
	\caption*{Table E: Selected generative AI risk controls (continued).}
	\label{tab:controls_cont2}
	\footnotesize
	\begin{tabular}{|m{0.25\linewidth} |m{0.70\linewidth} |}
		\hline
		\textbf{Name} & \textbf{Description} (Selected NIST AI RMF Action IDs)\\
		\hline		
		Narrow Scope & Systems are deployed for targeted business applications with documented and direct business value. (GV-1.2-002, MP-3.3-001, MP-5.1-011) \\ \hline
		Open Source & Open source code is used to promote explainability and transparency. (MG-4.2-007, MP-4.1-017) \\ 
		\hline
		Ownership & GAI systems and vendor relationships are owned by specific and documented internal personnel. (GV-6.1-009, GV-6.1-016, GV-6.2-008, MP-1.1-005, MP-1.1-008) \\ \hline
		Prohibited Use Policy & General abuse and misuse of GAI systems by internal parties is restricted by organizational policies. (GV-1.1-006, GV-1.2-003, GV-1.6-003, GV-3.2-003, GV-4.1-001, GV-6.1-017, GV-6.1-017) \\ 
		\hline
		RAG & Retreival augmented generation (RAG) is used to improve accuracy in generated content. (GV-1.2-002, MS-2.3-004, MS-2.5-005, MS-2.5-012, MS-2.9-003, MG-3.1-001, MG-3.1-006, MG-3.2-002, MG-3.2-003)  \\ \hline
		Rate-limiting  & GAI response times and query volumes are limited. (MS-2.6-007) \\ 
		\hline
		Redudancy & Rollover, fallback, and other redundancy mechanisms are available for GAI systems and address weights and other important system components. (GV-6.2-003, GV-6.2-007, GV-6.2-012, MG-2.4-012, MS-2.6-008)   \\ 
		\hline
		Refresh & Systems are retrained or re-tuned at a reasonable cadence. (MG-3.1-001, MG-3.2-011, MS-2.3-004, MS-2.12-003)  \\ 
		\hline
		Restrict Anonymous Use & Anonymous use of GAI systems is restricted. (GV-3.2-002)  \\ 
		\hline
		Restrict Anthropomorphization  & Human, animal, cyborg, emotional or other images or features that promote anthropomorphization of GAI systems are restricted. (GV-1.3-001, MS-2.5-009)   \\ 
		\hline
		Restrict Data Collection & All data collection is disclosed, collected data is protected and use in a transparent fashion. (GV-6.2-016,  MS-2.2-023, MS-2.10-013) \\ 
		\hline		
		Restrict Decision Making  & GAI systems are not employed for material decision-making tasks. (GV-1.3-001, GV-4.1-001, MP-1.1-018, MP-1.6-001, MP-3.4-017)  \\ \hline		
		Restrict Homogeneity & Feedback loops in which GAI systems are trained with GAI-generated data are restricted. (GV-1.3-004, MS-2.11-011)  \\ 
		\hline
		Restrict Internet Access & GAI systems are disconnected from the internet. (MP-2.2-007)  \\ 
		\hline		
		Restrict Location Tracking & Any location tracking is conducted with user consent, disclosed, aligned with relevant privacy policies and laws and potential threats to user safety are managed. (MS-2.10-002)  \\ 
		\hline		
		Restrict Minors & Use of organizational GAI systems by minors are restricted. \textcolor{red}{()}  \\ 
		\hline
		Restrict Regulated Dealings & GAI is not deployed in regulated dealings or for material decision making. (GV-1.1-004, GV-1.3-001, GV-4.1-001, GV-5.2-001, MP-2.3-013,  MS-2.11-018) \\ 
		\hline		
		Restrict Secondary Use & Any secondary use of GAI input data is conducted with user consent, disclosed, and aligned with relevant privacy policies and laws. (GV-6.1-016, GV-6.2-016)  \\ 
		\hline	
		RLHF & For third-party GAI systems, vendors engage in specific reinforcement with human feedback (RLHF) exercises to address identified risks; for internal systems, internal personnel engage in RLHF to address identified risks. (MG-2.1-002, MS-2.5-005, MS-2.9-003, MS-2.9-007)  \\ 
		\hline	
		Sensitive/Personal Data Removal & Personal, sensitive, biometric, or otherwise restricted data is minimized or eliminated from GAI training data. (GV-1.2-009, GV-1.6-003, MP-4.1-002, MP-4.1-016, MS-2.10-002, MS-2.10-003, MS-2.10-005, MS-2.10-014, MS-2.10-017, MS-2.10-018, MS-2.10-020)  \\ 
		\hline
		Session Limits & Time, query volume, and response rate are limited for GAI user sessions. (GV-4.1-001, MS-2.6-007, MS-2.6-010) \\ 
		\hline
		Supply Chain Audit & GAI system supply chains are audited and documented, with a focus on data poisoning, malware, and software and hardware vulnerabilities. (GV-4.1-004, GV-6.1-011, GV-6.1-022, GV-6.2-003, MG-2.3-001, MG-3.1-002, MP-5.1-003, MS-1.1-008, MS-2.6-001, MS-2.7-001)  \\
		\hline
		System Documentation & GAI systems are well-documented whether internal, open source, or vendor-provided. (GV-1.3-009, GV-1.4-002, GV-1.4-004, GV-1.4-005, GV-1.4-007, GV-1.6-007, GV-3.2-002, GV-3.2-009, GV-4.1-002, GV-4.2-011, GV-4.2-013, GV-4.3-002, GV-6.2-001, GV-6.2-014, MG-1.3-010, MG-2.2-016, MG-3.1-004, MG-3.1-009, MG-3.1-013, MG-3.1-015, MP-2.1-002, MP-2.3-027, MP-3.1-004, MP-3.4-015, MP-4.1-021, MP-4.2-003, MP-5.2-010, MS-1.3-002, MS-2.1-001, MS-2.2-014, MS-2.7-002, MS-2.7-012, MS-2.7-024, MS-2.8-007, MS-2.8-011) \\ 
		\hline
		System Prompt & System prompts are used to tune GAI systems to specific tasks and to mitigate risks. (GV-1.2-002, MS-2.3-004, MS-2.5-005, MS-2.5-012, MS-2.9-003, MG-3.1-001, MG-3.1-006, MG-3.2-002, MG-3.2-003) \\ 
		\hline
		Team Diversity & Teams that implement and manage GAI systems represent broad professional, educational, life-stage, and demographic diversity. (GV-2.1-004, GV-3.1-002, GV-3.1-004, GV-3.1-005, GV-3.2-008, MG-2.1-005, MP-1.2-003, MP-1.2-004, MP-1.2-007, MS-1.3-012, MS-1.3-017, MS-2.3-015, MS-3.3-012) \\ 
		\hline	
	\end{tabular}
\end{table}
\pagebreak

\begin{table}[H]
	\caption*{Table E: Selected generative AI risk controls (continued).}
	\label{tab:controls_cont3}
	\footnotesize
	\begin{tabular}{|m{0.25\linewidth} |m{0.70\linewidth} |}
		\hline
		\textbf{Name} & \textbf{Description} (Selected NIST AI RMF Action IDs)\\
		\hline	
		Temperature & Temperature settings are used to tune GAI systems to specific tasks and to mitigate risks.  (GV-1.2-002, MS-2.3-004, MS-2.5-005, MS-2.5-012, MS-2.9-003, MG-3.1-001, MG-3.1-006, MG-3.2-002, MG-3.2-003) \\ 
		\hline	
		Terms of Service & General abuse and misuse by external parties is prohibited by organizational policies. (GV-4.2-003, GV-4.2-005, GV-4.2-007, GV-6.1-016, GV-6.2-016, MP-4.1-021)  \\ 
		\hline
		Training  & Internal personnel recieve training on productivity and basic risk management for GAI systems. (GV-2.2-004, GV-3.2-002, GV-6.1-003, MS-1.1-014)  \\ 
		\hline
		User Feedback & GAI systems implement user feedback mechanisms. (GV-1.5-007, GV-1.5-009, GV-3.2-005, GV-5.1-001, GV-5.1-006, GV-5.1-007, GV-5.1-009, MG-1.3-005, MS-1.3-015, MS-1.3-016, MG-2.1-004, MG-2.2-012, MS-2.7-004, MS-4.2-012)  \\ 
		\hline
		User Recourse & Policies, processes, and technical mechanisms enable recourse for users who are harmed by GAI systems. (GV-1.5-010, GV-1.7-003, GV-5.1-001, GV-5.1-006, GV-5.1-009, MS-2.8-015, MS-2.8-019, MS-3.2-006, MS-4.2-012) \\ 
		\hline
		Validation & GAI systems are shown to reliably generate valid results for their targeted business application. (GV-1.2-009, GV-1.4-002, GV-1.4-004, GV-3.2-002, GV-5.1-005,  MG-2.2-016, MG-3.1-009, MG-3.1-014, MP-2.3-006, MP-2.3-013, MP-4.1-012, MS-2.3-005, MS-2.5-016, MS-2.9-002, MS-2.9-014) \\ 
		\hline
		XAI & Methods such as visualization, occlusion, model compression, pertubation studies, and similar are applied to increase explainability of GAI systems. (GV-1.4-002, GV-3.2-002, GV-5.1-005, MG-3.2-001, MP-2.2-006, MS-2.8-019, MS-2.9-001, MS-2.9-005, MS-2.9-006, MS-2.9-009, MS-2.9-011, MS-2.9-013, MS-2.9-015, MS-4.2-006) \\ 
		\hline
	\end{tabular}
\end{table}

\noindent\textbf{Usage Note}: Appendix E puts forward selected risk controls that organizations may apply for GAI risk management. Higher level controls are linked to specific GAI and AI RMF Playbook actions \cite{ai600-1}, \cite{playbook}. 

\begin{landscape} 
\thispagestyle{empty}	

% ---------- ----------
\section*{Appendix F: Example Low-risk Generative AI Measurement and Management Plan}\label{sec:appndxf}
% ---------- ----------

% ---------- ----------
\subsection*{F.1: Example Low-risk Generative AI Measurement and Management Plan by Trustworthy Characteristic}\label{appdxf1}
% ---------- ----------

\begin{table}[H]
	\caption*{Table F.1: Example risk measurement and management approaches suitable for low-risk GAI applications organized by trustworthy characteristic.}
	\footnotesize
	\begin{tabular}{|c|c|c|}
		\hline
		\multirow{2}{*}{\textbf{Function}} & \multicolumn{2}{|c|}{\textbf{Trustworthy Characteristic}}   \\
		\cline{2-3}
		& \textbf{Accountable and Transparent} & \textbf{Fair with Harmful Bias Managed}  \\
		\hline
		\textbf{Measure} & 
		\makecell[l]{
			\textbullet\hspace{3pt} An Evaluation on Large Language Model Outputs: \\\hspace{10pt}Discourse and Memorization (see Appendix B) \\
			\textbullet\hspace{3pt} Big-bench: Truthfulness \\
			\textbullet\hspace{3pt} DecodingTrust: Machine Ethics \\
			\textbullet\hspace{3pt} Evaluation Harness: ETHICS \\
			\textbullet\hspace{3pt} HELM: Copyright \\
			\textbullet\hspace{3pt} Mark My Words}
		& 
		\makecell[l]{
			\textbullet\hspace{3pt} BELEBELE \\
			\textbullet\hspace{3pt} Big-bench: Low-resource language, Non-English, Translation  \\
			\textbullet\hspace{3pt} Big-bench: Social bias, Racial bias, Gender bias, Religious bias \\
			\textbullet\hspace{3pt} Big-bench: Toxicity \\
			\textbullet\hspace{3pt} DecodingTrust: Fairness \\
			\textbullet\hspace{3pt} DecodingTrust: Stereotype Bias \\
			\textbullet\hspace{3pt} DecodingTrust: Toxicity \\
			\textbullet\hspace{3pt} C-Eval (Chinese evaluation suite) \\
			\textbullet\hspace{3pt} Evaluation Harness: CrowS-Pairs  \\
			\textbullet\hspace{3pt} Evaluation Harness: ToxiGen \\
			\textbullet\hspace{3pt} Finding New Biases in Language Models with a Holistic Descriptor Dataset \\
			\textbullet\hspace{3pt} From Pretraining Data to Language Models to Downstream Tasks:\\\hspace{10pt}Tracking the Trails of Political Biases Leading to Unfair NLP Models \\
			\textbullet\hspace{3pt} HELM: Bias \\
			\textbullet\hspace{3pt} HELM: Toxicity \\
			\textbullet\hspace{3pt} MT-bench \\
			\textbullet\hspace{3pt} The Self-Perception and Political Biases of ChatGPT \\
			\textbullet\hspace{3pt} Towards Measuring the Representation of\\\hspace{10pt} Subjective Global Opinions in Language Models
		}
		\\
		\hline		
		\textbf{Manage} &
		\makecell[l]{	
			\textbullet\hspace{3pt} Contract Review\\ 	
			\textbullet\hspace{3pt} Disclosure of AI Interaction\\ 		
			\textbullet\hspace{3pt} Instructions\\ 	
			\textbullet\hspace{3pt} Inventory\\ 	
			\textbullet\hspace{3pt} Ownership\\ 	
			\textbullet\hspace{3pt} Prohibited Use Policy\\ 
			\textbullet\hspace{3pt} Restrict Decision Making \\					
			\textbullet\hspace{3pt} System Documentation\\ 	
			\textbullet\hspace{3pt} Terms of Service\\ 	 	
		}
		& 
		\makecell[l]{	
			\textbullet\hspace{3pt} Content Moderation\\
			\textbullet\hspace{3pt} Failure Avoidance\\ 	
			\textbullet\hspace{3pt} Instructions\\ 	
			\textbullet\hspace{3pt} Inventory\\ 	
			\textbullet\hspace{3pt} Ownership\\ 
			\textbullet\hspace{3pt} Prohibited Use Policy\\ 	
			\textbullet\hspace{3pt} System Prompt\\ 
			\textbullet\hspace{3pt} Restrict Anonymous Use\\ 			
			\textbullet\hspace{3pt} Restrict Decision Making\\  				
			\textbullet\hspace{3pt} Temperature\\ 	
			\textbullet\hspace{3pt} Terms of Service\\ 	
		} 
		\\
		\hline
	\end{tabular}
	\label{table:low_risk_plan_by_tc1}
\end{table}

\vfill
\raisebox{-10pt}{\makebox[\linewidth]{\thepage}}

\pagebreak
\thispagestyle{empty}

\begin{table}[H]
	\caption*{Table F.1: Example risk measurement and management approaches suitable for low-risk GAI applications organized by trustworthy characteristic (continued).}
	\footnotesize
	\begin{tabular}{|c|c|c|c|c|}
		\hline
		\multirow{2}{*}{\textbf{Function}} & \multicolumn{4}{|c|}{\textbf{Trustworthy Characteristic}}   \\
		\cline{2-5}
		& \textbf{Interpretable and Explainable} & \textbf{Privacy-enhanced} & \textbf{Safe} & \textbf{Secure and Resilient} \\
		\hline
		\textbf{Measure} & 
		& 
		\makecell[l]{
			\textbullet\hspace{3pt} HELM: Copyright \\
			\textbullet\hspace{3pt} llmprivacy \\
			\textbullet\hspace{3pt} mimir			
		}
		&
		\makecell[l]{ 
			\textbullet\hspace{3pt} Big-bench: Convince Me \\ 
			\textbullet\hspace{3pt} Big-bench: Truthfulness \\ 
			\textbullet\hspace{3pt} HELM: Reiteration, Wedging \\ 
			\textbullet\hspace{3pt} Mark My Words \\ 
			\textbullet\hspace{3pt} MLCommons \\ 
			\textbullet\hspace{3pt} The WMDP Benchmark \\ 
		}	
		&
		\makecell[l]{
			\textbullet\hspace{3pt} Catastrophic Jailbreak of Open-source LLMs \\\hspace{10pt}via Exploiting Generation \\ 
			\textbullet\hspace{3pt} DecodingTrust: Adversarial Robustness, \\\hspace{10pt}Robustness Against Adversarial Demonstrations \\ 
			\textbullet\hspace{3pt} detect-pretrain-code \\ 
			\textbullet\hspace{3pt} In-The-Wild Jailbreak Prompts on LLMs \\ 
			\textbullet\hspace{3pt} JailbreakingLLMs \\ 
			\textbullet\hspace{3pt} llmprivacy \\ 
			\textbullet\hspace{3pt} mimir \\ 
			\textbullet\hspace{3pt} TAP: A Query-Efficient Method for Jailbreaking \\\hspace{10pt}Black-Box LLMs \\ 
		}			
		\\
		\hline		
		\textbf{Manage} &
		\makecell[l]{
			\textbullet\hspace{3pt} Instructions\\ 	
			\textbullet\hspace{3pt} Inventory\\ 
			\textbullet\hspace{3pt} System Documentation\\ 	
		}
		& 
		\makecell[l]{
			\textbullet\hspace{3pt} Content Moderation\\ 	
			\textbullet\hspace{3pt} Contract Review\\ 	
			\textbullet\hspace{3pt} Failure Avoidance\\ 	
			\textbullet\hspace{3pt} Inventory\\ 	
			\textbullet\hspace{3pt} Ownership\\ 	
			\textbullet\hspace{3pt} Prohibited Use Policy\\ 
			\textbullet\hspace{3pt} Restrict Anonymous Use\\ 					
			\textbullet\hspace{3pt} System Documentation\\ 	
			\textbullet\hspace{3pt} Terms of Service\\ 	
		}
		&
		\makecell[l]{ 	
			\textbullet\hspace{3pt} Content Moderation\\ 	
			\textbullet\hspace{3pt} Disclosure of AI Interaction\\ 	
			\textbullet\hspace{3pt} Failure Avoidance\\ 	
			\textbullet\hspace{3pt} Instructions\\ 	
			\textbullet\hspace{3pt} Inventory\\ 	
			\textbullet\hspace{3pt} Ownership\\ 	
			\textbullet\hspace{3pt} Prohibited Use Policy\\ 	
			\textbullet\hspace{3pt} Restrict Anonymous Use\\ 	
			\textbullet\hspace{3pt} Restrict Anthropomorphization \\ 				
			\textbullet\hspace{3pt} Restrict Decision Making\\ 				
			\textbullet\hspace{3pt} System Documentation\\ 	
			\textbullet\hspace{3pt} System Prompt\\ 	
			\textbullet\hspace{3pt} Temperature\\ 	
			\textbullet\hspace{3pt} Terms of Service \\
		}
		&
		\makecell[l]{
			\textbullet\hspace{3pt} Access Control\\ 	
			\textbullet\hspace{3pt} Approved List\\ 	
			\textbullet\hspace{3pt} Authentication\\ 	
			\textbullet\hspace{3pt} Change Management\\ 	
			\textbullet\hspace{3pt} Dependency Screening\\ 	
			\textbullet\hspace{3pt} Failure Avoidance\\ 	
			\textbullet\hspace{3pt} Inventory\\ 	
			\textbullet\hspace{3pt} Ownership\\  	
			\textbullet\hspace{3pt} Malware Screening\\ 
			\textbullet\hspace{3pt} Restrict Anonymous Use\\ 	
		}
		\\
		\hline
	\end{tabular}
	\label{table:low_risk_plan_by_tc_cont}
\end{table}

\end{landscape}
\pagebreak 

\begin{table}[H]
	\caption*{Table F.1: Example risk measurement and management approaches suitable for low-risk GAI applications organized by trustworthy characteristic (continued).}
	\footnotesize
	\begin{tabular}{|c|c|}
		\hline
		\multirow{2}{*}{\textbf{Function}} & \multicolumn{1}{|c|}{\textbf{Trustworthy Characteristic}}   \\
		\cline{2-2}
		& \textbf{Valid and Reliable} \\
		\hline
		\textbf{Measure} 
		& 
		\makecell[l]{ 
			\textbullet\hspace{3pt} Big-bench: Algorithms, Logical reasoning, Implicit reasoning, Mathematics, Arithmetic, \\\hspace{10pt}Algebra, Mathematical proof, Black-Box Fallacy, Negation, Computer code, Probabilistic reasoning, \\\hspace{10pt}Social reasoning, Analogical reasoning, Multi-step, Understanding the World \\ 
			\textbullet\hspace{3pt} Big-bench: Analytic entailment, Formal fallacies and syllogisms with negation, Entailed polarity \\ 
			\textbullet\hspace{3pt} Big-bench: Context Free Question Answering \\ 
			\textbullet\hspace{3pt} Big-bench: Contextual question answering, Reading comprehension, Question generation \\ 
			\textbullet\hspace{3pt} Big-bench: Morphology, Grammar, Syntax \\ 
			\textbullet\hspace{3pt} Big-bench: Out-of-Distribution \\ 
			\textbullet\hspace{3pt} Big-bench: Paraphrase \\ 
			\textbullet\hspace{3pt} Big-bench: Sufficient information \\ 
			\textbullet\hspace{3pt} Big-bench: Summarization \\ 
			\textbullet\hspace{3pt} DecodingTrust: Out-of-Distribution Robustness, Adversarial Robustness,\\\hspace{10pt}Robustness Against Adversarial Demonstrations \\ 
			\textbullet\hspace{3pt} Eval Gauntlet: Reading comprehension \\ 
			\textbullet\hspace{3pt} Eval Gauntlet: Commonsense reasoning, Symbolic problem solving, Programming \\ 
			\textbullet\hspace{3pt} Eval Gauntlet: Language Understanding \\ 
			\textbullet\hspace{3pt} Eval Gauntlet: World Knowledge \\ 
			\textbullet\hspace{3pt} Evaluation Harness: BLiMP \\ 
			\textbullet\hspace{3pt} Evaluation Harness: CoQA, ARC \\ 
			\textbullet\hspace{3pt} Evaluation Harness: GLUE \\ 
			\textbullet\hspace{3pt} Evaluation Harness: HellaSwag, OpenBookQA, TruthfulQA \\ 
			\textbullet\hspace{3pt} Evaluation Harness: MuTual \\ 
			\textbullet\hspace{3pt} Evaluation Harness: PIQA, PROST, MC-TACO, MathQA, LogiQA, DROP \\ 
			\textbullet\hspace{3pt} FLASK: Logical correctness, Logical robustness, Logical efficiency, Comprehension, Completeness  \\ 
			\textbullet\hspace{3pt} FLASK: Readability, Conciseness, Insightfulness \\ 
			\textbullet\hspace{3pt} HELM: Knowledge \\ 
			\textbullet\hspace{3pt} HELM: Language \\ 
			\textbullet\hspace{3pt} HELM: Text classification \\ 
			\textbullet\hspace{3pt} HELM: Question answering \\ 
			\textbullet\hspace{3pt} HELM: Reasoning \\ 
			\textbullet\hspace{3pt} HELM: Robustness to contrast sets \\ 
			\textbullet\hspace{3pt} HELM: Summarization \\ 
			\textbullet\hspace{3pt} Hugging Face: Fill-mask, Text generation  \\ 
			\textbullet\hspace{3pt} Hugging Face: Question answering \\ 
			\textbullet\hspace{3pt} Hugging Face: Summarization \\ 
			\textbullet\hspace{3pt} Hugging Face: Text classification, Token classification, Zero-shot classification \\ 
			\textbullet\hspace{3pt} MASSIVE  \\ 
			\textbullet\hspace{3pt} MT-bench \\ 
		}		
		\\
		\hline		
		\textbf{Manage} &
		\makecell[l]{ 	
			\textbullet\hspace{3pt} Content Moderation \\ 
			\textbullet\hspace{3pt} Disclosure of AI Interaction \\ 
			\textbullet\hspace{3pt} Failure Avoidance \\
			\textbullet\hspace{3pt} Instructions\\ 	
			\textbullet\hspace{3pt} Restrict Anthropomorphization \\		
			\textbullet\hspace{3pt} Restrict Decision Making \\ 					 
			\textbullet\hspace{3pt} System Documentation\\ 			
			\textbullet\hspace{3pt} System Prompt \\ 
			\textbullet\hspace{3pt} Temperature \\ 	
		}
		\\
		\hline
	\end{tabular}
	\label{table:low_risk_plan_by_tc_cont2}
\end{table}

% ---------- ----------
\subsection*{F.2: Example Low-risk Generative AI Measurement and Management Plan by Generative AI Risk}\label{appdxf2}
% ---------- ----------

\begin{table}[H]
	\caption*{Table F.2: Example risk measurement and management approaches suitable for low-risk GAI applications organized by GAI risk.}
	\scriptsize
	\begin{tabular}{|c|c|c|}
		\hline
		\multirow{2}{*}{\textbf{GAI Risk}} & \multicolumn{2}{|c|}{\textbf{Function}}   \\
		\cline{2-3}
		& \textbf{CBRN Information} &\textbf{Confabulation} \\
		\hline
		\textbf{Measure} & 
		\makecell[l]{ 
			\textbullet\hspace{3pt} Big-bench: Convince Me \\ 
			\textbullet\hspace{3pt} Big-bench: Truthfulness \\ 
			\textbullet\hspace{3pt} HELM: Reiteration, Wedging \\ 
			\textbullet\hspace{3pt} MLCommons \\ 
			\textbullet\hspace{3pt} The WMDP Benchmark \\ 
		}
		& 
		\makecell[l]{ 
			\textbullet\hspace{3pt} Big-bench: Algorithms, Logical reasoning, Implicit reasoning, Mathematics,\\\hspace{10pt}Arithmetic, Algebra, Mathematical proof, Black-Box Fallacy, Negation,\\\hspace{10pt}Computer code, Probabilistic reasoning, Social reasoning, Analogical reasoning, \\\hspace{10pt}Multi-step, Understanding the World \\ 
			\textbullet\hspace{3pt} Big-bench: Analytic entailment, Formal fallacies and syllogisms with negation,\\\hspace{10pt}Entailed polarity \\ 
			\textbullet\hspace{3pt} Big-bench: Context Free Question Answering \\ 
			\textbullet\hspace{3pt} Big-bench: Contextual question answering, Reading comprehension, Question generation \\ 
			\textbullet\hspace{3pt} Big-bench: Convince Me \\ 
			\textbullet\hspace{3pt} Big-bench: Low-resource language, Non-English, Translation \\ 
			\textbullet\hspace{3pt} Big-bench: Morphology, Grammar, Syntax \\ 
			\textbullet\hspace{3pt} Big-bench: Out-of-Distribution \\ 
			\textbullet\hspace{3pt} Big-bench: Paraphrase \\ 
			\textbullet\hspace{3pt} Big-bench: Sufficient information \\ 
			\textbullet\hspace{3pt} Big-bench: Summarization \\ 
			\textbullet\hspace{3pt} Big-bench: Truthfulness \\ 
			\textbullet\hspace{3pt} C-Eval (Chinese evaluation suite) \\ 
			\textbullet\hspace{3pt} DecodingTrust: Out-of-Distribution Robustness,\\\hspace{10pt}Robustness Against Adversarial Demonstrations \\ 
			\textbullet\hspace{3pt} Eval Gauntlet Reading comprehension \\ 
			\textbullet\hspace{3pt} Eval Gauntlet: Commonsense reasoning, Symbolic problem solving, Programming \\ 
			\textbullet\hspace{3pt} Eval Gauntlet: Language Understanding \\ 
			\textbullet\hspace{3pt} Eval Gauntlet: World Knowledge \\ 
			\textbullet\hspace{3pt} Evaluation Harness: BLiMP \\ 
			\textbullet\hspace{3pt} Evaluation Harness: CoQA, ARC \\ 
			\textbullet\hspace{3pt} Evaluation Harness: GLUE \\ 
			\textbullet\hspace{3pt} Evaluation Harness: HellaSwag, OpenBookQA, TruthfulQA \\ 
			\textbullet\hspace{3pt} Evaluation Harness: MuTual \\ 
			\textbullet\hspace{3pt} Evaluation Harness: PIQA, PROST, MC-TACO, MathQA, LogiQA, DROP \\ 
			\textbullet\hspace{3pt} FLASK: Logical correctness, Logical robustness, Logical efficiency, Comprehension, \\\hspace{10pt}Completeness \\ 
			\textbullet\hspace{3pt} FLASK: Readability, Conciseness, Insightfulness \\ 
			\textbullet\hspace{3pt} Finding New Biases in Language Models with a Holistic Descriptor Dataset \\ 
			\textbullet\hspace{3pt} HELM: Knowledge \\ 
			\textbullet\hspace{3pt} HELM: Language \\ 
			\textbullet\hspace{3pt} HELM: Language (Twitter AAE) \\ 
			\textbullet\hspace{3pt} HELM: Question answering \\ 
			\textbullet\hspace{3pt} HELM: Reasoning \\ 
			\textbullet\hspace{3pt} HELM: Reiteration, Wedging \\ 
			\textbullet\hspace{3pt} HELM: Robustness to contrast sets \\ 
			\textbullet\hspace{3pt} HELM: Summarization \\ 
			\textbullet\hspace{3pt} HELM: Text classification \\ 
			\textbullet\hspace{3pt} Hugging Face: Fill-mask, Text generation \\ 
			\textbullet\hspace{3pt} Hugging Face: Question answering \\ 
			\textbullet\hspace{3pt} Hugging Face: Summarization \\ 
			\textbullet\hspace{3pt} Hugging Face: Text classification, Token classification, Zero-shot classification \\ 
			\textbullet\hspace{3pt} MASSIVE \\ 
			\textbullet\hspace{3pt} MLCommons \\ 
			\textbullet\hspace{3pt} MT-bench  \\ 
		}
		\\ 
		\hline		
		\textbf{Manage} &
		\makecell[l]{ 
			\textbullet\hspace{3pt} Access Control \\ 
			\textbullet\hspace{3pt} Failure Avoidance \\ 
			\textbullet\hspace{3pt} Inventory\\ 	
			\textbullet\hspace{3pt} Ownership\\ 	
			\textbullet\hspace{3pt} Prohibited Use Policy \\ 
			\textbullet\hspace{3pt} Terms of Service \\ 
		}		
		&
		\makecell[l]{ 
			\textbullet\hspace{3pt} Content Moderation \\ 
			\textbullet\hspace{3pt} Disclosure of AI Interaction \\ 
			\textbullet\hspace{3pt} Failure Avoidance \\
			\textbullet\hspace{3pt} Instructions\\ 	
			\textbullet\hspace{3pt} Restrict Anthropomorphization \\			
			\textbullet\hspace{3pt} Restrict Decision Making \\ 					 
			\textbullet\hspace{3pt} System Documentation\\ 			
			\textbullet\hspace{3pt} System Prompt \\ 
			\textbullet\hspace{3pt} Temperature \\ 
		}	
		\\
		\hline
	\end{tabular}
	\label{table:low_risk_plan_by_gai_risk}
\end{table}

\pagebreak

\begin{landscape}
\thispagestyle{empty}

\begin{table}[H]
	\caption*{Table F.2: Example risk measurement and management approaches suitable for low-risk GAI applications organized by GAI risk (continued).}
	\scriptsize
	\begin{tabular}{|c|c|c|c|c|}
		\hline
		\multirow{2}{*}{\textbf{Function}} & \multicolumn{4}{|c|}{\textbf{GAI Risk}} \\
		\cline{2-5}
		& \textbf{Dangerous or Violent Recommendations} & \textbf{Data Privacy} & \textbf{Environmental} &
		\textbf{Human-AI Configuration} \\
		\hline
		\textbf{Measure} &
		\makecell[l]{ 	
			\textbullet\hspace{3pt} Big-bench: Convince Me\\ 	
			\textbullet\hspace{3pt} Big-bench: Toxicity\\ 	
			\textbullet\hspace{3pt} DecodingTrust: Adversarial Robustness,\\\hspace{10pt}Robustness Against Adversarial Demonstrations\\ 	
			\textbullet\hspace{3pt} DecodingTrust: Machine Ethics\\ 	
			\textbullet\hspace{3pt} DecodingTrust: Toxicity\\ 	
			\textbullet\hspace{3pt} Evaluation Harness: ToxiGen\\ 	
			\textbullet\hspace{3pt} HELM: Reiteration, Wedging\\ 	
			\textbullet\hspace{3pt} HELM: Toxicity\\ 	
			\textbullet\hspace{3pt} MLCommons	\\ 	
		} & 
		\makecell[l]{ 	
			\textbullet\hspace{3pt} An Evaluation on Large Language Model Outputs:\\\hspace{10pt}Discourse and Memorization (with human scoring,\\\hspace{10pt}see Appendix B)\\ 	
			\makecell[l]{\textbullet\hspace{3pt} Catastrophic Jailbreak of Open-source LLMs via\\\hspace{10pt}Exploiting Generation}\\ 	
			\textbullet\hspace{3pt} DecodingTrust: Machine Ethics\\ 	
			\textbullet\hspace{3pt} Evaluation Harness: ETHICS\\ 	
			\textbullet\hspace{3pt} HELM: Copyright\\ 	
			\textbullet\hspace{3pt} In-The-Wild Jailbreak Prompts on LLMs\\ 	
			\textbullet\hspace{3pt} JailbreakingLLMs\\ 	
			\textbullet\hspace{3pt} MLCommons\\ 	
			\textbullet\hspace{3pt} Mark My Words\\ 	
			\makecell[l]{\textbullet\hspace{3pt} TAP: A Query-Efficient Method for Jailbreaking\\\hspace{10pt}Black-Box LLMs}\\ 	
			\textbullet\hspace{3pt} detect-pretrain-code\\ 	
			\textbullet\hspace{3pt} llmprivacy\\ 	
			\textbullet\hspace{3pt} mimir 	\\ 	
		} &
		\makecell[l]{ 	
			\textbullet\hspace{3pt} HELM: Efficiency	\\ 	
		}
		&
		\\ 
		\hline		
		\textbf{Manage} &
		\makecell[l]{ 	
			\textbullet\hspace{3pt} Content Moderation\\ 
			\textbullet\hspace{3pt} Disclosure of AI Interaction \\				
			\textbullet\hspace{3pt} Failure Avoidance\\ 
			\textbullet\hspace{3pt} Instructions \\
			\textbullet\hspace{3pt} Inventory \\
			\textbullet\hspace{3pt} Ownership \\		 		
			\textbullet\hspace{3pt} Prohibited Use Policy\\
			\textbullet\hspace{3pt} Restrict Anonymous Use\\ 
			\textbullet\hspace{3pt} Restrict Anthropomorphization \\
			\textbullet\hspace{3pt} Restrict Decision making \\
			\textbullet\hspace{3pt} System Documentation \\
			\textbullet\hspace{3pt} System Prompt \\
			\textbullet\hspace{3pt} Temperature	\\
			\textbullet\hspace{3pt} Terms of Service\\ 		
		} & 
		\makecell[l]{ 	
			\textbullet\hspace{3pt} Content Moderation\\ 	
			\textbullet\hspace{3pt} Contract Review\\ 	
			\textbullet\hspace{3pt} Failure Avoidance\\ 	
			\textbullet\hspace{3pt} Inventory\\ 	
			\textbullet\hspace{3pt} Ownership\\ 	
			\textbullet\hspace{3pt} Prohibited Use Policy\\ 	
			\textbullet\hspace{3pt} Restrict Anonymous Use\\ 				
			\textbullet\hspace{3pt} System Documentation\\ 	
			\textbullet\hspace{3pt} Terms of Service\\ 	
		} & 
		\makecell[l]{ 	
			\textbullet\hspace{3pt} Access Control\\ 	
			\textbullet\hspace{3pt} Failure Avoidance\\ 	
			\textbullet\hspace{3pt} Inventory\\ 	
			\textbullet\hspace{3pt} Ownership\\ 
			\textbullet\hspace{3pt} Restrict Anonymous Use\\ 				
		} &
		\makecell[l]{ 	
			\textbullet\hspace{3pt} Content Moderation\\ 	
			\textbullet\hspace{3pt} Disclosure of AI Interaction\\ 	
			\textbullet\hspace{3pt} Failure Avoidance\\ 	
			\textbullet\hspace{3pt} Instructions\\ 	
			\textbullet\hspace{3pt} Inventory\\ 	
			\textbullet\hspace{3pt} Ownership\\ 	
			\textbullet\hspace{3pt} Prohibited Use Policy\\ 	
			\textbullet\hspace{3pt} Restrict Anonymous Use\\ 	
			\textbullet\hspace{3pt} Restrict Anthropomorphization\\ 	
			\textbullet\hspace{3pt} Restrict Decision Making\\ 			
			\textbullet\hspace{3pt} Terms of Service\\ 	
			\textbullet\hspace{3pt} Training\\
		}
		\\
		\hline
	\end{tabular}
	\label{table:low_risk_plan_by_gai_risk_cont}
\end{table}

\vfill
\raisebox{-10pt}{\makebox[\linewidth]{\thepage}}

\end{landscape}

\pagebreak

\begin{table}[H]
	\caption*{Table F.2: Example risk measurement and management approaches suitable for low-risk GAI applications organized by GAI risk (continued).}
	\scriptsize
	\begin{tabular}{|c|c|c|c|}
		\hline
		\multirow{2}{*}{\textbf{Function}} & \multicolumn{3}{|c|}{\textbf{GAI Risk}} \\
		\cline{2-4}
		& \textbf{Information Integrity} & \textbf{Information Security} & \textbf{Intellectual Property} \\
		\hline
		\textbf{Measure} &
		\makecell[l]{ 	
			\textbullet\hspace{3pt} Big-bench: Analytic entailment, Formal fallacies\\\hspace{10pt}and syllogisms with negation, Entailed polarity\\ 	
			\textbullet\hspace{3pt} Big-bench: Convince Me\\ 	
			\textbullet\hspace{3pt} Big-bench: Paraphrase\\ 	
			\textbullet\hspace{3pt} Big-bench: Sufficient information\\ 	
			\textbullet\hspace{3pt} Big-bench: Summarization\\ 	
			\textbullet\hspace{3pt} Big-bench: Truthfulness\\ 	
			\textbullet\hspace{3pt} DecodingTrust: Machine Ethics\\ 	
			\textbullet\hspace{3pt} DecodingTrust: Out-of-Distribution Robustness,\\\hspace{10pt}Robustness Against Adversarial Demonstrations,\\\hspace{10pt}Adversarial Robustness \\ 	
			\textbullet\hspace{3pt} Eval Gauntlet: Language Understanding\\ 	
			\textbullet\hspace{3pt} Eval Gauntlet: World Knowledge\\ 	
			\textbullet\hspace{3pt} Evaluation Harness: CoQA, ARC\\ 	
			\textbullet\hspace{3pt} Evaluation Harness: ETHICS\\ 	
			\textbullet\hspace{3pt} Evaluation Harness: GLUE\\ 	
			\textbullet\hspace{3pt} Evaluation Harness: HellaSwag, OpenBookQA,\\\hspace{10pt}TruthfulQA\\ 	
			\textbullet\hspace{3pt} Evaluation Harness: MuTual\\ 	
			\makecell[l]{\textbullet\hspace{3pt} Evaluation Harness: PIQA, PROST, MC-TACO,\\\hspace{10pt}MathQA, LogiQA, DROP}\\ 	
			\textbullet\hspace{3pt} FLASK: Logical correctness, Logical robustness,\\\hspace{10pt}Logical efficiency, Comprehension, Completeness\\ 	
			\textbullet\hspace{3pt} FLASK: Readability, Conciseness, Insightfulness\\ 	
			\textbullet\hspace{3pt} HELM: Knowledge\\ 	
			\textbullet\hspace{3pt} HELM: Language\\ 	
			\textbullet\hspace{3pt} HELM: Question answering\\ 	
			\textbullet\hspace{3pt} HELM: Reasoning\\ 	
			\textbullet\hspace{3pt} HELM: Reiteration, Wedging\\ 	
			\textbullet\hspace{3pt} HELM: Robustness to contrast sets\\ 	
			\textbullet\hspace{3pt} HELM: Summarization\\ 	
			\textbullet\hspace{3pt} HELM: Text classification\\ 	
			\textbullet\hspace{3pt} Hugging Face: Fill-mask, Text generation\\ 	
			\textbullet\hspace{3pt} Hugging Face: Question answering\\ 	
			\textbullet\hspace{3pt} Hugging Face: Summarization\\ 	
			\textbullet\hspace{3pt} MLCommons\\ 	
			\textbullet\hspace{3pt} MT-bench\\ 	
			\textbullet\hspace{3pt} Mark My Words	\\ 	
		} & \makecell[l]{ 	
			\textbullet\hspace{3pt} Big-bench: Convince Me\\ 	
			\textbullet\hspace{3pt} Big-bench: Out-of-Distribution\\ 	
			\textbullet\hspace{3pt} Catastrophic Jailbreak of Open-source\\\hspace{10pt}LLMs via Exploiting Generation\\ 	
			\textbullet\hspace{3pt} DecodingTrust: Out-of-Distribution \\\hspace{10pt}Robustness, Robustness Against\\\hspace{10pt}Adversarial Demonstrations, \\\hspace{10pt}Adversarial Robustness, \\ 	
			\textbullet\hspace{3pt} Eval Gauntlet: Commonsense \\\hspace{10pt}reasoning, Symbolic problem\\\hspace{10pt}solving, Programming\\ 	
			\textbullet\hspace{3pt} HELM: Copyright\\ 	
			\textbullet\hspace{3pt} In-The-Wild Jailbreak Prompts\\\hspace{10pt}on LLMs\\ 	
			\textbullet\hspace{3pt} JailbreakingLLMs\\ 	
			\textbullet\hspace{3pt} Mark My Words\\ 	
			\textbullet\hspace{3pt} TAP: A Query-Efficient Method for \\\hspace{10pt}Jailbreaking Black-Box LLMs\\ 	
			\textbullet\hspace{3pt} detect-pretrain-code\\ 	
			\textbullet\hspace{3pt} llmprivacy\\ 	
			\textbullet\hspace{3pt} mimir\\ 	
		} 
		& \makecell[l]{ 	
			\textbullet\hspace{3pt} An Evaluation on\\\hspace{10pt}Large Language Model Outputs:\\\hspace{10pt}Discourse and Memorization\\\hspace{10pt}(with human scoring,\\\hspace{10pt}see Appendix B)\\ 	
			\textbullet\hspace{3pt} HELM: Copyright\\ 	
			\textbullet\hspace{3pt} Mark My Words\\ 	
			\textbullet\hspace{3pt} llmprivacy\\ 	
			\textbullet\hspace{3pt} mimir	\\ 	
		} \\ 
		\hline		
		\textbf{Manage} &
		\makecell[l]{ 	
			\textbullet\hspace{3pt} Content Moderation\\ 	
			\textbullet\hspace{3pt} Disclosure of AI Interaction\\ 	
			\textbullet\hspace{3pt} Failure Avoidance\\ 	
			\textbullet\hspace{3pt} Inventory\\ 	
			\textbullet\hspace{3pt} Ownership\\ 	
			\textbullet\hspace{3pt} Prohibited Use Policy\\ 	
			\textbullet\hspace{3pt} Restrict Anonymous Use\\ 	
			\textbullet\hspace{3pt} Restrict Anthropomorphization\\				
			\textbullet\hspace{3pt} System Prompt\\ 	
			\textbullet\hspace{3pt} Temperature\\ 	
			\textbullet\hspace{3pt} Terms of Service\\ 	
		}
		& \makecell[l]{ 	
			\textbullet\hspace{3pt} Access Control\\ 	
			\textbullet\hspace{3pt} Approved List\\ 	
			\textbullet\hspace{3pt} Authentication\\ 	
			\textbullet\hspace{3pt} Change Management\\ 	
			\textbullet\hspace{3pt} Dependency Screening\\ 	
			\textbullet\hspace{3pt} Failure Avoidance\\ 	
			\textbullet\hspace{3pt} Inventory\\ 	
			\textbullet\hspace{3pt} Ownership\\ 	
			\textbullet\hspace{3pt} Malware Screening\\ 
			\textbullet\hspace{3pt} Restrict Anonymous Use\\ 	
		} 
		& \makecell[l]{ 	
			\textbullet\hspace{3pt} Contract Review\\ 	
			\textbullet\hspace{3pt} Disclosure of AI Interaction\\ 	
			\textbullet\hspace{3pt} Instructions\\ 	
			\textbullet\hspace{3pt} Inventory\\ 	
			\textbullet\hspace{3pt} Ownership\\ 	
			\textbullet\hspace{3pt} Prohibited Use Policy\\ 	
			\textbullet\hspace{3pt} Terms of Service\\ 	
		} \\
		\hline
	\end{tabular}
	\label{table:low_risk_plan_by_gai_risk_cont2}
\end{table}

\pagebreak

\begin{landscape}
\thispagestyle{empty}

\begin{table}[H]
	\caption*{Table F.2: Example risk measurement and management approaches suitable for low-risk GAI applications organized by GAI risk (continued).}
	\scriptsize
	\begin{tabular}{|c|c|c|c|c|}
		\hline
		\multirow{2}{*}{\textbf{Function}} & \multicolumn{3}{|c|}{\textbf{GAI Risk}} \\
		\cline{2-4}
		& \textbf{Obscene, Degrading, and/or Abusive Content} & \textbf{Toxicity, Bias, and Homogenization} & \textbf{Value Chain and Component Integration} \\
		\hline
		\textbf{Measure} & \makecell[l]{ 	
			\textbullet\hspace{3pt} Big-bench: Social bias, Racial bias,\\\hspace{10pt}Gender bias, Religious bias\\ 	
			\textbullet\hspace{3pt} Big-bench: Toxicity\\ 	
			\textbullet\hspace{3pt} DecodingTrust: Fairness\\ 	
			\textbullet\hspace{3pt} DecodingTrust: Stereotype Bias\\ 	
			\textbullet\hspace{3pt} DecodingTrust: Toxicity\\ 	
			\textbullet\hspace{3pt} Evaluation Harness: CrowS-Pairs\\ 	
			\textbullet\hspace{3pt} Evaluation Harness: ToxiGen\\ 	
			\textbullet\hspace{3pt} HELM: Bias\\ 	
			\textbullet\hspace{3pt} HELM: Toxicity\\ 	
		} 
		& \makecell[l]{ 	
			\textbullet\hspace{3pt} BELEBELE\\ 	
			\textbullet\hspace{3pt} Big-bench: Low-resource language, \\\hspace{10pt}Non-English, Translation\\ 	
			\textbullet\hspace{3pt} Big-bench: Out-of-Distribution\\ 	
			\textbullet\hspace{3pt} Big-bench: Social bias, Racial bias,\\\hspace{10pt}Gender bias, Religious bias\\ 	
			\textbullet\hspace{3pt} Big-bench: Toxicity\\ 	
			\textbullet\hspace{3pt} C-Eval (Chinese evaluation suite)\\ 	
			\textbullet\hspace{3pt} DecodingTrust: Fairness\\ 	
			\textbullet\hspace{3pt} DecodingTrust: Stereotype Bias\\ 	
			\textbullet\hspace{3pt} DecodingTrust: Toxicity\\ 	
			\textbullet\hspace{3pt} Eval Gauntlet: World Knowledge\\ 	
			\textbullet\hspace{3pt} Evaluation Harness: CrowS-Pairs\\ 	
			\textbullet\hspace{3pt} Evaluation Harness: ToxiGen\\ 	
			\textbullet\hspace{3pt} Finding New Biases in Language Models with\\\hspace{10pt}a Holistic Descriptor Dataset\\ 	
			\textbullet\hspace{3pt} From Pretraining Data to Language Models\\\hspace{10pt}to Downstream Tasks: 
			Tracking the Trails\\\hspace{10pt}of Political Biases Leading to Unfair NLP Models\\ 	
			\textbullet\hspace{3pt} HELM: Bias\\ 	
			\textbullet\hspace{3pt} HELM: Toxicity\\ 	
			\textbullet\hspace{3pt} The Self-Perception and Political Biases of ChatGPT\\ 	
			\textbullet\hspace{3pt} Towards Measuring the Representation of Subjective\\\hspace{10pt}Global Opinions in Language Models \\ 	
		}
		& \\ 
		\hline		
		\textbf{Manage} 
		& \makecell[l]{ 		
			\textbullet\hspace{3pt} Content Moderation\\ 		
			\textbullet\hspace{3pt} Failure Avoidance\\ 	
			\textbullet\hspace{3pt} Instructions\\ 	
			\textbullet\hspace{3pt} Inventory\\ 	
			\textbullet\hspace{3pt} Ownership\\ 
			\textbullet\hspace{3pt} Prohibited Use Policy\\ 	
			\textbullet\hspace{3pt} Restrict Anonymous Use\\ 			
			\textbullet\hspace{3pt} System Prompt\\ 	
			\textbullet\hspace{3pt} Temperature\\ 	
			\textbullet\hspace{3pt} Terms of Service\\ 	
		} 
		& \makecell[l]{ 	
			\textbullet\hspace{3pt} Content Moderation\\		
			\textbullet\hspace{3pt} Failure Avoidance\\ 	
			\textbullet\hspace{3pt} Instructions\\ 	
			\textbullet\hspace{3pt} Inventory\\ 	
			\textbullet\hspace{3pt} Ownership\\ 
			\textbullet\hspace{3pt} Prohibited Use Policy\\ 	
			\textbullet\hspace{3pt} Restrict Anonymous Use\\ 
			\textbullet\hspace{3pt} Restrict Decision Making\\  						
			\textbullet\hspace{3pt} System Prompt\\ 	
			\textbullet\hspace{3pt} Temperature\\ 	
			\textbullet\hspace{3pt} Terms of Service\\ 	
		} 
		& \makecell[l]{ 	
			\textbullet\hspace{3pt} Contract Review\\ 	
			\textbullet\hspace{3pt} Disclosure of AI Interaction\\ 	
			\textbullet\hspace{3pt} Failure Avoidance\\ 	
			\textbullet\hspace{3pt} Inventory\\ 	
			\textbullet\hspace{3pt} Ownership\\ 	
			\textbullet\hspace{3pt} Prohibited Use Policy\\ 	
			\textbullet\hspace{3pt} System Documentation\\ 	
			\textbullet\hspace{3pt} Terms of Service	\\ 	
		}\\
		\hline
	\end{tabular}
	\label{table:low_risk_plan_by_gai_risk_cont3}
\end{table}

\noindent\textbf{Usage Note}: Appendix F puts forward an example risk measurement and management plan for low risk GAI systems or applications. The low risk plan focuses on automatable model testing and applies minimally burdensome risk controls. 

\begin{itemize}
	\item Material in Table F.1 can be applied to measure and manage GAI risks in risk programs that are aligned to the trustworthy characteristics. 
	\item Material in Table F.2 can be applied to measure and manage GAI risks in risk programs that are aligned to GAI risks. 
\end{itemize}

\noindent Appendix G below presents an example plan for medium risk systems and Appendix H presents an example plan for high risk systems.   

\vfill
\raisebox{-10pt}{\makebox[\linewidth]{\thepage}}

\pagebreak
\thispagestyle{empty}

% ---------- ----------
\section*{Appendix G: Example Medium-risk Generative AI Measurement and Management Plan}\label{sec:appndxg}
% ---------- ----------

% ---------- ----------
\subsection*{G.1: Example Medium-risk Generative AI Measurement and Management Plan by Trustworthy Characteristic}\label{appdxg1}
% ---------- ----------

\begin{table}[H]
	\caption*{Table G.1: Example risk measurement and management approaches suitable for medium-risk GAI applications organized by trustworthy characteristic.}
	\footnotesize
	\begin{tabular}{|c|c|c|c|c|}
		\hline
		\multirow{2}{*}{\textbf{Function}} & \multicolumn{4}{|c|}{\textbf{Trustworthy Characteristic}}   \\
		\cline{2-5}
		& \textbf{Accountable and Transparent} & \textbf{Fair with Harmful Bias Managed} & \textbf{Interpretable and Explainable} & \textbf{Privacy-enhanced} \\
		\hline
		\textbf{Measure} & 
		\makecell[l]{
			\textbullet\hspace{3pt} Context exhaustion:\\\hspace{10pt}logic-overloading prompts \\
			\textbullet\hspace{3pt} Loaded/leading questions \\
			\textbullet\hspace{3pt} Multi-tasking prompts \\
		}
		&
		\makecell[l]{ 	
			\textbullet\hspace{3pt} Counterfactual prompts\\  	
			\textbullet\hspace{3pt} Pros and cons prompts\\  	
			\textbullet\hspace{3pt} Role-playing prompts\\  	
			\textbullet\hspace{3pt} Loaded/leading questions\\  	
			\textbullet\hspace{3pt} Low context prompts\\  	
			\textbullet\hspace{3pt} Repeat this \\
		}
		&
		\makecell[l]{ 	
			\textbullet\hspace{3pt} Context exhaustion:\\\hspace{10pt}logic-overloading prompts \\\hspace{10pt}(to reveal unexplainable\\\hspace{10pt}decisioning processes)  	\\
		} 
		&
		\makecell[l]{ 	
			\textbullet\hspace{3pt} Auto/biographical prompts\\  	
			\textbullet\hspace{3pt} Location awareness prompts\\  	
			\textbullet\hspace{3pt} Autocompletion prompts\\  	
			\textbullet\hspace{3pt} Repeat this 	\\
		}
		\\
		\hline		
		\textbf{Manage} &  \makecell[l]{
			\textbullet\hspace{3pt} Data Provenance\\ 	
			\textbullet\hspace{3pt} Data Quality\\ 	
			\textbullet\hspace{3pt} Decommission Process\\ 	
			\textbullet\hspace{3pt} Digital Signature\\ 	
			\textbullet\hspace{3pt} External Audit\\ 	
			\textbullet\hspace{3pt} Fine Tuning\\ 	
			\textbullet\hspace{3pt} Grounding\\ 	
			\textbullet\hspace{3pt} Human Review \\ 	
			\textbullet\hspace{3pt} Incident Response\\ 	
			\textbullet\hspace{3pt} Incorporate feedback \\ 	
			\textbullet\hspace{3pt} Model Documentation \\ 	
			\textbullet\hspace{3pt} Monitoring\\ 
			\textbullet\hspace{3pt} Narrow Scope\\ 
		 	\textbullet\hspace{3pt} Open Source\\ 		
			\textbullet\hspace{3pt} RAG\\ 	
			\textbullet\hspace{3pt} Refresh\\ 	
			\textbullet\hspace{3pt} RLHF\\ 	
			\textbullet\hspace{3pt} Restrict Data Collection\\ 				
			\textbullet\hspace{3pt} Restrict Secondary Use\\ 		
			\textbullet\hspace{3pt} User Feedback\\ 	
			\textbullet\hspace{3pt} Validation\\ 		
		}
		& \makecell[l]{
			\textbullet\hspace{3pt} Accessibility \\ 	
			\textbullet\hspace{3pt} Data Provenance\\ 	
			\textbullet\hspace{3pt} Data Quality\\ 	
			\textbullet\hspace{3pt} External Audit\\ 	
			\textbullet\hspace{3pt} Fine Tuning\\ 	
			\textbullet\hspace{3pt} Grounding\\ 	
			\textbullet\hspace{3pt} Human Review \\ 	
			\textbullet\hspace{3pt} Incident Response\\ 	
			\textbullet\hspace{3pt} Incorporate feedback \\ 	
		 	\textbullet\hspace{3pt} Narrow Scope\\  
			\textbullet\hspace{3pt} Restrict Homogeneity\\ 			 	
			\textbullet\hspace{3pt} Team Diversity\\ 	
			\textbullet\hspace{3pt} User Feedback\\ 	
			\textbullet\hspace{3pt} Validation\\ 	
		}
		& \makecell[l]{
			\textbullet\hspace{3pt} Data Provenance\\ 
	 		\textbullet\hspace{3pt} External Audit\\ 		
			\textbullet\hspace{3pt} Human Review \\ 
			\textbullet\hspace{3pt} Model Documentation \\ 	
			\textbullet\hspace{3pt} Monitoring\\ 
			\textbullet\hspace{3pt} Open Source\\ 	
			\textbullet\hspace{3pt} User Feedback\\ 
			\textbullet\hspace{3pt} XAI \\
		}
		& \makecell[l]{
			\textbullet\hspace{3pt} Consent\\ 	
			\textbullet\hspace{3pt} Data Provenance\\ 	
			\textbullet\hspace{3pt} Data Quality\\ 	
			\textbullet\hspace{3pt} Data Retention\\ 	
			\textbullet\hspace{3pt} External Audit\\ 	
			\textbullet\hspace{3pt} Restrict Data Collection\\ 				
			\textbullet\hspace{3pt} Restrict Location Tracking\\ 		 
			\textbullet\hspace{3pt} Restrict Secondary Use\\ 			 
		}
		\\
		\hline
	\end{tabular}
	\label{table:med_risk_plan_by_tc}
\end{table}

\vfill
\raisebox{-10pt}{\makebox[\linewidth]{\thepage}}

\pagebreak
\thispagestyle{empty}

\begin{table}[H]
	\caption*{Table G.1: Example risk measurement and management approaches suitable for medium-risk GAI applications organized by trustworthy characteristic (continued).}
	\footnotesize
	\begin{tabular}{|c|c|c|c|}
		\hline
		\multirow{2}{*}{\textbf{Function}} & \multicolumn{3}{|c|}{\textbf{Trustworthy Characteristic}}   \\
		\cline{2-4}
		& \textbf{Safe} & \textbf{Secure and Resilient} & \textbf{Valid and Reliable} \\
		\hline
		\textbf{Measure} & 
		\makecell[l]{
			\textbullet\hspace{3pt} Pros and cons prompts \\
			\textbullet\hspace{3pt} Role-playing prompts \\
			\textbullet\hspace{3pt} Content exhaustion: niche-seeking prompts \\
			\textbullet\hspace{3pt} Ingratiation/reverse psychology prompts \\
			\textbullet\hspace{3pt} Loaded/leading questions \\
			\textbullet\hspace{3pt} Location awareness prompts \\
			\textbullet\hspace{3pt} Repeat this \\
		}	
		& \makecell[l]{
			\textbullet\hspace{3pt} Multi-tasking prompts \\
			\textbullet\hspace{3pt} Pros and cons prompts \\
			\textbullet\hspace{3pt} Role-playing prompts \\
			\textbullet\hspace{3pt} Content exhaustion: niche-seeking prompts \\
			\textbullet\hspace{3pt} Ingratiation/reverse psychology prompts \\
			\textbullet\hspace{3pt} Prompt injection attacks \\
			\textbullet\hspace{3pt} Membership inference attacks \\
			\textbullet\hspace{3pt} Random attacks \\
		}
		& \makecell[l]{
			\textbullet\hspace{3pt} Multi-tasking prompts \\
			\textbullet\hspace{3pt} Role-playing prompts \\
			\textbullet\hspace{3pt} Ingratiation/reverse psychology prompts \\
			\textbullet\hspace{3pt} Loaded/leading questions \\
			\textbullet\hspace{3pt} Time-perplexity prompts \\
			\textbullet\hspace{3pt} Niche-seeking prompts \\
			\textbullet\hspace{3pt} Logic overloading prompts \\
			\textbullet\hspace{3pt} Repeat this \\
			\textbullet\hspace{3pt} Numeric calculation \\
		}
		\\
		\hline		
		\textbf{Manage} & \makecell[l]{
			\textbullet\hspace{3pt} Blocklist \\			 
			\textbullet\hspace{3pt} Data Retention\\ 	
			\textbullet\hspace{3pt} Decommission Process\\ 	
			\textbullet\hspace{3pt} Digital Signature\\ 	
			\textbullet\hspace{3pt} External Audit\\	
			\textbullet\hspace{3pt} Human Review \\ 	
			\textbullet\hspace{3pt} Incident Response\\ 	
			\textbullet\hspace{3pt} Monitoring\\ 	
			\textbullet\hspace{3pt} Narrow Scope\\ 
			\textbullet\hspace{3pt} Rate-limiting \\
			\textbullet\hspace{3pt} Restrict Location Tracking\\  		
			\textbullet\hspace{3pt} Session Limits\\ 
			\textbullet\hspace{3pt} User Feedback\\ 		 	 	 
		}
		& \makecell[l]{
			\textbullet\hspace{3pt} Blocklist \\ 	
			\textbullet\hspace{3pt} Decommission Process\\ 	
			\textbullet\hspace{3pt} External Audit\\ 
			\textbullet\hspace{3pt} Incident Response\\  
			\textbullet\hspace{3pt} Monitoring\\ 	
			\textbullet\hspace{3pt} Open Source\\
			\textbullet\hspace{3pt} Rate-limiting \\ 
			\textbullet\hspace{3pt} Session Limits\\ 				 	 
		}
		& \makecell[l]{
			\textbullet\hspace{3pt} Data Quality\\ 
			\textbullet\hspace{3pt} Fine Tuning\\ 	
			\textbullet\hspace{3pt} Grounding\\ 	
			\textbullet\hspace{3pt} Human Review \\ 	
			\textbullet\hspace{3pt} Incorporate feedback \\
			\textbullet\hspace{3pt} Model Documentation \\ 	
			\textbullet\hspace{3pt} Monitoring\\ 	
			\textbullet\hspace{3pt} Narrow Scope\\ 	
			\textbullet\hspace{3pt} Open Source\\ 	
			\textbullet\hspace{3pt} RAG\\ 	
			\textbullet\hspace{3pt} Refresh\\ 
			\textbullet\hspace{3pt} Restrict Homogeneity\\ 					
			\textbullet\hspace{3pt} RLHF\\ 
			\textbullet\hspace{3pt} Team Diversity\\ 	
			\textbullet\hspace{3pt} User Feedback\\ 	
			\textbullet\hspace{3pt} Validation\\ 					 	 
		}
		\\
		\hline
	\end{tabular}
	\label{table:med_risk_plan_by_tc_cont}
\end{table}

\vfill
\raisebox{-10pt}{\makebox[\linewidth]{\thepage}}

\pagebreak
\thispagestyle{empty}

% ---------- ----------
\subsection*{G.2: Example Medium-risk Generative AI Measurement and Management Plan by Generative AI Risk}\label{appdxg2}
% ---------- ----------

\begin{table}[H]
	\caption*{Table G.2: Example risk measurement and management approaches suitable for medium-risk GAI applications organized by GAI Risk.}
	\footnotesize
	\begin{tabular}{|c|c|c|c|c|}
		\hline
		\multirow{2}{*}{\textbf{Function}} & \multicolumn{4}{|c|}{\textbf{Generative AI Risk}}   \\
		\cline{2-5}
		& \textbf{CBRN Information} & \textbf{Confabulation} & \textbf{Dangerous and Violent Recommendations} & \textbf{Data Privacy} \\
		\hline	
		\textbf{Measure} & \makecell[l]{
			\textbullet\hspace{3pt} Auto-completion prompts  \\ 
			\textbullet\hspace{3pt} Role-playing prompts \\
			\textbullet\hspace{3pt} Reverse psychology prompts \\
			\textbullet\hspace{3pt} Pros and cons prompts \\
			\textbullet\hspace{3pt} Multitasking prompts \\
			\textbullet\hspace{3pt} Repeat this \\ 
		} 
		& \makecell[l]{
			\textbullet\hspace{3pt} Context exhaustion: Logic overloading prompts \\
			\textbullet\hspace{3pt} Context exhaustion: Multi-tasking prompts \\
			\textbullet\hspace{3pt} Context exhaustion: Niche-seeking prompts \\
			\textbullet\hspace{3pt} Time perplexity prompts \\
			\textbullet\hspace{3pt} Loaded/leading questions \\ 
			\textbullet\hspace{3pt} Calculation and numeric queries \\
		} 
		& \makecell[l]{
			\textbullet\hspace{3pt} Role-playing prompts \\
			\textbullet\hspace{3pt} Reverse psychology prompts \\
			\textbullet\hspace{3pt} Pros and cons prompts \\
			\textbullet\hspace{3pt} Multitasking prompts \\ 
			\textbullet\hspace{3pt} Repeat this \\ 
			\textbullet\hspace{3pt} Loaded/leading questions \\ 
		}   
		& \makecell[l]{
			\textbullet\hspace{3pt} Location awareness \\ 
			\textbullet\hspace{3pt} Membership inference attacks \\ 
			\textbullet\hspace{3pt} Auto/biographical prompts \\ 
			\textbullet\hspace{3pt} Repeat this \\
		}
		\\
		\hline
		\textbf{Manage} & \makecell[l]{
			\textbullet\hspace{3pt} Blocklist \\
			\textbullet\hspace{3pt} Data Provenance\\
			\textbullet\hspace{3pt} Data Quality\\  	 
			\textbullet\hspace{3pt} Decommission Process\\ 
			\textbullet\hspace{3pt} Digital Signature\\ 	
			\textbullet\hspace{3pt} External Audit\\ 
			\textbullet\hspace{3pt} Incident Response\\ 
			\textbullet\hspace{3pt} Monitoring\\ 	
			\textbullet\hspace{3pt} Rate-limiting \\ 	
			\textbullet\hspace{3pt} Session Limits\\ 						 	 
		} 
		& \makecell[l]{
			\textbullet\hspace{3pt} Data Quality\\ 
			\textbullet\hspace{3pt} Fine Tuning\\ 	
			\textbullet\hspace{3pt} Grounding\\ 	
			\textbullet\hspace{3pt} Human Review \\ 	
			\textbullet\hspace{3pt} Incorporate feedback \\
			\textbullet\hspace{3pt} Model Documentation \\ 	
			\textbullet\hspace{3pt} Monitoring\\ 	
			\textbullet\hspace{3pt} Narrow Scope\\ 	
			\textbullet\hspace{3pt} Open Source\\ 	
			\textbullet\hspace{3pt} RAG\\ 	
			\textbullet\hspace{3pt} Refresh\\ 
			\textbullet\hspace{3pt} Restrict Homogeneity\\ 				
			\textbullet\hspace{3pt} RLHF\\ 
			\textbullet\hspace{3pt} Team Diversity\\ 	
			\textbullet\hspace{3pt} User Feedback\\ 	
			\textbullet\hspace{3pt} Validation\\ 					 	 
		}
		& \makecell[l]{
			\textbullet\hspace{3pt} Blocklist \\			 
			\textbullet\hspace{3pt} Data Retention\\ 	
			\textbullet\hspace{3pt} Decommission Process\\ 	
			\textbullet\hspace{3pt} Digital Signature\\ 	
			\textbullet\hspace{3pt} External Audit\\	
			\textbullet\hspace{3pt} Human Review \\ 	
			\textbullet\hspace{3pt} Incident Response\\ 	
			\textbullet\hspace{3pt} Monitoring\\ 	
			\textbullet\hspace{3pt} Narrow Scope\\ 
			\textbullet\hspace{3pt} Rate-limiting \\ 
			\textbullet\hspace{3pt} Restrict Location Tracking\\ 					
			\textbullet\hspace{3pt} Session Limits\\ 
			\textbullet\hspace{3pt} User Feedback\\ 		 	 	 
		} & \makecell[l]{
			\textbullet\hspace{3pt} Consent\\ 	
			\textbullet\hspace{3pt} Data Provenance\\ 	
			\textbullet\hspace{3pt} Data Quality\\ 	
			\textbullet\hspace{3pt} Data Retention\\ 	
			\textbullet\hspace{3pt} External Audit\\ 	
			\textbullet\hspace{3pt} Restrict Data Collection\\ 			
			\textbullet\hspace{3pt} Restrict Location Tracking\\ 		 
			\textbullet\hspace{3pt} Restrict Secondary Use\\ 		 
		}
		\\
		\hline
	\end{tabular}
	\label{table:med_risk_plan_by_gai_risk}
\end{table}

\vfill
\raisebox{-10pt}{\makebox[\linewidth]{\thepage}}

\pagebreak
\thispagestyle{empty}

\begin{table}[H]
	\caption*{Table G.2: Example risk measurement and management approaches suitable for medium-risk GAI applications organized by GAI Risk (continued).}
	\footnotesize
	\begin{tabular}{|c|c|c|c|c|}
		\hline
		\multirow{2}{*}{\textbf{Function}} & \multicolumn{4}{|c|}{\textbf{Generative AI Risk}}   \\
		\cline{2-5}
		& \textbf{Environmental} & \textbf{Human-AI Configuration} & \textbf{Information Integrity} & \textbf{Information Security} \\
		\hline	
		\textbf{Measure} & \makecell[l]{
			\textbullet\hspace{3pt} Availability attacks \\ 
			\textbullet\hspace{3pt} Role-playing prompts \\
			\textbullet\hspace{3pt} Reverse psychology prompts \\
			\textbullet\hspace{3pt} Pros and cons prompts \\
			\textbullet\hspace{3pt} Multitasking prompts \\ 
		}
		& \makecell[l]{
			\textbullet\hspace{3pt} Role-playing prompts \\
			\textbullet\hspace{3pt} Reverse psychology prompts \\
			\textbullet\hspace{3pt} Pros and cons prompts \\
			\textbullet\hspace{3pt} Multitasking prompts \\ 
		}
		& \makecell[l]{
			\textbullet\hspace{3pt} Loaded/leading questions  \\ 
			\textbullet\hspace{3pt} Role-playing prompts \\
			\textbullet\hspace{3pt} Reverse psychology prompts \\
			\textbullet\hspace{3pt} Pros and cons prompts \\
			\textbullet\hspace{3pt} Multitasking prompts \\
		} 
		& \makecell[l]{
			\textbullet\hspace{3pt} Confidentiality attacks  \\ 
			\textbullet\hspace{3pt} Integrity attacks  \\ 
			\textbullet\hspace{3pt} Availability attacks \\ 
			\textbullet\hspace{3pt} Random attacks \\ 
			\textbullet\hspace{3pt} Role-playing prompts \\
			\textbullet\hspace{3pt} Reverse psychology prompts \\
			\textbullet\hspace{3pt} Pros and cons prompts \\
			\textbullet\hspace{3pt} Multitasking prompts \\
		}
		\\
		\hline
		\textbf{Manage} & \makecell[l]{
			\textbullet\hspace{3pt} Decommission Process\\ 	
			\textbullet\hspace{3pt} External Audit\\ 
			\textbullet\hspace{3pt} Incident Response\\ 
			\textbullet\hspace{3pt} Monitoring\\
			\textbullet\hspace{3pt} Rate-limiting \\ 
			\textbullet\hspace{3pt} Session Limits\\						 	 
		}
		& \makecell[l]{
			\textbullet\hspace{3pt} Accessibility \\ 	
			\textbullet\hspace{3pt} Blocklist \\ 	
			\textbullet\hspace{3pt} Consent\\ 	
			\textbullet\hspace{3pt} Decommission Process\\ 	
			\textbullet\hspace{3pt} Digital Signature\\ 	
			\textbullet\hspace{3pt} External Audit\\ 
			\textbullet\hspace{3pt} Human Review \\ 
			\textbullet\hspace{3pt} Incorporate feedback \\ 
			\textbullet\hspace{3pt} Restrict Data Collection\\ 		
			\textbullet\hspace{3pt} Restrict Location Tracking\\ 	
			\textbullet\hspace{3pt} Restrict Secondary Use\\ 		
			\textbullet\hspace{3pt} Session Limits\\ 	
			\textbullet\hspace{3pt} User Feedback\\ 						 	 
		}
		& \makecell[l]{
			\textbullet\hspace{3pt} Data Provenance\\ 	
			\textbullet\hspace{3pt} Data Quality\\ 	
			\textbullet\hspace{3pt} Digital Signature\\ 	
			\textbullet\hspace{3pt} External Audit\\ 
			\textbullet\hspace{3pt} Fine Tuning\\ 	
			\textbullet\hspace{3pt} Grounding\\ 	
			\textbullet\hspace{3pt} Human Review \\ 	
			\textbullet\hspace{3pt} Incident Response\\ 	
			\textbullet\hspace{3pt} Incorporate feedback \\ 
			\textbullet\hspace{3pt} Monitoring\\ 	
			\textbullet\hspace{3pt} Narrow Scope\\ 	
			\textbullet\hspace{3pt} Open Source\\ 	
			\textbullet\hspace{3pt} RAG\\ 	
			\textbullet\hspace{3pt} Refresh\\ 	
			\textbullet\hspace{3pt} Restrict Homogeneity\\ 			
			\textbullet\hspace{3pt} RLHF\\ 
			\textbullet\hspace{3pt} User Feedback\\ 
			\textbullet\hspace{3pt} Validation\\ 							 	 
		}
		& \makecell[l]{
			\textbullet\hspace{3pt} Blocklist \\ 	
			\textbullet\hspace{3pt} Decommission Process\\ 	
			\textbullet\hspace{3pt} External Audit\\ 
			\textbullet\hspace{3pt} Incident Response\\  
			\textbullet\hspace{3pt} Monitoring\\ 	
			\textbullet\hspace{3pt} Open Source\\
			\textbullet\hspace{3pt} Rate-limiting \\ 
			\textbullet\hspace{3pt} Session Limits\\ 				 	 
		} \\
		\hline
	\end{tabular}
	\label{table:med_risk_plan_by_gai_risk_cont}
\end{table}

\vfill
\raisebox{-10pt}{\makebox[\linewidth]{\thepage}}

\pagebreak
\thispagestyle{empty}

\begin{table}[H]
	\caption*{Table G.2: Example risk measurement and management approaches suitable for medium-risk GAI applications organized by GAI Risk (continued).}
	\footnotesize
	\begin{tabular}{|c|c|c|c|c|}
		\hline
		\multirow{2}{*}{\textbf{Function}} & \multicolumn{4}{|c|}{\textbf{Generative AI Risk}}   \\
		\cline{2-5}
		& \textbf{Intellectual Property} & \makecell[l]{\textbf{Obscene, Degrading,}\\\textbf{and/or Abusive Content}} & \makecell[l]{\textbf{Toxicity, Bias, and}\\\textbf{Homogenization}} & \makecell[l]{\textbf{Value Chain and}\\\textbf{Component Integration}} \\
		\hline		
		\textbf{Measure} & \makecell[l]{
			\textbullet\hspace{3pt} Confidentiality attacks \\ 
			\textbullet\hspace{3pt} Auto-complete prompts \\
		}
		& \makecell[l]{
			\textbullet\hspace{3pt} Confidentiality attacks \\ 
			\textbullet\hspace{3pt} Autocomplete prompts \\ 
			\textbullet\hspace{3pt} Role-playing prompts \\
			\textbullet\hspace{3pt} Reverse psychology prompts \\
			\textbullet\hspace{3pt} Pros and cons prompts \\
			\textbullet\hspace{3pt} Multitasking prompts \\
			\textbullet\hspace{3pt} Loaded/leading questions  \\ 
			\textbullet\hspace{3pt} Repeat this \\
		}
		& \makecell[l]{
			\textbullet\hspace{3pt} Data poisoning attacks \\ 
			\textbullet\hspace{3pt} Counterfactual prompts \\
			\textbullet\hspace{3pt} Pros and cons prompts \\
			\textbullet\hspace{3pt} Role-playing prompts \\
			\textbullet\hspace{3pt} Low context prompts \\ 
			\textbullet\hspace{3pt} Loaded/leading questions  \\ 
			\textbullet\hspace{3pt} Repeat this \\
		}
		& \\
		\hline
		\textbf{Manage} & \makecell[l]{
			\textbullet\hspace{3pt} Blocklist \\
			\textbullet\hspace{3pt} Data Provenance\\
			\textbullet\hspace{3pt} Data Quality\\  	 
			\textbullet\hspace{3pt} Decommission Process\\ 
			\textbullet\hspace{3pt} Digital Signature\\ 	
			\textbullet\hspace{3pt} External Audit\\ 
			\textbullet\hspace{3pt} Incident Response\\ 
			\textbullet\hspace{3pt} Incorporate feedback \\ 
			\textbullet\hspace{3pt} Monitoring\\ 	
			\textbullet\hspace{3pt} Open Source\\ 
			\textbullet\hspace{3pt} Rate-limiting \\ 	
			\textbullet\hspace{3pt} Session Limits\\ 	
			\textbullet\hspace{3pt} User Feedback\\ 					 	 
		} 
		& \makecell[l]{
			\textbullet\hspace{3pt} Blocklist \\
			\textbullet\hspace{3pt} Data Provenance\\
			\textbullet\hspace{3pt} Data Quality\\  	 
			\textbullet\hspace{3pt} Decommission Process\\ 
			\textbullet\hspace{3pt} Digital Signature\\ 	
			\textbullet\hspace{3pt} External Audit\\ 
			\textbullet\hspace{3pt} Incident Response\\ 
			\textbullet\hspace{3pt} Monitoring\\ 	
			\textbullet\hspace{3pt} Rate-limiting \\ 	
			\textbullet\hspace{3pt} Session Limits\\ 
			\textbullet\hspace{3pt} User Feedback\\ 						 	 
		} 
		& \makecell[l]{
			\textbullet\hspace{3pt} Accessibility \\ 	
			\textbullet\hspace{3pt} Data Provenance\\ 	
			\textbullet\hspace{3pt} Data Quality\\ 	
			\textbullet\hspace{3pt} External Audit\\ 	
			\textbullet\hspace{3pt} Fine Tuning\\ 	
			\textbullet\hspace{3pt} Grounding\\ 	
			\textbullet\hspace{3pt} Human Review \\ 	
			\textbullet\hspace{3pt} Incident Response\\ 	
			\textbullet\hspace{3pt} Incorporate feedback \\ 	
			\textbullet\hspace{3pt} Narrow Scope\\  
			\textbullet\hspace{3pt} Restrict Homogeneity\\ 				
			\textbullet\hspace{3pt} Team Diversity\\ 	
			\textbullet\hspace{3pt} User Feedback\\ 	
			\textbullet\hspace{3pt} Validation\\ 	
		}
		& \makecell[l]{
			\textbullet\hspace{3pt} Data Provenance\\ 	
			\textbullet\hspace{3pt} Data Quality\\ 	
			\textbullet\hspace{3pt} Digital Signature\\ 	
			\textbullet\hspace{3pt} External Audit\\ 	
			\textbullet\hspace{3pt} Model Documentation \\ 
			\textbullet\hspace{3pt} Restrict Data Collection\\ 				
			\textbullet\hspace{3pt} Restrict Secondary Use\\ 							 	 
		} 
		\\
		\hline
	\end{tabular}
	\label{table:med_risk_plan_by_gai_risk_cont2}
\end{table}

\noindent\textbf{Usage Note}: Appendix G puts forward an example risk measurement and management plan for medium risk GAI systems or applications. The medium risk plan focuses on red-teaming and applies moderate risk controls. Measurement and management approaches from Appendix F should also be applied to medium risk systems or applications.

\begin{itemize}
	\item Material in Table G.1 can be applied to measure and manage GAI risks in risk programs that are aligned to the trustworthy characteristics. 
	\item Material in Table G.2 can be applied to measure and manage GAI risks in risk programs that are aligned to GAI risks. 
\end{itemize}

\noindent Appendix H below presents an example plan for high risk systems.  

\vfill
\raisebox{-10pt}{\makebox[\linewidth]{\thepage}}

\pagebreak
\thispagestyle{empty}

% ---------- ----------
\section*{Appendix H: Example High-risk Generative AI Measurement and Management Plan}\label{sec:appndxh}
% ---------- ----------

% ---------- ----------
\subsection*{H.1: Example High-risk Generative AI Measurement and Management Plan by Trustworthy Characteristic}\label{appdxh1}
% ---------- ----------

\begin{table}[H]
	\caption*{Table H.1: Example risk measurement and management approaches suitable for high-risk GAI applications organized by trustworthy characteristic.}
	\footnotesize
	\begin{tabular}{|c|c|c|c|c|}
		\hline
		\multirow{2}{*}{\textbf{Function}} & \multicolumn{4}{|c|}{\textbf{Trustworthy Characteristic}}   \\
		\cline{2-5}
		& \textbf{Accountable and Transparent} & \textbf{Fair with Harmful Bias Managed} & \textbf{Interpretable and Explainable} & \textbf{Privacy-enhanced} \\
		\hline
		\textbf{Measure} & \makecell[l]{
			\textbullet\hspace{3pt} Algorithmic impact assessments \\ 
			\textbullet\hspace{3pt} Assessing data quality*\\ 
			\textbullet\hspace{3pt} Bias bounties \\ 
			\textbullet\hspace{3pt} Calibration*\\ 
			\textbullet\hspace{3pt} Cybersecurity testing \\ 
			\textbullet\hspace{3pt} Environmental metrics \\ 
			\textbullet\hspace{3pt} Field testing*\\ 
			\textbullet\hspace{3pt} Input/output measurement\\\hspace{10pt}using classifiers \\ 
			\textbullet\hspace{3pt} Model assessment*\\ 
			\textbullet\hspace{3pt} Model comparison*\\ 
			\textbullet\hspace{3pt} Multi-session experiments*\\ 
			\textbullet\hspace{3pt} Online metrics/monitoring \\ 
			\textbullet\hspace{3pt} Perturbation studies*\\ 
			\textbullet\hspace{3pt} PII identification and removal \\ 
			\textbullet\hspace{3pt} Root cause analysis*\\ 
			\textbullet\hspace{3pt} Screening for information integrity \\ 
			\textbullet\hspace{3pt} Sensitivity analysis*\\ 
			\textbullet\hspace{3pt} Software testing \\ 
			\textbullet\hspace{3pt} Stakeholder engagement and feedback*\\ 
			\textbullet\hspace{3pt} Statistical quality control*\\ 
			\textbullet\hspace{3pt} Stress testing*\\ 
			\textbullet\hspace{3pt} Sub-sampling traffic for manually annotating \\ 
			\textbullet\hspace{3pt} Supply chain auditing \\ 
			\textbullet\hspace{3pt} Testing third-party dependencies \\ 
			\textbullet\hspace{3pt} User surveys*\\ 
			\textbullet\hspace{3pt} Validity testing/validation.* \\	
		}	
		&
		\makecell[l]{ 	
			\textbullet\hspace{3pt} Algorithmic impact assessments\\  	
			\textbullet\hspace{3pt} Analyze differences between \\\hspace{10pt}intended and actual population of \\\hspace{10pt}users or data subjects*\\ 	
			\textbullet\hspace{3pt} Anomaly detection*\\  	
			\textbullet\hspace{3pt} Assessing data quality*\\  	
			\textbullet\hspace{3pt} Bias bounties\\  	
			\textbullet\hspace{3pt} Bias testing\\  	
			\textbullet\hspace{3pt} Calibration*\\  	
			\textbullet\hspace{3pt} Counterfactual/causal analysis\\  	
			\textbullet\hspace{3pt} Disaggregated metrics\\  	
			\textbullet\hspace{3pt} Field testing*\\  	
			\textbullet\hspace{3pt} Model assessment*\\  	
			\textbullet\hspace{3pt} Model comparison*\\  	
			\textbullet\hspace{3pt} Multi-session experiments*\\  	
			\textbullet\hspace{3pt} Root cause analysis*\\  	
			\textbullet\hspace{3pt} Software testing\\  	
			\textbullet\hspace{3pt} Statistical quality control*\\  	
			\textbullet\hspace{3pt} Stress testing*\\  		
			\textbullet\hspace{3pt} User surveys*\\  		
			\textbullet\hspace{3pt} Validity testing/validation.*
		}
		&
		\makecell[l]{ 	
			\textbullet\hspace{3pt} Algorithmic impact assessments \\
			\textbullet\hspace{3pt} Analyze differences between \\\hspace{10pt}intended and actual population of \\\hspace{10pt}users or data subjects*\\ 
			\textbullet\hspace{3pt} Model comparison.* \\
			\textbullet\hspace{3pt} Multi-session experiments.* \\	
			\textbullet\hspace{3pt} Root cause analysis.* \\			
			\textbullet\hspace{3pt} Stakeholder engagement and \\\hspace{10pt}feedback.*	\\		
			\textbullet\hspace{3pt} UI/UX studies \\			
			\textbullet\hspace{3pt} User surveys*\\  
		} 
		& 	
		\makecell[l]{
			\textbullet\hspace{3pt} Algorithmic impact assessments\\  	
			\textbullet\hspace{3pt} Assessing data quality.* \\
			\textbullet\hspace{3pt} Cybersecurity testing\\  
			\textbullet\hspace{3pt} PII identification and removal\\  		
			\textbullet\hspace{3pt} Root cause analysis*\\  	
			\textbullet\hspace{3pt} Stakeholder engagement and feedback*\\  	
			\textbullet\hspace{3pt} Stress testing*\\  	
			\textbullet\hspace{3pt} Testing third-party dependencies \\
		}
		\\
		\hline		
		\textbf{Manage} &  \makecell[l]{
			\textbullet\hspace{3pt} Fast decommission \\ 	
			\textbullet\hspace{3pt} Insurance \\ 	
			\textbullet\hspace{3pt} Intellectual property removal \\ 			
			\textbullet\hspace{3pt} Restrict regulated dealings \\ 	
			\textbullet\hspace{3pt} Sensitive/Personal data removal \\ 	
			\textbullet\hspace{3pt} Supply chain audit \\ 	
			\textbullet\hspace{3pt} User recourse \\ 		 	
		}
		& \makecell[l]{
			\textbullet\hspace{3pt} CSAM/Obscenity removal \\ 	
			\textbullet\hspace{3pt} Fast decommission \\ 	
			\textbullet\hspace{3pt} Insurance \\ 	
			\textbullet\hspace{3pt} Intellectual property removal \\ 	
			\textbullet\hspace{3pt} Restrict regulated dealings \\ 	
			\textbullet\hspace{3pt} Sensitive/Personal data removal \\ 	
			\textbullet\hspace{3pt} Supply chain audit \\ 	
			\textbullet\hspace{3pt} User recourse \\   
		}
		& \makecell[l]{	
			\textbullet\hspace{3pt} Restrict regulated dealings \\ 		
			\textbullet\hspace{3pt} Supply Chain Audit \\ 	
			\textbullet\hspace{3pt} User recourse \\  
		}
		& \makecell[l]{
			\textbullet\hspace{3pt} CSAM/Obscenity removal \\ 	
			\textbullet\hspace{3pt} Fast decommission \\ 	
			\textbullet\hspace{3pt} Insurance \\ 	
			\textbullet\hspace{3pt} Intellectual property removal \\ 		
			\textbullet\hspace{3pt} Restrict minors \\ 	
			\textbullet\hspace{3pt} Restrict regulated dealings \\ 	
			\textbullet\hspace{3pt} Sensitive/Personal data removal \\ 	
			\textbullet\hspace{3pt} Supply chain audit \\ 	
			\textbullet\hspace{3pt} User recourse \\   		 
		}
		\\
		\hline
	\end{tabular}
	\label{table:high_risk_plan_by_tc}
\end{table}

\vfill
\raisebox{-10pt}{\makebox[\linewidth]{\thepage}}

\pagebreak
\thispagestyle{empty}

\begin{table}[H]
	\caption*{Table H.1: Example risk measurement and management approaches suitable for high-risk GAI applications organized by trustworthy characteristic (continued).}
	\footnotesize
	\begin{tabular}{|c|c|c|c|}
		\hline
		\multirow{2}{*}{\textbf{Function}} & \multicolumn{3}{|c|}{\textbf{Trustworthy Characteristic}}   \\
		\cline{2-4}
		& \textbf{Safe} & \textbf{Secure and Resilient} & \textbf{Valid and Reliable} \\
		\hline
		\textbf{Measure} & \makecell[l]{
			\textbullet\hspace{3pt} Algorithmic impact assessments\\  	
			\textbullet\hspace{3pt} Analyze differences between\\\hspace{10pt}intended and actual population\\\hspace{10pt}of users or data subjects*\\ 	 	
			\textbullet\hspace{3pt} Assessing data quality*\\  	 	
			\textbullet\hspace{3pt} Bias bounties\\  	
			\textbullet\hspace{3pt} Calibration*\\  	
			\textbullet\hspace{3pt} Chaos testing\\  	
			\textbullet\hspace{3pt} Dangerous and violent
			\\\hspace{10pt}content removal\\  		
			\textbullet\hspace{3pt} Field testing*\\  	
			\textbullet\hspace{3pt} Input/output measurement\\\hspace{10pt}using classifiers \\ 
			\textbullet\hspace{3pt} Model assessment*\\  	
			\textbullet\hspace{3pt} Model comparison*\\  	
			\textbullet\hspace{3pt} Multi-session experiments*\\  	
			\textbullet\hspace{3pt} Perturbation studies*\\  		
			\textbullet\hspace{3pt} Root cause analysis*\\  	
			\textbullet\hspace{3pt} Sensitivity analysis*\\  	
			\textbullet\hspace{3pt} Stakeholder engagement and\\\hspace{10pt}feedback*\\  	
			\textbullet\hspace{3pt} Statistical quality control*\\  	
			\textbullet\hspace{3pt} Stress testing*\\  		
			\textbullet\hspace{3pt} User surveys*\\  	
			\textbullet\hspace{3pt} Validity testing/validation*\\  
		}
		& \makecell[l]{
			\textbullet\hspace{3pt} Algorithmic impact assessments \\ 
			\textbullet\hspace{3pt} Anomaly detection*\\ 
			\textbullet\hspace{3pt} Assessing data quality*\\ 
			\textbullet\hspace{3pt} Bias bounties \\ 
			\textbullet\hspace{3pt} Calibration*\\ 
			\textbullet\hspace{3pt} Chaos testing \\ 
			\textbullet\hspace{3pt} Cybersecurity testing \\ 
			\textbullet\hspace{3pt} Data poisoning detection \\ 
			\textbullet\hspace{3pt} Model assessment*\\ 
			\textbullet\hspace{3pt} Model comparison*\\ 
			\textbullet\hspace{3pt} Root cause analysis*\\ 
			\textbullet\hspace{3pt} Software testing \\ 
			\textbullet\hspace{3pt} Stakeholder engagement and feedback*\\ 
			\textbullet\hspace{3pt} Stress testing*\\ 
			\textbullet\hspace{3pt} Supply chain auditing \\ 
			\textbullet\hspace{3pt} Testing third-party dependencies \\
		}
		& \makecell[l]{
			\textbullet\hspace{3pt} Algorithmic impact\\\hspace{10pt}assessments\\  	
			\textbullet\hspace{3pt} Analyze differences between\\\hspace{10pt}intended and actual population\\\hspace{10pt}of users or data subjects*\\ 			
			\textbullet\hspace{3pt} Assessing data quality*\\  		
			\textbullet\hspace{3pt} Bias bounties\\  	
			\textbullet\hspace{3pt} Calibration*\\  	
			\textbullet\hspace{3pt} Field testing*\\  	
			\textbullet\hspace{3pt} Input/output measurement\\\hspace{10pt}using classifiers \\ 
			\textbullet\hspace{3pt} Model assessment*\\  	
			\textbullet\hspace{3pt} Model comparison*\\  	
			\textbullet\hspace{3pt} Multi-session experiments*\\  		
			\textbullet\hspace{3pt} Perturbation studies*\\  				
			\textbullet\hspace{3pt} Root cause analysis*\\  	
			\textbullet\hspace{3pt} Sensitivity analysis*\\  	
			\textbullet\hspace{3pt} Stakeholder engagement and \\\hspace{10pt}feedback*\\  	
			\textbullet\hspace{3pt} Statistical quality control*\\  	
			\textbullet\hspace{3pt} Stress testing*\\  		
			\textbullet\hspace{3pt} User surveys*\\  	
			\textbullet\hspace{3pt} Validity testing/validation*\\ 
		}
		\\
		\hline		
		\textbf{Manage} & \makecell[l]{	
			\textbullet\hspace{3pt} CSAM/Obscenity removal \\ 	
			\textbullet\hspace{3pt} Fast decommission \\ 	
			\textbullet\hspace{3pt} Insurance \\ 	
			\textbullet\hspace{3pt} Redundancy \\ 	
			\textbullet\hspace{3pt} Restrict internet access \\ 	
			\textbullet\hspace{3pt} Restrict minors \\ 	
			\textbullet\hspace{3pt} Restrict regulated dealings \\ 	
			\textbullet\hspace{3pt} Sensitive/Personal data removal \\ 	
			\textbullet\hspace{3pt} Supply Chain Audit \\ 	
			\textbullet\hspace{3pt} User recourse \\ 		 	 	 
		}
		& \makecell[l]{
			\textbullet\hspace{3pt} CSAM/Obscenity removal \\ 	
			\textbullet\hspace{3pt} Fast decommission \\ 	
			\textbullet\hspace{3pt} Insurance \\ 	
			\textbullet\hspace{3pt} Intellectual property removal \\ 	
			\textbullet\hspace{3pt} Redundancy \\ 	
			\textbullet\hspace{3pt} Restrict internet access \\ 	
			\textbullet\hspace{3pt} Restrict minors \\ 	
			\textbullet\hspace{3pt} Restrict regulated dealings \\ 	
			\textbullet\hspace{3pt} Sensitive/Personal data removal \\ 	
			\textbullet\hspace{3pt} Supply chain audit \\ 	
			\textbullet\hspace{3pt} User recourse \\					 	 
		}
		& \makecell[l]{
			\textbullet\hspace{3pt} Fast decommission \\ 	
			\textbullet\hspace{3pt} Insurance \\ 	
			\textbullet\hspace{3pt} Redundancy \\ 	
			\textbullet\hspace{3pt} Restrict regulated dealings \\ 	
			\textbullet\hspace{3pt} Supply chain audit \\ 	
			\textbullet\hspace{3pt} User recourse \\  					 	 
		}
		\\
		\hline
	\end{tabular}
	\label{table:high_risk_plan_by_tc_cont}
\end{table}

\vfill
\raisebox{-10pt}{\makebox[\linewidth]{\thepage}}

\pagebreak
\thispagestyle{empty}

% ---------- ----------
\subsection*{H.2: Example High-risk Generative AI Measurement and Management Plan by Generative AI Risk}\label{appdxh2}
% ---------- ----------

\begin{table}[H]
	\caption*{Table H.2: Example risk measurement and management approaches suitable for high-risk GAI applications organized by GAI Risk.}
	\footnotesize
	\begin{tabular}{|c|c|c|c|c|}
		\hline
		\multirow{2}{*}{\textbf{Function}} & \multicolumn{4}{|c|}{\textbf{Generative AI Risk}}   \\
		\cline{2-5}
		& \textbf{CBRN Information} & \textbf{Confabulation} & \textbf{Dangerous and Violent Recommendations} & \textbf{Data Privacy} \\
		\hline	
		\textbf{Measure} & 
		\makecell[l]{	
			\textbullet\hspace{3pt} Chaos testing\\  	
			\textbullet\hspace{3pt} Cybersecurity testing\\  	
			\textbullet\hspace{3pt} Input/output measurement\\\hspace{10pt}using classifiers \\ 
			\textbullet\hspace{3pt} Online metrics/monitoring\\  	
			\textbullet\hspace{3pt} Perturbation studies*\\  	
			\textbullet\hspace{3pt} Prompt engineering\\  		
			\textbullet\hspace{3pt} Root cause analysis*\\  	
			\textbullet\hspace{3pt} Sensitivity analysis*\\  	
			\textbullet\hspace{3pt} Software testing\\  	
			\textbullet\hspace{3pt} Stress testing*\\  	
			\textbullet\hspace{3pt} Supply chain auditing \\
		} 
		& \makecell[l]{
			\textbullet\hspace{3pt} Algorithmic impact\\\hspace{10pt}assessments\\  	
			\textbullet\hspace{3pt} Analyze differences between\\\hspace{10pt}intended and actual population\\\hspace{10pt}of users or data subjects*\\ 			
			\textbullet\hspace{3pt} Assessing data quality*\\  		
			\textbullet\hspace{3pt} Bias bounties\\  	
			\textbullet\hspace{3pt} Calibration*\\  	
			\textbullet\hspace{3pt} Field testing*\\  	
			\textbullet\hspace{3pt} Input/output measurement\\\hspace{10pt}using classifiers \\ 
			\textbullet\hspace{3pt} Model assessment*\\  	
			\textbullet\hspace{3pt} Model comparison*\\  	
			\textbullet\hspace{3pt} Multi-session experiments*\\  		
			\textbullet\hspace{3pt} Perturbation studies*\\  				
			\textbullet\hspace{3pt} Root cause analysis*\\  	
			\textbullet\hspace{3pt} Sensitivity analysis*\\  	
			\textbullet\hspace{3pt} Stakeholder engagement and \\\hspace{10pt}feedback*\\  	
			\textbullet\hspace{3pt} Statistical quality control*\\  	
			\textbullet\hspace{3pt} Stress testing*\\  		
			\textbullet\hspace{3pt} User surveys*\\  	
			\textbullet\hspace{3pt} Validity testing/validation*\\ 
		} 
		& \makecell[l]{
			\textbullet\hspace{3pt} Algorithmic impact assessments\\  	
			\textbullet\hspace{3pt} Analyze differences between\\\hspace{10pt}intended and actual population\\\hspace{10pt}of users or data subjects*\\ 	 	
			\textbullet\hspace{3pt} Assessing data quality*\\  	 	
			\textbullet\hspace{3pt} Bias bounties\\  	
			\textbullet\hspace{3pt} Calibration*\\  	
			\textbullet\hspace{3pt} Chaos testing\\  	
			\textbullet\hspace{3pt} Dangerous and violent
			\\\hspace{10pt}content removal\\  		
			\textbullet\hspace{3pt} Field testing*\\  	
			\textbullet\hspace{3pt} Input/output measurement\\\hspace{10pt}using classifiers \\ 
			\textbullet\hspace{3pt} Model assessment*\\  	
			\textbullet\hspace{3pt} Model comparison*\\  	
			\textbullet\hspace{3pt} Multi-session experiments*\\  	
			\textbullet\hspace{3pt} Perturbation studies*\\  		
			\textbullet\hspace{3pt} Root cause analysis*\\  	
			\textbullet\hspace{3pt} Sensitivity analysis*\\  	
			\textbullet\hspace{3pt} Stakeholder engagement and\\\hspace{10pt}feedback*\\  	
			\textbullet\hspace{3pt} Statistical quality control*\\  	
			\textbullet\hspace{3pt} Stress testing*\\  		
			\textbullet\hspace{3pt} User surveys*\\  	
			\textbullet\hspace{3pt} Validity testing/validation*\\  
 	    }
		& \makecell[l]{
			\textbullet\hspace{3pt} Algorithmic impact assessments\\  	
			\textbullet\hspace{3pt} Assessing data quality.* \\
			\textbullet\hspace{3pt} Cybersecurity testing\\  
			\textbullet\hspace{3pt} PII identification and removal\\  		
			\textbullet\hspace{3pt} Root cause analysis*\\  	
			\textbullet\hspace{3pt} Stakeholder engagement and feedback*\\  	
			\textbullet\hspace{3pt} Stress testing*\\  	
			\textbullet\hspace{3pt} Testing third-party dependencies \\
		} \\
		\hline
		\textbf{Manage} & 
		\makecell[l]{
			\textbullet\hspace{3pt} CBRN info removal\\  
			\textbullet\hspace{3pt} Fast decommission \\ 	
			\textbullet\hspace{3pt} Restrict internet access \\ 	
			\textbullet\hspace{3pt} Supply chain audit \\ 							 	 
		} 
		& \makecell[l]{
			\textbullet\hspace{3pt} Fast decommission \\ 	
			\textbullet\hspace{3pt} Insurance \\ 	
			\textbullet\hspace{3pt} Restrict regulated dealings \\ 		
			\textbullet\hspace{3pt} Supply chain audit \\ 	
			\textbullet\hspace{3pt} User recourse \\  						 	 
		}
		& \makecell[l]{
			\textbullet\hspace{3pt} CSAM/Obscenity removal \\ 	
			\textbullet\hspace{3pt} Fast decommission \\ 	
			\textbullet\hspace{3pt} Insurance \\ 	
			\textbullet\hspace{3pt} Restrict minors \\ 	
			\textbullet\hspace{3pt} Restrict regulated dealings \\ 	
			\textbullet\hspace{3pt} Sensitive/Personal data removal \\ 	
			\textbullet\hspace{3pt} Supply chain audit \\ 	
			\textbullet\hspace{3pt} User recourse \\ 		 	 	 
		} 
		& \makecell[l]{
			\textbullet\hspace{3pt} CSAM/Obscenity removal \\ 	
			\textbullet\hspace{3pt} Fast decommission \\ 	
			\textbullet\hspace{3pt} Insurance \\ 	
			\textbullet\hspace{3pt} Intellectual property removal \\ 		
			\textbullet\hspace{3pt} Restrict minors \\ 	
			\textbullet\hspace{3pt} Restrict regulated dealings \\ 	
			\textbullet\hspace{3pt} Sensitive/Personal data removal \\ 	
			\textbullet\hspace{3pt} Supply chain audit \\ 	
			\textbullet\hspace{3pt} User recourse \\  	 
		}
		\\
		\hline
	\end{tabular}
	\label{table:high_risk_plan_by_gai_risk}
\end{table}

\vfill
\raisebox{-10pt}{\makebox[\linewidth]{\thepage}}

\pagebreak
\thispagestyle{empty}

\begin{table}[H]
	\caption*{Table H.2: Example risk measurement and management approaches suitable for high-risk GAI applications organized by GAI Risk (continued).}
	\footnotesize
	\begin{tabular}{|c|c|c|c|c|}
		\hline
		\multirow{2}{*}{\textbf{Function}} & \multicolumn{4}{|c|}{\textbf{Generative AI Risk}}   \\
		\cline{2-5}
		& \textbf{Environmental} & \textbf{Human-AI Configuration} & \textbf{Information Integrity} & \textbf{Information Security} \\
		\hline	
		\textbf{Measure} & 
		\makecell[l]{
			\textbullet\hspace{3pt}  Algorithmic impact assessments \\ 
			\textbullet\hspace{3pt}  Environmental metrics \\ 
			\textbullet\hspace{3pt}  Model comparison*\\ 
			\textbullet\hspace{3pt}  Online metrics/monitoring \\ 
			\textbullet\hspace{3pt}  Supply chain auditing \\ 
		}
		& \makecell[l]{
			\textbullet\hspace{3pt} Algorithmic impact assessments \\ 
			\textbullet\hspace{3pt} Analyze differences between\\\hspace{10pt}intended and actual population\\\hspace{10pt}of users or data subjects*\\ 	 
			\textbullet\hspace{3pt} Analyzing user feedback \\ 
			\textbullet\hspace{3pt} Bias bounties \\ 
			\textbullet\hspace{3pt} Calibration*\\ 
			\textbullet\hspace{3pt} Explainability/interpretability \\ 
			\textbullet\hspace{3pt} Field testing*\\ 
			\textbullet\hspace{3pt} Model assessment*\\ 
			\textbullet\hspace{3pt} Model comparison*\\ 
			\textbullet\hspace{3pt} Multi-session experiments*\\ 
			\textbullet\hspace{3pt} Root cause analysis*\\ 
			\textbullet\hspace{3pt} Stakeholder engagement\\\hspace{10pt}and feedback*\\ 
			\textbullet\hspace{3pt} UI/UX studies \\ 
			\textbullet\hspace{3pt} User surveys*\\ 
			\textbullet\hspace{3pt} Validity testing/validation*\\ 
		}
		& \makecell[l]{
			\textbullet\hspace{3pt} Algorithmic impact assessments \\ 
			\textbullet\hspace{3pt} Assessing data quality*\\ 
			\textbullet\hspace{3pt} Calibration*\\ 
			\textbullet\hspace{3pt} Human content moderation\\  
			\textbullet\hspace{3pt} Data poisoning detection \\ 
			\textbullet\hspace{3pt} Field testing*\\ 
			\textbullet\hspace{3pt} Model assessment*\\ 
			\textbullet\hspace{3pt} Model comparison*\\ 
			\textbullet\hspace{3pt} Multi-session experiments*\\ 
			\textbullet\hspace{3pt} Perturbation studies*\\ 
			\textbullet\hspace{3pt} Root cause analysis*\\ 
			\textbullet\hspace{3pt} Screening for information integrity \\ 
			\textbullet\hspace{3pt} Sensitivity analysis*\\ 
			\textbullet\hspace{3pt} Stakeholder engagement and feedback*\\ 
			\textbullet\hspace{3pt} Statistical quality control*\\ 
			\textbullet\hspace{3pt} Supply chain auditing \\ 
			\textbullet\hspace{3pt} Testing third-party dependencies \\
			\textbullet\hspace{3pt} User surveys*\\ 
			\textbullet\hspace{3pt} Validity testing/validation.* \\
		} 
		& \makecell[l]{
			\textbullet\hspace{3pt} Algorithmic impact assessments \\ 
			\textbullet\hspace{3pt} Anomaly detection*\\ 
			\textbullet\hspace{3pt} Assessing data quality*\\ 
			\textbullet\hspace{3pt} Bias bounties \\ 
			\textbullet\hspace{3pt} Calibration*\\ 
			\textbullet\hspace{3pt} Chaos testing \\ 
			\textbullet\hspace{3pt} Cybersecurity testing \\ 
			\textbullet\hspace{3pt} Data poisoning detection \\ 
			\textbullet\hspace{3pt} Model assessment*\\ 
			\textbullet\hspace{3pt} Model comparison*\\ 
			\textbullet\hspace{3pt} Root cause analysis*\\ 
			\textbullet\hspace{3pt} Software testing \\ 
			\textbullet\hspace{3pt} Stakeholder engagement and feedback*\\ 
			\textbullet\hspace{3pt} Stress testing*\\ 
			\textbullet\hspace{3pt} Supply chain auditing \\ 
			\textbullet\hspace{3pt} Testing third-party dependencies \\
		}
		\\
		\hline
		\textbf{Manage} 
		& \makecell[l]{
			\textbullet\hspace{3pt} Fast decommission \\ 	
			\textbullet\hspace{3pt} Insurance \\ 		
			\textbullet\hspace{3pt} Supply chain audit \\ 	
			\textbullet\hspace{3pt} User recourse \\						 	 
		}
		& \makecell[l]{
			\textbullet\hspace{3pt} CSAM/Obscenity removal \\ 	
			\textbullet\hspace{3pt} Fast decommission \\ 		
			\textbullet\hspace{3pt} Intellectual property removal \\ 	
			\textbullet\hspace{3pt} Restrict minors \\ 	
			\textbullet\hspace{3pt} Restrict regulated dealings \\ 	
			\textbullet\hspace{3pt} Sensitive/Personal data removal \\ 	
			\textbullet\hspace{3pt} User recourse \\
		}
		& \makecell[l]{
			\textbullet\hspace{3pt} CSAM/Obscenity removal \\ 	
			\textbullet\hspace{3pt} Fast decommission \\ 	
			\textbullet\hspace{3pt} Insurance \\ 	
			\textbullet\hspace{3pt} Intellectual property removal \\ 	
			\textbullet\hspace{3pt} Restrict internet access \\ 	
			\textbullet\hspace{3pt} Restrict minors \\ 	
			\textbullet\hspace{3pt} Restrict regulated dealings \\ 	
			\textbullet\hspace{3pt} Sensitive/Personal data removal \\ 	
			\textbullet\hspace{3pt} Supply chain audit \\ 	
			\textbullet\hspace{3pt} User recourse \\ 						 	 		 	 
		}
		& \makecell[l]{
			\textbullet\hspace{3pt} CSAM/Obscenity removal \\ 	
			\textbullet\hspace{3pt} Fast decommission \\ 	
			\textbullet\hspace{3pt} Insurance \\ 	
			\textbullet\hspace{3pt} Intellectual property removal \\ 	
			\textbullet\hspace{3pt} Redundancy \\ 	
			\textbullet\hspace{3pt} Restrict internet access \\ 	
			\textbullet\hspace{3pt} Restrict minors \\ 	
			\textbullet\hspace{3pt} Restrict regulated dealings \\ 	
			\textbullet\hspace{3pt} Sensitive/Personal data removal \\ 	
			\textbullet\hspace{3pt} Supply chain audit \\ 	
			\textbullet\hspace{3pt} User recourse \\				 	 
		} 
		\\
		\hline
	\end{tabular}
	\label{table:high_risk_plan_by_gai_risk_cont}
\end{table}

\vfill
\raisebox{-10pt}{\makebox[\linewidth]{\thepage}}

\pagebreak
\thispagestyle{empty}

\begin{table}[H]
	\caption*{Table H.2: Example risk measurement and management approaches suitable for high-risk GAI applications organized by GAI Risk (continued).}
	\footnotesize
	\begin{tabular}{|c|c|c|c|c|}
		\hline
		\multirow{2}{*}{\textbf{Function}} & \multicolumn{4}{|c|}{\textbf{Generative AI Risk}}   \\
		\cline{2-5}
		& \textbf{Intellectual Property} & \makecell[l]{\textbf{Obscene, Degrading,}\\\textbf{and/or Abusive Content}} & \makecell[l]{\textbf{Toxicity, Bias, and}\\\textbf{Homogenization}} & \makecell[l]{\textbf{Value Chain and}\\\textbf{Component Integration}} \\
		\hline		
		\textbf{Measure} 
		& \makecell[l]{
			\textbullet\hspace{3pt} Algorithmic impact assessments \\ 
			\textbullet\hspace{3pt} Assessing data quality*\\ 
			\textbullet\hspace{3pt} Cybersecurity testing \\ 
			\textbullet\hspace{3pt} Field testing*\\ 
			\textbullet\hspace{3pt} Input/output measurement\\\hspace{10pt}using classifiers \\  
			\textbullet\hspace{3pt} Model comparison*\\ 
			\textbullet\hspace{3pt} Root cause analysis*\\ 
			\textbullet\hspace{3pt} Stakeholder engagement and feedback*\\ 
			\textbullet\hspace{3pt} Sub-sampling traffic for\\\hspace{10pt}manually annotating \\ 
			\textbullet\hspace{3pt} Supply chain auditing \\ 
			\textbullet\hspace{3pt} Testing third-party dependencies \\ 
			\textbullet\hspace{3pt} User surveys*\\ 
		}
		& \makecell[l]{
			\textbullet\hspace{3pt} Algorithmic impact assessments \\ 
			\textbullet\hspace{3pt} Assessing data quality*\\ 
			\textbullet\hspace{3pt} Calibration*\\ 
			\textbullet\hspace{3pt} Field testing*\\ 
			\textbullet\hspace{3pt} Input/output measurement\\\hspace{10pt}using classifiers \\ 
			\textbullet\hspace{3pt} Model assessment*\\ 
			\textbullet\hspace{3pt} Model comparison*\\ 
			\textbullet\hspace{3pt} Root cause analysis*\\ 
			\textbullet\hspace{3pt} Small user studies \\ 
			\textbullet\hspace{3pt} Software testing \\ 
			\textbullet\hspace{3pt} Stakeholder engagement and feedback*\\ 
			\textbullet\hspace{3pt} Statistical quality control*\\ 
			\textbullet\hspace{3pt} Stress testing*\\ 
			\textbullet\hspace{3pt} Supply chain auditing \\
			\textbullet\hspace{3pt} Testing third-party dependencies \\ 
			\textbullet\hspace{3pt} User surveys*\\ 
		}
		& \makecell[l]{
			\textbullet\hspace{3pt} Algorithmic impact assessments\\  	
			\textbullet\hspace{3pt} Analyze differences between \\\hspace{10pt}intended and actual population of \\\hspace{10pt}users or data subjects*\\ 	
			\textbullet\hspace{3pt} Anomaly detection*\\  	
			\textbullet\hspace{3pt} Assessing data quality*\\  	
			\textbullet\hspace{3pt} Bias bounties\\  	
			\textbullet\hspace{3pt} Bias testing\\  	
			\textbullet\hspace{3pt} Calibration*\\  	
			\textbullet\hspace{3pt} Counterfactual/causal analysis\\  	
			\textbullet\hspace{3pt} Disaggregated metrics\\  	
			\textbullet\hspace{3pt} Field testing*\\  	
			\textbullet\hspace{3pt} Model assessment*\\  	
			\textbullet\hspace{3pt} Model comparison*\\  	
			\textbullet\hspace{3pt} Multi-session experiments*\\  	
			\textbullet\hspace{3pt} Root cause analysis*\\  	
			\textbullet\hspace{3pt} Software testing\\  	
			\textbullet\hspace{3pt} Statistical quality control*\\  	
			\textbullet\hspace{3pt} Stress testing*\\  		
			\textbullet\hspace{3pt} User surveys*\\  		
			\textbullet\hspace{3pt} Validity testing/validation.*
		}
		& \makecell[l]{
			\textbullet\hspace{3pt} Assessing data quality*\\ 
			\textbullet\hspace{3pt} Model assessment*\\ 
			\textbullet\hspace{3pt} Model comparison*\\ 
			\textbullet\hspace{3pt} Software testing \\ 
			\textbullet\hspace{3pt} Supply chain auditing \\ 
			\textbullet\hspace{3pt} Testing third-party dependencies \\
		}
		\\
		\hline
		\textbf{Manage} 
		& \makecell[l]{
			\textbullet\hspace{3pt} Fast decommission \\ 	
			\textbullet\hspace{3pt} Insurance \\ 	
			\textbullet\hspace{3pt} Intellectual property removal \\ 	
			\textbullet\hspace{3pt} Restrict internet access \\ 		
			\textbullet\hspace{3pt} Supply chain audit \\ 	
			\textbullet\hspace{3pt} User recourse \\ 					 	 
		} 
		& \makecell[l]{
			\textbullet\hspace{3pt} CSAM/Obscenity removal \\ 	
			\textbullet\hspace{3pt} Fast decommission \\ 	
			\textbullet\hspace{3pt} Insurance \\ 	
			\textbullet\hspace{3pt} Restrict internet access \\ 	
			\textbullet\hspace{3pt} Restrict minors \\ 	
			\textbullet\hspace{3pt} Restrict regulated dealings \\ 	
			\textbullet\hspace{3pt} Sensitive/Personal data removal \\ 	
			\textbullet\hspace{3pt} Supply chain audit \\ 	
			\textbullet\hspace{3pt} User recourse \\						 	 
		} 
		& \makecell[l]{
			\textbullet\hspace{3pt} CSAM/Obscenity removal \\ 	
			\textbullet\hspace{3pt} Fast decommission \\ 	
			\textbullet\hspace{3pt} Insurance \\ 	
			\textbullet\hspace{3pt} Intellectual property removal \\ 	
			\textbullet\hspace{3pt} Restrict regulated dealings \\ 	
			\textbullet\hspace{3pt} Sensitive/Personal data removal \\ 	
			\textbullet\hspace{3pt} Supply chain audit \\ 	
			\textbullet\hspace{3pt} User recourse \\  
		}
		& \makecell[l]{
			\textbullet\hspace{3pt} CSAM/Obscenity removal \\ 		
			\textbullet\hspace{3pt} Intellectual property removal \\ 	
			\textbullet\hspace{3pt} Redundancy \\ 		
			\textbullet\hspace{3pt} Sensitive/Personal data removal \\ 	
			\textbullet\hspace{3pt} Supply chain audit \\ 							 	 
		} 
		\\
		\hline
	\end{tabular}
	\label{table:high_risk_plan_by_gai_risk_cont2}
\end{table}

\noindent\textbf{Usage Note}: Appendix H puts forward an example risk measurement and management plan for high risk GAI systems or applications. The high risk plan focuses on field testing and applies extensive risk controls. Measurement and management approaches from Appendices F and G should also be applied to high risk systems or applications.

\begin{itemize}
	\item Material in Table H.1 can be applied to measure and manage GAI risks in risk programs that are aligned to the trustworthy characteristics. 
	\item Material in Table H.2 can be applied to measure and manage GAI risks in risk programs that are aligned to GAI risks. 
\end{itemize}
 
\vfill
\raisebox{-10pt}{\makebox[\linewidth]{\thepage}}

\end{landscape}

\end{document}