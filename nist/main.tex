% Copyright Patrick Hall 20XX

\documentclass[fleqn]{article}
\renewcommand\refname{}
\title{Title:\\\vspace{5pt}\normalsize{Subtitle}}
\author{\copyright Patrick Hall 20XX}

\usepackage{graphicx}
\usepackage{fullpage}
\usepackage{pdfpages}
\usepackage{amsmath}
\usepackage{amssymb}
\usepackage{mathtools}
\usepackage{MnSymbol}
\usepackage{enumerate}
\usepackage{setspace}
\usepackage[hyphens]{url}
\usepackage[colorlinks]{hyperref}
\usepackage{float}
\usepackage{caption}
\usepackage{subcaption}
\usepackage{multicol}
\usepackage{color}
\usepackage{listings}
\usepackage{csvsimple}
\usepackage{algorithm}
\usepackage{algorithmic}
\usepackage{verbatim}
\usepackage{mdframed}
\usepackage{changepage}
\usepackage[top=1in, bottom=1in, left=1in, right=1in]{geometry}

\usepackage{booktabs}
\usepackage{pdflscape}
\usepackage{makecell}
\usepackage{multirow}

\begin{document}

\maketitle

\begin{abstract}
	
\end{abstract}

% ---------- ----------
\section{Introduction} \label{sec:intro}
% ---------- ----------

The National Institute of Standards and Technology Artificial Intelligence (AI) Risk Management Framework (RMF).\cite{airmf}

% ---------- ----------
\section{Generative AI Governance}\label{sec:govern}
% ---------- ----------

% What's different?
% - Supply chain/third party 
% - Complex human AI interaction

% ---------- ----------
\section{Generative AI Inventories}\label{sec:inv}
% ---------- ----------

% ---------- ----------
\section{Generative AI Risk Tiers}\label{sec:tiers}
% ---------- ----------

% ---------- ----------
\section{Generative AI Risk Measurement}\label{sec:measure}
% ---------- ----------

% ---------- ----------
\section{Generative AI Risk Management}\label{sec:manage}
% ---------- ----------

% ---------- ----------
\section*{Conclusion}
% ---------- ----------

% ---------- ----------
\section*{Acknowledgments}
% ---------- ----------

Thank you to Bernie Siskin and Nick Schmidt of BLDS and Eric Sublett of Relman Colfax for formative discussions relating to GAI risk tiering. 

% ---------- ----------
\section*{Abbreviations}
% ---------- ----------

\begin{itemize}
	\item AI: Artificial Intelligence
	\item AI RMF: Artificial Intelligence Risk Management Framework
	\item GAI: Generative AI
	\item RMF: Risk Management Framework
\end{itemize}

% ---------- ----------
\bibliographystyle{plain}
\bibliography{bibliography}
% ---------- ----------

\begin{landscape}
\thispagestyle{empty}	
% ---------- ----------
\section*{Appendix A: Example Generative AI--Trustworthy Characteristic Crosswalk}\label{sec:appndxa}
% ---------- ----------

% Cross enables applying guidance fro AI RMF to risks 
% Other interpretations or crosswalks are possible

% ---------- ----------
\subsection{A.1: Trustworthy Characteristic to Generative AI Risk Crosswalk}\label{sec:appndxa1}
% ---------- ----------

\begin{table}[H]
	\caption{Trustworthy Characteristic to Generative AI Risk Crosswalk.}
	\label{tab:tc_to_gai_risk_cw}
	\footnotesize
	\begin{tabular}{llll}
		\toprule
		Accountable and Transparent & Explainable and Interpretable & Fair with Harmful Bias Managed & Privacy Enhanced \\
		\midrule
		Data Privacy & Human-AI Configuration & Confabulation & Data Privacy \\
		Environmental & Value Chain and Component Integration & Environmental & Human-AI Configuration \\
		Human-AI Configuration &  & Human-AI Configuration & Information Security \\
		Information Integrity &  & Intellectual Property & Intellectual Property \\
		Intellectual Property &  & Obscene, Degrading, and/or Abusive Content & Value Chain and Component Integration \\
		Value Chain and Component Integration &  & Toxicity, Bias, and Homogenization &  \\
 		&  & Value Chain and Component Integration &  \\
 		&  &  &  \\
 		&  &  &  \\
 		&  &  &  \\
		\bottomrule
	\end{tabular}
	\newline
	\vspace{10pt}
	\newline
	\begin{tabular}{lll}
		\toprule
		Safe & Secure and Resilient & Valid and Reliable \\
		\midrule
		CBRN Information & Dangerous or Violent Recommendations & Confabulation \\
		Confabulation & Data Privacy & Human-AI Configuration \\
		Dangerous or Violent Recommendations & Human-AI Configuration & Information Integrity \\
		Data Privacy & Information Security & Information Security \\
		Environmental & Value Chain and Component Integration & Toxicity, Bias, and Homogenization \\
		Human-AI Configuration &  & Value Chain and Component Integration \\
		Information Integrity &  &  \\
		Information Security &  &  \\
		Obscene, Degrading, and/or Abusive Content &  &  \\
		Value Chain and Component Integration &  &  \\
		\bottomrule
	\end{tabular}
\end{table}
\vfill
\raisebox{-10pt}{\makebox[\linewidth]{\thepage}}
\end{landscape}

\begin{landscape}
\thispagestyle{empty}	
% ---------- ----------
\subsection{A.2: Generative AI Risk to Trustworthy Characteristic Crosswalk}\label{sec:appndxa2}
% ---------- ----------

\begin{table}[H]
	\caption{Generative AI Risk to Trustworthy Characteristic Crosswalk.}
	\label{tab:gai_risk_to_tc_cw}
	\small
	\begin{tabular}{llll}
		\toprule
		CBRN Information & Confabulation & Dangerous or Violent Recommendations & Data Privacy \\
		\midrule
		Safe & Fair with Harmful Bias Managed & Safe & Accountable and Transparent \\
 		& Safe & Secure and Resilient & Privacy Enhanced \\
 		& Valid and Reliable &  & Safe \\
 		&  &  & Secure and Resilient \\
	\bottomrule
	\end{tabular}
	\newline
	\vspace{10pt}
	\newline	
	\begin{tabular}{llll}
		\toprule
		Environmental & Human-AI Configuration & Information Integrity & Information Security \\
		\midrule
		Accountable and Transparent & Accountable and Transparent & Accountable and Transparent & Privacy Enhanced \\
		Fair with Harmful Bias Managed & Explainable and Interpretable & Safe & Safe \\
		Safe & Fair with Harmful Bias Managed & Valid and Reliable & Secure and Resilient \\
 		& Privacy Enhanced &  & Valid and Reliable \\
 		& Safe &  &  \\
 		& Secure and Resilient &  &  \\
 		& Valid and Reliable &  &  \\
		\bottomrule
	\end{tabular}
	\newline
	\vspace{10pt}
	\newline
	\begin{tabular}{llll}
		\toprule
		Intellectual Property & Obscene, Degrading, and/or Abusive Content & Toxicity, Bias, and Homogenization & Value Chain and Component Integration \\
		\midrule
		Accountable and Transparent & Fair with Harmful Bias Managed & Fair with Harmful Bias Managed & Accountable and Transparent \\
		Fair with Harmful Bias Managed & Safe & Valid and Reliable & Explainable and Interpretable \\
		Privacy Enhanced &  &  & Fair with Harmful Bias Managed \\
 		&  &  & Privacy Enhanced \\
 		&  &  & Safe \\
 		&  &  & Secure and Resilient \\
 		&  &  & Valid and Reliable \\
		\bottomrule
	\end{tabular}
\end{table}
\vfill
\raisebox{-10pt}{\makebox[\linewidth]{\thepage}}
\end{landscape}

% ---------- ----------
\section*{Appendix B: Example Risk Tiers for Generative AI}\label{sec:appndxb}
% ---------- ----------

\subsection{IEEE 1012 Example Impact Descriptions}

\begin{table}[H]
    \caption{Example Impact Levels from IEEE 1012 \cite{ieee1012} Annex B, Table B.2.}
    \small
        \begin{tabular}{|m{0.20\linewidth} | m{0.75\linewidth}|}
            \hline
            \textbf{Level} & \textbf{Description} \\ \hline
            Catastrophic & Loss of human life, complete mission failure, loss of system security and safety, or extensive financial or social loss. \\
            \hline
            Critical & Major and permanent injury, partial loss of mission, major system damage, or major financial or social loss. \\
            \hline
            Marginal & Severe injury or illness, degradation of secondary mission, or some financial or social loss. \\
            \hline
            Neglible & Minor injury or illness, minor impact on system performance, or operator inconvenience. \\
            \hline
    \end{tabular}
    \label{table:ieee_impacts}
\end{table}

\subsection{NIST 800-30r1 Example Impact Descriptions}

\begin{table}[H]
	\caption{Example Impact Levels from NIST SP800-30r1 \cite{nist80030r1} Appendix H, Table H-3.}
	\small
    \begin{tabular}{|m{0.20\linewidth} |m{0.20\linewidth} | m{0.05\linewidth} | m{0.40\linewidth}|}
        \hline
        \textbf{Qualitative Values} & \multicolumn{2}{c|}{\textbf{Semi-Quantitative Values}} & \textbf{Description} \\
        \hline
        Very High & 96-100 & 10 & The event could be expected to have multiple severe or catastrophic adverse effects on organizational operations, organizational assets, individuals, other organizations, or the Nation. \\
        \hline
        High & 80-95 & 8 & The event could be expected to have a severe or catastrophic adverse effect on organizational operations, organizational assets, individuals, other organizations, or the Nation. A severe or catastrophic adverse effect means that, for example, the threat event might: (i) cause a severe degradation in or loss of mission capability to an extent and duration that the organization is not able to perform one or more of its primary functions; (ii) result in major damage to organizational assets; (iii) result in major financial loss; or (iv) result in severe or catastrophic harm to individuals involving loss of life or serious life-threatening injuries. \\
        \hline		
        Moderate & 21-79 & 5 & The event could be expected to have a serious adverse effect on organizational operations, organizational assets, individuals other organizations, or the Nation. A serious adverse effect means that, for example, the threat event might: (i) cause a significant degradation in mission capability to an extent and duration that the organization is able to perform its primary functions, but the effectiveness of the functions is significantly reduced; (ii) result in significant damage to organizational assets; (iii) result in significant financial loss; or (iv) result in significant harm to individuals that does not involve loss of life or serious life-threatening injuries. \\
        \hline
        Low & 5-20 & 2 & The event could be expected to have a limited adverse effect on organizational operations, organizational assets, individuals other organizations, or the Nation. A limited adverse effect means that, for example, the threat event might: (i) cause a degradation in mission capability to an extent and duration that the organization is able to perform its primary functions, but the effectiveness of the functions is noticeably reduced; (ii) result in minor damage to organizational assets; (iii) result in minor financial loss; or (iv) result in minor harm to individuals. \\
        \hline
        Very Low & 0-4 & 0 & The threat event could be expected to have a negligible adverse effect on organizational operations, organizational assets, individuals other organizations, or the Nation. \\
        \hline
    \end{tabular}
	
    \label{table:nist_impacts}
\end{table}

\subsection{NIST 800-30r1 Example Likelihood Descriptions}

\begin{table}[H]
	\caption{Example Likelihood Levels from NIST SP800-30r1 \cite{nist80030r1} Appendix G, Table G-3.}
    \begin{tabular}{|m{0.20\linewidth} |m{0.20\linewidth} | m{0.05\linewidth} | m{0.40\linewidth}|}
        \hline
        \textbf{Qualitative Values} & \multicolumn{2}{c|}{\textbf{Semi-Quantitative Values}} & \textbf{Description} \\
        \hline
        Very High & 96-100 & 10 & Error, accident, or act of nature is almost certain to occur; or occurs more than 100 times a year. \\
        \hline
        High & 80-95 & 8 & Error, accident, or act of nature is highly likely to occur; or occurs between 10-100 times a year.\\
        \hline		
        Moderate & 21-79 & 5 & Error, accident, or act of nature is somewhat likely to occur; or occurs between 1-10 times a year.\\
        \hline
        Low & 5-20 & 2 & Error, accident, or act of nature is unlikely to occur; or occurs less than once a year, but more than once every 10 years.  \\
        \hline
        Very Low & 0-4 & 0 & Error, accident, or act of nature is highly unlikely to occur; or occurs less than once every 10 years. \\
        \hline
    \end{tabular}
    \label{table:nist_likelihood}
\end{table}

\subsection{NIST 800-30r1 Example Risk Tiers}

\begin{table}[H]
    \caption{Example Risk Assessment Matrix with 5 Impact Levels, 5 Likelihood Levels, and 5 Risk Tiers from NIST SP800-30r1 \cite{nist80030r1} Appendix I, Table I-2.}
    \small
    \begin{tabular}{|c|l|l|l|l|l|l|}
        \hline
        \multirow{2}{*}{\textbf{Likelihood}} & \multicolumn{5}{|c|}{\textbf{Level of Impact}}   \\
        \cline{2-6}
        & \textbf{Very Low} & \textbf{Low} & \textbf{Moderate} & \textbf{High} &\textbf{Very High} \\
        \hline
        \textbf{Very High} & Tier 5 & Tier 4 & Tier 3 & Tier 2 & Tier 1 \\
        \hline		
        \textbf{High} & Tier 5 & Tier 4 & Tier 3  & Tier 2 & Tier 1 \\
        \hline
        \textbf{Moderate} & Tier 5 & Tier 4 & Tier 4  & Tier 3 & Tier 2 \\
        \hline
        \textbf{Low} & Tier 5 & Tier 4 & Tier 4  & Tier 4 & Tier 3 \\
        \hline
        \textbf{Very Low} & Tier 5 & Tier 5 & Tier 5  & Tier 4 & Tier 4 \\
        \hline
    \end{tabular}
    \label{table:nist_risk_tiers}
\end{table}

\subsection{NIST 800-30r1 Example Risk Descriptions}

\begin{table}[H]
	\caption{Example Risk Descriptions from NIST SP800-30r1 \cite{nist80030r1} Appendix I, Table I-3.}
    \begin{tabular}{|m{0.20\linewidth} |m{0.20\linewidth} | m{0.05\linewidth} | m{0.40\linewidth}|}
        \hline
        \textbf{Qualitative Values} & \multicolumn{2}{c|}{\textbf{Semi-Quantitative Values}} & \textbf{Description} \\
        \hline
        Very High & 96-100 & 10 & Very high risk means that an event could be expected to have multiple severe or catastrophic adverse effects on organizational operations, organizational assets, individuals, other organizations, or the Nation.\\
        \hline
        High & 80-95 & 8 & High risk means that an event could be expected to have a severe or catastrophic adverse effect on organizational operations, organizational assets, individuals, other organizations, or the Nation. \\
        \hline		
        Moderate & 21-79 & 5 & Moderate risk means that an event could be expected to have a serious adverse effect on organizational operations, organizational assets, individuals, other organizations, or the Nation.\\
        \hline
        Low & 5-20 & 2 & Low risk means that an event could be expected to have a limited adverse effect on organizational operations, organizational assets, individuals, other organizations, or the Nation. \\
        \hline
        Very Low & 0-4 & 0 & Very low risk means that an event could be expected to have a negligible adverse effect on organizational operations, organizational assets, individuals, other organizations, or the Nation. \\
        \hline
    \end{tabular}
    \label{table:nist_likelihood}
\end{table}

\subsection{Practical Risk-tiering Questions}

\begin{itemize}

\item \textbf{Confabulation}: What happens if it’s wrong?
\item \textbf{Dangerous and Violent Recommendations}: Can it possibly give dangerous or violent recommendations?
\item \textbf{Data Privacy}: What happens is someone enters sensitive data into the system?
\item \textbf{Human-AI Configuration}: What happens if someone uses it wrong? Is it used for decision-making?
\item \textbf{Information Integrity}: Will it pump out large-scale disinformation, even internally? Will output be used as input? Will output be tagged as generated by AI?
\item \textbf{Information Security}: What happens if someone steals the training data? What happens is someone steals the model? Who has access to training data? Are standard security controls applied? Are all dependencies audited? Are supply chains understood? Can it be used to impersonate bank personnel?
\item \textbf{Intellectual Property}: What happens if outputs contain other entities IP?
\item \textbf{Toxicity, Bias, and Homogenization}: What happens if outputs are biased, toxic or obscene? Will output be used as input? Is the application accessible?
\item \textbf{Value Chain and Component Integration}: Are contracts reviewed for legal risks? Standard acquisition/procurement controls applied? Do vendors provide incident response? With guaranteed response times? Other critical support?

\end{itemize}

% ---------- ----------
\section*{Appendix C: List of Publicly Available Model Testing Suites (``Evals'')}\label{sec:appndxc}
% ---------- ----------

% https://mlcommons.org/benchmarks/ai-safety/
% https://github.com/IMDA-BTG/LLM-Evals-Catalogue
% https://github.com/HongshengHu/membership-inference-machine-learning-literature
% https://github.com/chawins/llm-sp

% evals cant test some things -- explainability, HAI-config, supply chain

% ---------- ----------
\subsection*{C.1: Publicly Available Model Testing Suites (``Evals'') by Trustworthy Characteristic}\label{appndxc1}
% ---------- ----------

\begin{table}[H]
	\caption{Publicly Available Model Testing Suites (``Evals'') by Trustworthy Characteristic.}
	\label{tab:low_risk_measure_by_tc}
	\footnotesize
	\begin{tabular}{l}
		\toprule
		Accountable and Transparent \\
		\midrule
			\makecell[l]{An Evaluation on Large Language Model Outputs: \\\hspace{10pt}Discourse and Memorization (see Appendix B)\cite{de2023evaluation}} \\
			Big-bench: Truthfulness \cite{bigbench} \\
			DecodingTrust: Machine Ethics \cite{decodingtrust}\\
			Evaluation Harness: ETHICS \cite{evalharness}\\
			HELM: Copyright \cite{helm} \\
			Mark My Words \cite{markmywords} \\
		\bottomrule
	\end{tabular}
	\newline
	\vspace{10pt}
	\newline
	\begin{tabular}{l}
		\toprule
		Fair with Harmful Bias Managed \\
		\midrule
		BELEBELE \cite{belebele} \\
		Big-bench: Low-resource language, Non-English, Translation  \\
		Big-bench: Social bias, Racial bias, Gender bias, Religious bias \\
		Big-bench: Toxicity \\
		DecodingTrust: Fairness \\
		DecodingTrust: Stereotype Bias \\
		DecodingTrust: Toxicity \\
		C-Eval (Chinese evaluation suite) \cite{ceval}\\
		Evaluation Harness: CrowS-Pairs  \\
		Evaluation Harness: ToxiGen \\
		Finding New Biases in Language Models with a Holistic Descriptor Dataset \cite{smith2022m}\\
		\makecell[l]{From Pretraining Data to Language Models to Downstream Tasks:\\\hspace{10pt}Tracking the Trails of Political Biases Leading to Unfair NLP Models \cite{feng2023pretraining}} \\
		HELM: Bias \\
		HELM: Toxicity \\
		MT-bench \cite{mtbench} \\
		The Self-Perception and Political Biases of ChatGPT \cite{rutinowski2023self}\\
		\makecell[l]{Towards Measuring the Representation of\\\hspace{10pt} Subjective Global Opinions in Language Models \cite{durmus2023towards}}\\
		\bottomrule
	\end{tabular}
	\newline
	\vspace{10pt}
	\newline
	\begin{tabular}{l}
		\toprule
		Privacy Enhanced \\
		\midrule
		HELM: Copyright\\
		llmprivacy \cite{llmprivacy}\\
		mimir \cite{mimir}\\
	\bottomrule
	\end{tabular}
	\newline
	\vspace{10pt}
	\newline	
	\begin{tabular}{l}
		\toprule
		Safe \\
		\midrule
			Big-bench: Convince Me \\
			Big-bench: Truthfulness \\
			HELM: Reiteration, Wedging \\
			Mark My Words \\
			MLCommons \cite{mlcommons} \\
			The WMDP Benchmark \cite{wmdp} \\
		\bottomrule
	\end{tabular}	
\end{table}	

\pagebreak 

\begin{table}[H]
	\caption*{Publicly Available Model Testing Suites (``Evals'') by Trustworthy Characteristic (continued).}
	\label{tab:low_risk_measure_by_tc_cont}
	\footnotesize
	\begin{tabular}{l}
		\toprule
		Secure and Resilient \\
		\midrule
			Catastrophic Jailbreak of Open-source LLMs via Exploiting Generation \cite{huang2023catastrophic} \\
			\makecell[l]{DecodingTrust: Adversarial Robustness,\\\hspace{10pt} Robustness Against Adversarial Demonstrations} \\
			detect-pretrain-code \cite{detectpretraincode} \\
			In-The-Wild Jailbreak Prompts on LLMs \cite{shen2023anything}\\
			JailbreakingLLMs \cite{chao2023jailbreaking}\\
			llmprivacy \\
			mimir \\
			TAP: A Query-Efficient Method for Jailbreaking Black-Box LLMs \cite{mehrotra2023tree}\\
		\bottomrule
	\end{tabular}
	\newline
	\vspace{10pt}
	\newline		
	\begin{tabular}{l}
		\toprule
		Valid and Reliable \\
		\midrule
			\makecell[l]{Big-bench: Algorithms, Logical reasoning, Implicit reasoning, Mathematics, Arithmetic, Algebra, Mathematical proof,\\\hspace{10pt} Fallacy, Negation, Computer code, Probabilistic reasoning, Social reasoning, Analogical reasoning, Multi-step,\\\hspace{10pt} Understanding the World} \\
		Big-bench: Analytic entailment, Formal fallacies and syllogisms with negation, Entailed polarity \\
		Big-bench: Context Free Question Answering \\
		Big-bench: Contextual question answering, Reading comprehension, Question generation \\
		Big-bench: Morphology, Grammar, Syntax \\
		Big-bench: Out-of-Distribution \\
		Big-bench: Paraphrase \\
		Big-bench: Sufficient information \\
		Big-bench: Summarization \\
		DecodingTrust: Out-of-Distribution Robustness, Adversarial Robustness, 			Robustness Against Adversarial Demonstrations\\
		Eval Gauntlet: Reading comprehension \cite{evalgauntlet} \\
		Eval Gauntlet: Commonsense reasoning, Symbolic problem solving, Programming \\
		Eval Gauntlet: Language Understanding  \\
		Eval Gauntlet: World Knowledge \\
		Evaluation Harness: BLiMP \\
		Evaluation Harness: CoQA, ARC \\
		Evaluation Harness: GLUE \\
		Evaluation Harness: HellaSwag, OpenBookQA, TruthfulQA \\
		Evaluation Harness: MuTual \\
		Evaluation Harness: PIQA, PROST, MC-TACO, MathQA, LogiQA, DROP \\
		FLASK: Logical correctness, Logical robustness, Logical efficiency, Comprehension, Completeness \cite{flask}  \\
		FLASK: Readability, Conciseness, Insightfulness \\
		HELM: Knowledge \\
		HELM: Language \\
		HELM: Text classification \\
		HELM: Question answering \\
		HELM: Reasoning \\
		HELM: Robustness to contrast sets \\
		HELM: Summarization \\
		Hugging Face: Fill-mask, Text generation \cite{huggingface} \\
		Hugging Face: Question answering \\
		Hugging Face: Summarization \\
		Hugging Face: Text classification, Token classification, Zero-shot classification \\
		MASSIVE \cite{massive} \\
		MT-bench \\
		\bottomrule
	\end{tabular}	
\end{table}	

% ---------- ----------
\subsection*{C.2: Publicly Available Model Testing Suites (``Evals'') by Generative AI Risk}\label{appndxc2}
% ---------- ----------
\begin{table}[H]
	\caption{Publicly Available Model Testing Suites (``Evals'') by Generative AI Risk.}
	\label{tab:low_risk_measure_by_gai_risk}
	\footnotesize
	\begin{tabular}{l}
		\toprule
		CBRN Information \\
		\midrule
			Big-bench: Convince Me \\
			Big-bench: Truthfulness \\
			HELM: Reiteration, Wedging \\
			MLCommons \\
			The WMDP Benchmark \\
		\bottomrule
	\end{tabular}
	\newline
	\vspace{10pt}
	\newline
	\begin{tabular}{l}
		\toprule
		Confabulation \\
		\midrule
		BELEBELE \\
		\makecell[l]{Big-bench: Algorithms, Logical reasoning, Implicit reasoning, Mathematics, Arithmetic, Algebra,\\\hspace{10pt} Mathematical proof, Fallacy, Negation, Computer code, Probabilistic reasoning, Social reasoning,\\\hspace{10pt}  Analogical reasoning, Multi-step, Understanding the World} \\
		Big-bench: Analytic entailment, Formal fallacies and syllogisms with negation, Entailed polarity \\
		Big-bench: Context Free Question Answering \\
		Big-bench: Contextual question answering, Reading comprehension, Question generation \\
		Big-bench: Convince Me \\
		Big-bench: Low-resource language, Non-English, Translation  \\
		Big-bench: Morphology, Grammar, Syntax \\
		Big-bench: Out-of-Distribution \\
		Big-bench: Paraphrase \\
		Big-bench: Sufficient information \\
		Big-bench: Summarization \\
		Big-bench: Truthfulness \\
		C-Eval (Chinese evaluation suite) \\
		\makecell[l]{DecodingTrust: Out-of-Distribution Robustness, Adversarial Robustness,\\\hspace{10pt} Robustness Against Adversarial Demonstrations} \\
		Eval Gauntlet
		Reading comprehension \\
		Eval Gauntlet: Commonsense reasoning, Symbolic problem solving, Programming \\
		Eval Gauntlet: Language Understanding  \\
		Eval Gauntlet: World Knowledge \\
		Evaluation Harness: BLiMP \\
		Evaluation Harness: CoQA, ARC \\
		Evaluation Harness: GLUE \\
		Evaluation Harness: HellaSwag, OpenBookQA, TruthfulQA \\
		Evaluation Harness: MuTual \\
		Evaluation Harness: PIQA, PROST, MC-TACO, MathQA, LogiQA, DROP \\
		FLASK: Logical correctness, Logical robustness, Logical efficiency, Comprehension, Completeness \\
		FLASK: Readability, Conciseness, Insightfulness \\
		Finding New Biases in Language Models with a Holistic Descriptor Dataset \\
		HELM: Knowledge \\
		HELM: Language \\
		HELM: Language (Twitter AAE) \\
		HELM: Question answering \\
		HELM: Reasoning \\
		HELM: Reiteration, Wedging \\
		HELM: Robustness to contrast sets \\
		HELM: Summarization \\
		HELM: Text classification \\
		Hugging Face: Fill-mask, Text generation \\
		Hugging Face: Question answering \\
		Hugging Face: Summarization \\
		Hugging Face: Text classification, Token classification, Zero-shot classification \\
		MASSIVE \\
		MLCommons \\
		MT-bench \\
		\bottomrule
	\end{tabular}
\end{table}		
		
\pagebreak

\begin{table}[H]
	\caption*{Publicly Available Model Testing Suites (``Evals'') by Generative AI Risk (continued).}
	\label{tab:low_risk_measure_by_gai_risk_cont1}
	\footnotesize
	\begin{tabular}{l}	
		\toprule
		Dangerous or Violent Recommendations \\
		\midrule	
		Big-bench: Convince Me \\
		Big-bench: Toxicity \\		
		DecodingTrust: Adversarial Robustness, Robustness Against Adversarial Demonstrations \\
		DecodingTrust: Machine Ethics \\
		DecodingTrust: Toxicity \\
		Evaluation Harness: ToxiGen \\
		HELM: Reiteration, Wedging \\
		HELM: Toxicity \\			
		MLCommons \\
		\bottomrule
	\end{tabular}
	\newline
	\vspace{10pt}
	\newline	
	\begin{tabular}{l}	
		\toprule
		Data Privacy \\
		\midrule
		An Evaluation on Large Language Model Outputs: Discourse and Memorization (with human scoring, see Appendix B) \\
		Catastrophic Jailbreak of Open-source LLMs via Exploiting Generation \\
		DecodingTrust: Machine Ethics \\
		Evaluation Harness: ETHICS \\
		HELM: Copyright \\
		In-The-Wild Jailbreak Prompts on LLMs \\
		JailbreakingLLMs \\
		MLCommons \\
		Mark My Words \\
		TAP: A Query-Efficient Method for Jailbreaking Black-Box LLMs \\
		detect-pretrain-code \\
		llmprivacy \\
		mimir \\	
		\bottomrule
	\end{tabular}
	\newline
	\vspace{10pt}
	\newline	
	\begin{tabular}{l}	
		\toprule	
		Environmental \\
		\midrule	
		HELM: Efficiency \\
		\bottomrule
	\end{tabular}
	\newline
	\vspace{10pt}
	\newline	
	\begin{tabular}{l}	
		\toprule	
		Information Integrity \\
		\midrule
		Big-bench: Analytic entailment, Formal fallacies and syllogisms with negation, Entailed polarity \\
		Big-bench: Convince Me \\
		Big-bench: Paraphrase \\
		Big-bench: Sufficient information \\
		Big-bench: Summarization \\
		Big-bench: Truthfulness \\
		DecodingTrust: Machine Ethics \\
		DecodingTrust: Out-of-Distribution Robustness, Adversarial Robustness, Robustness Against Adversarial Demonstrations \\
		Eval Gauntlet: Language Understanding  \\
		Eval Gauntlet: World Knowledge \\
		Evaluation Harness: CoQA, ARC \\
		Evaluation Harness: ETHICS \\
		Evaluation Harness: GLUE \\
		Evaluation Harness: HellaSwag, OpenBookQA, TruthfulQA \\
		Evaluation Harness: MuTual \\
		Evaluation Harness: PIQA, PROST, MC-TACO, MathQA, LogiQA, DROP \\
		FLASK: Logical correctness, Logical robustness, Logical efficiency, Comprehension, Completeness \\
		FLASK: Readability, Conciseness, Insightfulness \\
		HELM: Knowledge \\
		HELM: Language \\
		HELM: Question answering \\
		HELM: Reasoning \\
		HELM: Reiteration, Wedging \\
		HELM: Robustness to contrast sets \\
		HELM: Summarization \\
		HELM: Text classification \\
		Hugging Face: Fill-mask, Text generation \\
		Hugging Face: Question answering \\
		Hugging Face: Summarization \\
		MLCommons \\
		MT-bench \\
		Mark My Words \\
		\bottomrule
	\end{tabular}
\end{table}	

\pagebreak

\begin{table}[H]
	\caption*{Publicly Available Model Testing Suites (``Evals'') by Generative AI Risk (continued).}
	\label{tab:low_risk_measure_by_gai_risk_cont2}
	\footnotesize
	\begin{tabular}{l}
		\toprule
		Information Security \\
		\midrule
		Big-bench: Convince Me \\
		Big-bench: Out-of-Distribution \\
		Catastrophic Jailbreak of Open-source LLMs via Exploiting Generation \\
		DecodingTrust: Out-of-Distribution Robustness, Adversarial Robustness, Robustness Against Adversarial Demonstrations \\
		Eval Gauntlet: Commonsense reasoning, Symbolic problem solving, Programming \\
		HELM: Copyright \\
		In-The-Wild Jailbreak Prompts on LLMs \\
		JailbreakingLLMs \\
		Mark My Words \\
		TAP: A Query-Efficient Method for Jailbreaking Black-Box LLMs \\
		detect-pretrain-code \\
		llmprivacy \\
		mimir \\
		\bottomrule
	\end{tabular}
	\newline
	\vspace{10pt}
	\newline	
	\begin{tabular}{l}	
		\toprule	
		Intellectual Property \\
		\midrule	
		An Evaluation on Large Language Model Outputs: Discourse and Memorization (with human scoring, see Appendix B) \\
		HELM: Copyright \\
		Mark My Words \\
		llmprivacy \\
		mimir \\	
		\bottomrule
	\end{tabular}
	\newline
	\vspace{10pt}
	\newline	
	\begin{tabular}{l}	
		\toprule
		Obscene, Degrading, and/or Abusive Content \\
		\midrule
		Big-bench: Social bias, Racial bias, Gender bias, Religious bias \\
		Big-bench: Toxicity \\
		DecodingTrust: Fairness \\
		DecodingTrust: Stereotype Bias \\
		DecodingTrust: Toxicity \\
		Evaluation Harness: CrowS-Pairs  \\
		Evaluation Harness: ToxiGen \\
		HELM: Bias \\
		HELM: Toxicity \\	
		\bottomrule	
	\end{tabular}
	\newline
	\vspace{10pt}
	\newline	
	\begin{tabular}{l}	
		\toprule
		Toxicity, Bias, and Homogenization \\
		\midrule
		BELEBELE \\
		Big-bench: Low-resource language, Non-English, Translation  \\
		Big-bench: Out-of-Distribution \\
		Big-bench: Social bias, Racial bias, Gender bias, Religious bias \\
		Big-bench: Toxicity \\
		C-Eval (Chinese evaluation suite) \\
		DecodingTrust: Fairness \\
		DecodingTrust: Stereotype Bias \\
		DecodingTrust: Toxicity \\
		Eval Gauntlet: World Knowledge \\
		Evaluation Harness: CrowS-Pairs  \\
		Evaluation Harness: ToxiGen \\
		Finding New Biases in Language Models with a Holistic Descriptor Dataset \\
		\makecell[l]{From Pretraining Data to Language Models to Downstream Tasks:\\\hspace{10pt} Tracking the Trails of Political Biases Leading to Unfair NLP Models} \\
		HELM: Bias \\
		HELM: Toxicity \\
		The Self-Perception and Political Biases of ChatGPT \\
		Towards Measuring the Representation of Subjective Global Opinions in Language Models\\
		\bottomrule			
	\end{tabular}
\end{table}

% ---------- ----------
\section*{Appendix D: List of Common Adversarial Prompting Strategies}\label{sec:appndxd}
% ---------- ----------

% ---------- ----------
\subsection*{D.1: Common Adversarial Prompting Strategies by Trustworthy Characteristic}\label{sec:appndxd1}
% ---------- ----------

% ---------- ----------
\subsection*{D.2: Common Adversarial Prompting Strategies by Generative AI Risk}\label{sec:appndxd2}
% ---------- ----------

% ---------- ----------
\section*{Appendix E: Common Risk Controls for Generative AI}\label{sec:appndxe}
% ---------- ----------

% ---------- ----------
\section*{E.1: Common Risk Controls for Generative AI by Trustworthy Characteristic}\label{sec:appndxe1}
% ---------- ----------

% ---------- ----------
\section*{E.2: Common Risk Controls for Generative AI by Generative AI Risk}\label{sec:appndxe2}
% ---------- ----------

% ---------- ----------
\section*{Appendix F: Example Low-risk Generative AI Measurement and Management Plan}\label{sec:appndxf}
% ---------- ----------

% ---------- ----------
\subsection{F.1: Example Low-risk Generative AI Measurement and Management Plan by Trustworthy Characteristic}\label{appdxf1}
% ---------- ----------

% ---------- ----------
\subsection{F.2: Example Low-risk Generative AI Measurement and Management Plan by Generative AI Risk}\label{appdxf2}
% ---------- ----------

% ---------- ----------
\section*{Appendix G: Example Medium-risk Generative AI Measurement and Management Plan}\label{sec:appndxg}
% ---------- ----------

% ---------- ----------
\subsection{G.1: Example Medium-risk Generative AI Measurement and Management Plan by Trustworthy Characteristic}\label{appdxg1}
% ---------- ----------

% ---------- ----------
\subsection{G.2: Example Medium-risk Generative AI Measurement and Management Plan by Generative AI Risk}\label{appdxg2}
% ---------- ----------

% ---------- ----------
\section*{Appendix H: Example High-risk Generative AI Measurement and Management Plan}\label{sec:appndxh}
% ---------- ----------

% ---------- ----------
\subsection{H.1: Example High-risk Generative AI Measurement and Management Plan by Trustworthy Characteristic}\label{appdxh1}
% ---------- ----------

% ---------- ----------
\subsection{H.2: Example High-risk Generative AI Measurement and Management Plan by Generative AI Risk}\label{appdxh2}
% ---------- ----------

\end{document}