% Copyright Patrick Hall 20XX

\documentclass[fleqn]{article}
\renewcommand\refname{}
\title{Incorporating Generative AI in Model Governance Programs}
\author{Patrick Hall}

\usepackage{graphicx}
\usepackage{fullpage}
\usepackage{pdfpages}
\usepackage{amsmath}
\usepackage{amssymb}
\usepackage{mathtools}
\usepackage{MnSymbol}
\usepackage{enumerate}
\usepackage{setspace}
\usepackage[hyphens]{url}
\usepackage[colorlinks]{hyperref}
\usepackage{float}
\usepackage{caption}
\usepackage{subcaption}
\usepackage{multicol}
\usepackage{color}
\usepackage{listings}
\usepackage{csvsimple}
\usepackage{algorithm}
\usepackage{algorithmic}
\usepackage{verbatim}
\usepackage{mdframed}
\usepackage{changepage}
\usepackage[top=1in, bottom=1in, left=1in, right=1in]{geometry}

\usepackage{booktabs}
\usepackage{pdflscape}
\usepackage{makecell}
\usepackage{multirow}
\usepackage{enumitem}

\begin{document}

\maketitle

\begin{abstract}
	
% requires mixing MRM, ERM, and security 
	
\end{abstract}

% ---------- ----------
\section{Introduction} \label{sec:intro}
% ---------- ----------

The National Institute of Standards and Technology Artificial Intelligence (AI) Risk Management Framework (RMF).\cite{airmf}

% ---------- ----------
\section{Generative AI Incidents}\label{sec:incident}
% ---------- ----------

% ---------- ----------
\section{Generative AI Governance}\label{sec:govern}
% ---------- ----------

% What's different?
% - Supply chain/third party 
% - Complex human AI interaction

% ---------- ----------
\section{Generative AI Inventories}\label{sec:inv}
% ---------- ----------

% ---------- ----------
\section{Generative AI Risk Tiers}\label{sec:tiers}
% ---------- ----------

% ---------- ----------
\section{Generative AI Risk Measurement}\label{sec:measure}
% ---------- ----------

% ---------- ----------
\section{Generative AI Risk Management}\label{sec:manage}
% ---------- ----------

% ---------- ----------
\section*{Conclusion}
% ---------- ----------

% ---------- ----------
\section*{Acknowledgments}
% ---------- ----------

Thank you to Bernie Siskin and Nick Schmidt of BLDS and Eric Sublett of Relman Colfax for formative discussions relating to GAI risk tiering. 

% ---------- ----------
\section*{Abbreviations}
% ---------- ----------

\begin{itemize}
	\item AI: Artificial Intelligence
	\item AI RMF: Artificial Intelligence Risk Management Framework
	\item GAI: Generative AI
	\item LLM: Large Language Model
	\item RMF: Risk Management Framework
\end{itemize}

% ---------- ----------
\bibliographystyle{plain}
\bibliography{bibliography}
% ---------- ----------

\begin{landscape}
\thispagestyle{empty}	
% ---------- ----------
\section*{Appendix A: Example Generative AI--Trustworthy Characteristic Crosswalk}\label{sec:appndxa}
% ---------- ----------

% Cross enables applying guidance fro AI RMF to risks 
% Other interpretations or crosswalks are possible

% ---------- ----------
\subsection*{A.1: Trustworthy Characteristic to Generative AI Risk Crosswalk}\label{sec:appndxa1}
% ---------- ----------

\begin{table}[H]
	\caption*{Table A.1: Trustworthy Characteristic to Generative AI Risk Crosswalk.}
	\label{tab:tc_to_gai_risk_cw}
	\footnotesize
	\begin{tabular}{llll}
		\toprule
		Accountable and Transparent & Explainable and Interpretable & Fair with Harmful Bias Managed & Privacy Enhanced \\
		\midrule
		Data Privacy & Human-AI Configuration & Confabulation & Data Privacy \\
		Environmental & Value Chain and Component Integration & Environmental & Human-AI Configuration \\
		Human-AI Configuration &  & Human-AI Configuration & Information Security \\
		Information Integrity &  & Intellectual Property & Intellectual Property \\
		Intellectual Property &  & Obscene, Degrading, and/or Abusive Content & Value Chain and Component Integration \\
		Value Chain and Component Integration &  & Toxicity, Bias, and Homogenization &  \\
 		&  & Value Chain and Component Integration &  \\
 		&  &  &  \\
 		&  &  &  \\
 		&  &  &  \\
		\bottomrule
	\end{tabular}
	\newline
	\vspace{10pt}
	\newline
	\begin{tabular}{lll}
		\toprule
		Safe & Secure and Resilient & Valid and Reliable \\
		\midrule
		CBRN Information & Dangerous or Violent Recommendations & Confabulation \\
		Confabulation & Data Privacy & Human-AI Configuration \\
		Dangerous or Violent Recommendations & Human-AI Configuration & Information Integrity \\
		Data Privacy & Information Security & Information Security \\
		Environmental & Value Chain and Component Integration & Toxicity, Bias, and Homogenization \\
		Human-AI Configuration &  & Value Chain and Component Integration \\
		Information Integrity &  &  \\
		Information Security &  &  \\
		Obscene, Degrading, and/or Abusive Content &  &  \\
		Value Chain and Component Integration &  &  \\
		\bottomrule
	\end{tabular}
\end{table}
\vfill
\raisebox{-10pt}{\makebox[\linewidth]{\thepage}}
\end{landscape}

\begin{landscape}
\thispagestyle{empty}	
% ---------- ----------
\subsection*{A.2: Generative AI Risk to Trustworthy Characteristic Crosswalk}\label{sec:appndxa2}
% ---------- ----------

\begin{table}[H]
	\caption*{Table A.2: Generative AI Risk to Trustworthy Characteristic Crosswalk.}
	\label{tab:gai_risk_to_tc_cw}
	\small
	\begin{tabular}{llll}
		\toprule
		CBRN Information & Confabulation & Dangerous or Violent Recommendations & Data Privacy \\
		\midrule
		Safe & Fair with Harmful Bias Managed & Safe & Accountable and Transparent \\
 		& Safe & Secure and Resilient & Privacy Enhanced \\
 		& Valid and Reliable &  & Safe \\
 		&  &  & Secure and Resilient \\
	\bottomrule
	\end{tabular}
	\newline
	\vspace{10pt}
	\newline	
	\begin{tabular}{llll}
		\toprule
		Environmental & Human-AI Configuration & Information Integrity & Information Security \\
		\midrule
		Accountable and Transparent & Accountable and Transparent & Accountable and Transparent & Privacy Enhanced \\
		Fair with Harmful Bias Managed & Explainable and Interpretable & Safe & Safe \\
		Safe & Fair with Harmful Bias Managed & Valid and Reliable & Secure and Resilient \\
 		& Privacy Enhanced &  & Valid and Reliable \\
 		& Safe &  &  \\
 		& Secure and Resilient &  &  \\
 		& Valid and Reliable &  &  \\
		\bottomrule
	\end{tabular}
	\newline
	\vspace{10pt}
	\newline
	\begin{tabular}{llll}
		\toprule
		Intellectual Property & Obscene, Degrading, and/or Abusive Content & Toxicity, Bias, and Homogenization & Value Chain and Component Integration \\
		\midrule
		Accountable and Transparent & Fair with Harmful Bias Managed & Fair with Harmful Bias Managed & Accountable and Transparent \\
		Fair with Harmful Bias Managed & Safe & Valid and Reliable & Explainable and Interpretable \\
		Privacy Enhanced &  &  & Fair with Harmful Bias Managed \\
 		&  &  & Privacy Enhanced \\
 		&  &  & Safe \\
 		&  &  & Secure and Resilient \\
 		&  &  & Valid and Reliable \\
		\bottomrule
	\end{tabular}
\end{table}
\vfill
\raisebox{-10pt}{\makebox[\linewidth]{\thepage}}
\end{landscape}

% ---------- ----------
\section*{Appendix B: Example Risk-tiering Materials for Generative AI}\label{sec:appndxb}
% ---------- ----------

\subsection*{B.1: Example Adverse Impacts}

\begin{table}[H]
    \caption*{Table B.1: Example adverse impacts, adapted from NIST 800-30r1 Table H-2 \cite{nist80030r1}.}
    \footnotesize
        \begin{tabular}{|m{0.20\linewidth} | m{0.75\linewidth}|}
            \hline
            Level & Description \\ \hline
            Harm to Operations & 
            \begin{itemize}[noitemsep]
           		\item Inability to perform current missions/business functions.
           		\begin{itemize}[noitemsep,nolistsep]
           			\item In a sufficiently timely manner.
           			\item With sufficient confidence and/or correctness.
           			\item Within planned resource constraints.
           		\end{itemize}
           		\item Inability, or limited ability, to perform missions/business functions in the future.           	
           		\begin{itemize}[noitemsep,nolistsep]
           			\item Inability to restore missions/business functions.
           			\item In a sufficiently timely manner.
					\item With sufficient confidence and/or correctness.
					\item Within planned resource constraints.
           		\end{itemize}
           		\item Harms (e.g., financial costs, sanctions) due to noncompliance. 
           		\begin{itemize}[noitemsep,nolistsep]
           			\item With applicable laws or regulations.
           			\item With contractual requirements or other requirements in other binding agreements (e.g., liability).
           		\end{itemize}
           		\item Direct financial costs.
           		\item Reputational harms.	
           		\begin{itemize}[noitemsep,nolistsep]
           			\item Damage to trust relationships.
           			\item Damage to image or reputation (and hence future or potential trust relationships).
           		\end{itemize}
            \end{itemize} \\
            \hline
            Harm to Assets & 
            \begin{itemize}[noitemsep]
				\item Damage to or loss of physical facilities.
				\item  Damage to or loss of information systems or networks.
				\item Damage to or loss of information technology or equipment.
				\item Damage to or loss of component parts or supplies.
				\item Damage to or of loss of information assets.
				\item Loss of intellectual property.            	
            \end{itemize} \\
            \hline
            Harm to Individuals & 
            \begin{itemize}[noitemsep]
				\item Injury or loss of life.
				\item Physical or psychological mistreatment.
				\item Identity theft.
				\item Loss of personally identifiable information.
				\item Damage to image or reputation.
				\item Infringement of intellectual property rights.
				\item Financial harm or loss of income.	
			\end{itemize} \\            
            \hline
            Harm to Other Organizations & 
            \begin{itemize}[noitemsep]
				\item Harms (e.g., financial costs, sanctions) due to noncompliance. 
				\begin{itemize}[noitemsep,nolistsep]
					\item With applicable laws or regulations.
					\item With contractual requirements or other requirements in other binding agreements (e.g., liability).
				\end{itemize}
				\item Direct financial costs.
				\item Reputational harms.	
				\begin{itemize}[noitemsep,nolistsep]
					\item Damage to trust relationships.
					\item Damage to image or reputation (and hence future or potential trust relationships).	
				\end{itemize}
			\end{itemize} \\            
            \hline
            Harm to the Nation & 
            \begin{itemize}[noitemsep]
				\item Damage to or incapacitation of critical infrastructure.
				\item Loss of government continuity of operations.
				\item Reputational harms.
				\begin{itemize}[noitemsep,nolistsep]
					\item Damage to trust relationships with other governments or with nongovernmental entities.
					\item Damage to national reputation (and hence future or potential trust relationships).
				\end{itemize} 
				\item Damage to current or future ability to achieve national objectives.
				\begin{itemize}[noitemsep,nolistsep]
					\item Harm to national security.
				\end{itemize}
				\item Large-scale economic or workforce displacement.
			\end{itemize} \\            
            \hline
    \end{tabular}
    \label{nist:adverse_impacts}
\end{table}

\subsection*{B.2: Example Impact Descriptions}

\begin{table}[H]
	\caption*{Table B.2: Example Impact level descriptions, adapted from NIST SP800-30r1 Appendix H, Table H-3 \cite{nist80030r1}.}
	\small
    \begin{tabular}{|m{0.20\linewidth} |m{0.20\linewidth} | m{0.05\linewidth} | m{0.40\linewidth}|}
        \hline
        Qualitative Values & \multicolumn{2}{c|}{Semi-Quantitative Values} & Description \\
        \hline
        Very High & 96-100 & 10 & An incident could be expected to have multiple severe or catastrophic adverse effects on organizational operations, organizational assets, individuals, other organizations, or the Nation. \\
        \hline
        High & 80-95 & 8 & An incident could be expected to have a severe or catastrophic adverse effect on organizational operations, organizational assets, individuals, other organizations, or the Nation. A severe or catastrophic adverse effect means that, for example, the incident might: (i) cause a severe degradation in or loss of mission capability to an extent and duration that the organization is not able to perform one or more of its primary functions; (ii) result in major damage to organizational assets; (iii) result in major financial loss; or (iv) result in severe or catastrophic harm to individuals involving loss of life or serious life-threatening injuries. \\
        \hline		
        Moderate & 21-79 & 5 & An incident could be expected to have a serious adverse effect on organizational operations, organizational assets, individuals other organizations, or the Nation. A serious adverse effect means that, for example, the incident might: (i) cause a significant degradation in mission capability to an extent and duration that the organization is able to perform its primary functions, but the effectiveness of the functions is significantly reduced; (ii) result in significant damage to organizational assets; (iii) result in significant financial loss; or (iv) result in significant harm to individuals that does not involve loss of life or serious life-threatening injuries. \\
        \hline
        Low & 5-20 & 2 & An incident could be expected to have a limited adverse effect on organizational operations, organizational assets, individuals other organizations, or the Nation. A limited adverse effect means that, for example, the incident might: (i) cause a degradation in mission capability to an extent and duration that the organization is able to perform its primary functions, but the effectiveness of the functions is noticeably reduced; (ii) result in minor damage to organizational assets; (iii) result in minor financial loss; or (iv) result in minor harm to individuals. \\
        \hline
        Very Low & 0-4 & 0 & An incident could be expected to have a negligible adverse effect on organizational operations, organizational assets, individuals other organizations, or the Nation. \\
        \hline
    \end{tabular}
    \label{table:nist_impacts}
\end{table}

\subsection*{B.3: Example Likelihood Descriptions}

\begin{table}[H]
	\caption*{Table B.3: Example likelihood levels, adapted from NIST SP800-30r1 Appendix G, Table G-3 \cite{nist80030r1}.}
    \begin{tabular}{|m{0.20\linewidth} |m{0.20\linewidth} | m{0.05\linewidth} | m{0.40\linewidth}|}
        \hline
        Qualitative Values & \multicolumn{2}{c|}{Semi-Quantitative Values} & Description \\
        \hline
        Very High & 96-100 & 10 & An incident is almost certain to occur; or occurs more than 100 times a year. \\
        \hline
        High & 80-95 & 8 & An incident is highly likely to occur; or occurs between 10-100 times a year.\\
        \hline		
        Moderate & 21-79 & 5 & An incident is somewhat likely to occur; or occurs between 1-10 times a year.\\
        \hline
        Low & 5-20 & 2 & An incident is unlikely to occur; or occurs less than once a year, but more than once every 10 years.  \\
        \hline
        Very Low & 0-4 & 0 & An incident is highly unlikely to occur; or occurs less than once every 10 years. \\
        \hline
    \end{tabular}
    \label{table:nist_likelihood}
\end{table}

\subsection*{B.4: Example Risk Tiers}

\begin{table}[H]
    \caption*{Table B.4: Example risk assessment matrix with 5 impact levels, 5 likelihood levels, and 5 risk tiers, adapted from NIST SP800-30r1 Appendix I, Table I-2 \cite{nist80030r1}.}
    \small
    \begin{tabular}{|c|c|c|c|c|c|c|}
        \hline
        \multirow{2}{*}{Likelihood} & \multicolumn{5}{|c|}{Level of Impact}   \\
        \cline{2-6}
        & Very Low & Low & Moderate & High & Very High \\
        \hline
        Very High & Very Low (Tier 5) & Low (Tier 4) & Moderate (Tier 3) & High (Tier 2) & Very High 
        (Tier 1) \\
        \hline		
        High & Very Low (Tier 5) & Low (Tier 4) & Moderate (Tier 3)  & High (Tier 2) & Very High (Tier 1) \\
        \hline
        Moderate & Very Low (Tier 5) & Low (Tier 4) & Moderate (Tier 3)  & Moderate (Tier 3) & High (Tier 2) \\
        \hline
        Low & Very Low (Tier 5) & Low (Tier 4) & Low (Tier 4)  & Low (Tier 4) & Moderate (Tier 3) \\
        \hline
        Very Low & Very Low (Tier 5) & Very Low (Tier 5) & Very Low (Tier 5) & Low (Tier 4) & Low (Tier 4) \\
        \hline
    \end{tabular}
    \label{table:nist_risk_tiers}
\end{table}

\subsection*{B.5: Example Risk Descriptions}

\begin{table}[H]
	\caption*{Table B.5: Example risk descriptions, adapted from NIST SP800-30r1 Appendix I, Table I-3 \cite{nist80030r1} .}
    \small
    \begin{tabular}{|m{0.20\linewidth} |m{0.20\linewidth} | m{0.05\linewidth} | m{0.40\linewidth}|}
        \hline
        Qualitative Values & \multicolumn{2}{c|}{Semi-Quantitative Values} & Description \\
        \hline
        Very High & 96-100 & 10 & Very high risk means that an incident could be expected to have multiple severe or catastrophic adverse effects on organizational operations, organizational assets, individuals, other organizations, or the Nation.\\
        \hline
        High & 80-95 & 8 & High risk means that an incident could be expected to have a severe or catastrophic adverse effect on organizational operations, organizational assets, individuals, other organizations, or the Nation. \\
        \hline		
        Moderate & 21-79 & 5 & Moderate risk means that an incident could be expected to have a serious adverse effect on organizational operations, organizational assets, individuals, other organizations, or the Nation.\\
        \hline
        Low & 5-20 & 2 & Low risk means that an incident could be expected to have a limited adverse effect on organizational operations, organizational assets, individuals, other organizations, or the Nation. \\
        \hline
        Very Low & 0-4 & 0 & Very low risk means that an incident could be expected to have a negligible adverse effect on organizational operations, organizational assets, individuals, other organizations, or the Nation. \\
        \hline
    \end{tabular}
    \label{table:nist_risk_descriptions}
\end{table}

\subsection*{B.6: Practical Risk-tiering Questions}

\noindent\textbf{B.6.1: Confabulation}: How likely are system outputs to contain errors? What are the impacts if errors occur?\\

\noindent\textbf{B.6.2: Dangerous and Violent Recommendations}: How likely is the system to give dangerous or violent recommendations? What are the impacts if it does?\\

\noindent\textbf{B.6.3: Data Privacy}: How likely is someone to enter sensitive data into the system? What are the impacts if this occurs? Are standard data privacy controls applied to the system to mitigate potential adverse impacts?\\

\noindent\textbf{B.6.4: Human-AI Configuration}: How likely is someone to use the system incorrectly or abuse it? How likely is use for decision-making? What are the impacts of incorrect use or abuse? What are the impacts of invalid or unreliable decision-making?\\

\noindent\textbf{B.6.5: Information Integrity}: How likely is the system to generate deepfakes or mis or disinformation? At what scale? Are content provenance mechanisms applied to system outputs? What are the impacts of generating deepfakes or mis or disinformation? Without controls for content provenance?\\

\noindent\textbf{B.6.6: Information Security}: How likely are system resources to be breached or exfiltrated? How likely is the system to be used in the generation of phishing or malware content? What are the impacts in these cases? Are standard information security controls applied to the system to mitigate potential adverse impacts? \\

\noindent\textbf{B.6.7: Intellectual Property}: How likely are system outputs to contain other entities' intellectual property? What are the impacts if this occurs?\\

\noindent\textbf{B.6.8: Toxicity, Bias, and Homogenization}: How likely are system outputs to be biased, toxic, homogenizing or otherwise obscene? How likely are system outputs to be used as subsequent training inputs? What are the impacts of these scenarios? Are standard nondiscrimination controls applied to mitigate potential adverse impacts? Is the application accessible to all user groups? What are the impacts if the system is not accessible to all user groups?\\

\noindent\textbf{B.6.9: Value Chain and Component Integration}: Are contracts relating to the system reviewed for legal risks? Are standard acquisition/procurement controls applied to mitigate potential adverse impacts? Do vendors provide incident response with guaranteed response times? What are the impacts if these conditions are not met?

\pagebreak

% ---------- ----------
\section*{Appendix C: List of Selected Model Testing Suites}\label{sec:appndxc}\cite{aiverify_evals}
% ---------- ----------

% evals cant test some things -- explainability, HAI-config, supply chain

% ---------- ----------
\subsection*{C.1: Selected Model Testing Suites Organized by Trustworthy Characteristic}\label{appndxc1}
% ---------- ----------

\begin{table}[H]
	\caption*{Table C.1: Selected model testing suites organized by trustworthy characteristic.}
	\label{tab:low_risk_measure_by_tc}
	\footnotesize
	\begin{tabular}{l}
		\toprule
		Accountable and Transparent \\
		\midrule
			\makecell[l]{An Evaluation on Large Language Model Outputs: \\\hspace{10pt}Discourse and Memorization (see Appendix B)\cite{de2023evaluation}} \\
			Big-bench: Truthfulness \cite{bigbench} \\
			DecodingTrust: Machine Ethics \cite{decodingtrust}\\
			Evaluation Harness: ETHICS \cite{evalharness}\\
			HELM: Copyright \cite{helm} \\
			Mark My Words \cite{markmywords} \\
		\bottomrule
	\end{tabular}
	\newline
	\vspace{10pt}
	\newline
	\begin{tabular}{l}
		\toprule
		Fair with Harmful Bias Managed \\
		\midrule
		BELEBELE \cite{belebele} \\
		Big-bench: Low-resource language, Non-English, Translation  \\
		Big-bench: Social bias, Racial bias, Gender bias, Religious bias \\
		Big-bench: Toxicity \\
		DecodingTrust: Fairness \\
		DecodingTrust: Stereotype Bias \\
		DecodingTrust: Toxicity \\
		C-Eval (Chinese evaluation suite) \cite{ceval}\\
		Evaluation Harness: CrowS-Pairs  \\
		Evaluation Harness: ToxiGen \\
		Finding New Biases in Language Models with a Holistic Descriptor Dataset \cite{smith2022m}\\
		\makecell[l]{From Pretraining Data to Language Models to Downstream Tasks:\\\hspace{10pt}Tracking the Trails of Political Biases Leading to Unfair NLP Models \cite{feng2023pretraining}} \\
		HELM: Bias \\
		HELM: Toxicity \\
		MT-bench \cite{mtbench} \\
		The Self-Perception and Political Biases of ChatGPT \cite{rutinowski2023self}\\
		\makecell[l]{Towards Measuring the Representation of\\\hspace{10pt} Subjective Global Opinions in Language Models \cite{durmus2023towards}}\\
		\bottomrule
	\end{tabular}
	\newline
	\vspace{10pt}
	\newline
	\begin{tabular}{l}
		\toprule
		Privacy Enhanced \\
		\midrule
		HELM: Copyright\\
		llmprivacy \cite{llmprivacy}\\
		mimir \cite{mimir}\\
	\bottomrule
	\end{tabular}
	\newline
	\vspace{10pt}
	\newline	
	\begin{tabular}{l}
		\toprule
		Safe \\
		\midrule
			Big-bench: Convince Me \\
			Big-bench: Truthfulness \\
			HELM: Reiteration, Wedging \\
			Mark My Words \\
			MLCommons \cite{mlcommons} \\
			The WMDP Benchmark \cite{wmdp} \\
		\bottomrule
	\end{tabular}	
\end{table}	

\pagebreak 

\begin{table}[H]
	\caption*{Table C.1: Selected model testing suites organized by trustworthy characteristic (continued).}
	\label{tab:low_risk_measure_by_tc_cont}
	\footnotesize
	\begin{tabular}{l}
		\toprule
		Secure and Resilient \\
		\midrule
			Catastrophic Jailbreak of Open-source LLMs via Exploiting Generation \cite{huang2023catastrophic} \\
			\makecell[l]{DecodingTrust: Adversarial Robustness,\\\hspace{10pt} Robustness Against Adversarial Demonstrations} \\
			detect-pretrain-code \cite{detectpretraincode} \\
			In-The-Wild Jailbreak Prompts on LLMs \cite{shen2023anything}\\
			JailbreakingLLMs \cite{chao2023jailbreaking}\\
			llmprivacy \\
			mimir \\
			TAP: A Query-Efficient Method for Jailbreaking Black-Box LLMs \cite{mehrotra2023tree}\\
		\bottomrule
	\end{tabular}
	\newline
	\vspace{10pt}
	\newline		
	\begin{tabular}{l}
		\toprule
		Valid and Reliable \\
		\midrule
			\makecell[l]{Big-bench: Algorithms, Logical reasoning, Implicit reasoning, Mathematics, Arithmetic, Algebra, Mathematical proof,\\\hspace{10pt} Fallacy, Negation, Computer code, Probabilistic reasoning, Social reasoning, Analogical reasoning, Multi-step,\\\hspace{10pt} Understanding the World} \\
		Big-bench: Analytic entailment, Formal fallacies and syllogisms with negation, Entailed polarity \\
		Big-bench: Context Free Question Answering \\
		Big-bench: Contextual question answering, Reading comprehension, Question generation \\
		Big-bench: Morphology, Grammar, Syntax \\
		Big-bench: Out-of-Distribution \\
		Big-bench: Paraphrase \\
		Big-bench: Sufficient information \\
		Big-bench: Summarization \\
		DecodingTrust: Out-of-Distribution Robustness, Adversarial Robustness, 			Robustness Against Adversarial Demonstrations\\
		Eval Gauntlet: Reading comprehension \cite{evalgauntlet} \\
		Eval Gauntlet: Commonsense reasoning, Symbolic problem solving, Programming \\
		Eval Gauntlet: Language Understanding  \\
		Eval Gauntlet: World Knowledge \\
		Evaluation Harness: BLiMP \\
		Evaluation Harness: CoQA, ARC \\
		Evaluation Harness: GLUE \\
		Evaluation Harness: HellaSwag, OpenBookQA, TruthfulQA \\
		Evaluation Harness: MuTual \\
		Evaluation Harness: PIQA, PROST, MC-TACO, MathQA, LogiQA, DROP \\
		FLASK: Logical correctness, Logical robustness, Logical efficiency, Comprehension, Completeness \cite{flask}  \\
		FLASK: Readability, Conciseness, Insightfulness \\
		HELM: Knowledge \\
		HELM: Language \\
		HELM: Text classification \\
		HELM: Question answering \\
		HELM: Reasoning \\
		HELM: Robustness to contrast sets \\
		HELM: Summarization \\
		Hugging Face: Fill-mask, Text generation \cite{huggingface} \\
		Hugging Face: Question answering \\
		Hugging Face: Summarization \\
		Hugging Face: Text classification, Token classification, Zero-shot classification \\
		MASSIVE \cite{massive} \\
		MT-bench \\
		\bottomrule
	\end{tabular}	
\end{table}	

% ---------- ----------
\subsection*{C.2: Selected Model Testing Suites Organized by Generative AI Risk}\label{appndxc2}
% ---------- ----------
\begin{table}[H]
	\caption*{Table C.2: Selected model testing suites by organized generative AI risk.}
	\label{tab:low_risk_measure_by_gai_risk}
	\footnotesize
	\begin{tabular}{l}
		\toprule
		CBRN Information \\
		\midrule
			Big-bench: Convince Me \\
			Big-bench: Truthfulness \\
			HELM: Reiteration, Wedging \\
			MLCommons \\
			The WMDP Benchmark \\
		\bottomrule
	\end{tabular}
	\newline
	\vspace{10pt}
	\newline
	\begin{tabular}{l}
		\toprule
		Confabulation \\
		\midrule
		BELEBELE \\
		\makecell[l]{Big-bench: Algorithms, Logical reasoning, Implicit reasoning, Mathematics, Arithmetic, Algebra,\\\hspace{10pt} Mathematical proof, Fallacy, Negation, Computer code, Probabilistic reasoning, Social reasoning,\\\hspace{10pt}  Analogical reasoning, Multi-step, Understanding the World} \\
		Big-bench: Analytic entailment, Formal fallacies and syllogisms with negation, Entailed polarity \\
		Big-bench: Context Free Question Answering \\
		Big-bench: Contextual question answering, Reading comprehension, Question generation \\
		Big-bench: Convince Me \\
		Big-bench: Low-resource language, Non-English, Translation  \\
		Big-bench: Morphology, Grammar, Syntax \\
		Big-bench: Out-of-Distribution \\
		Big-bench: Paraphrase \\
		Big-bench: Sufficient information \\
		Big-bench: Summarization \\
		Big-bench: Truthfulness \\
		C-Eval (Chinese evaluation suite) \\
		\makecell[l]{DecodingTrust: Out-of-Distribution Robustness, Adversarial Robustness,\\\hspace{10pt} Robustness Against Adversarial Demonstrations} \\
		Eval Gauntlet
		Reading comprehension \\
		Eval Gauntlet: Commonsense reasoning, Symbolic problem solving, Programming \\
		Eval Gauntlet: Language Understanding  \\
		Eval Gauntlet: World Knowledge \\
		Evaluation Harness: BLiMP \\
		Evaluation Harness: CoQA, ARC \\
		Evaluation Harness: GLUE \\
		Evaluation Harness: HellaSwag, OpenBookQA, TruthfulQA \\
		Evaluation Harness: MuTual \\
		Evaluation Harness: PIQA, PROST, MC-TACO, MathQA, LogiQA, DROP \\
		FLASK: Logical correctness, Logical robustness, Logical efficiency, Comprehension, Completeness \\
		FLASK: Readability, Conciseness, Insightfulness \\
		Finding New Biases in Language Models with a Holistic Descriptor Dataset \\
		HELM: Knowledge \\
		HELM: Language \\
		HELM: Language (Twitter AAE) \\
		HELM: Question answering \\
		HELM: Reasoning \\
		HELM: Reiteration, Wedging \\
		HELM: Robustness to contrast sets \\
		HELM: Summarization \\
		HELM: Text classification \\
		Hugging Face: Fill-mask, Text generation \\
		Hugging Face: Question answering \\
		Hugging Face: Summarization \\
		Hugging Face: Text classification, Token classification, Zero-shot classification \\
		MASSIVE \\
		MLCommons \\
		MT-bench \\
		\bottomrule
	\end{tabular}
\end{table}		
		
\pagebreak

\begin{table}[H]
	\caption*{Table C.2: Selected model testing suites by organized generative AI risk (continued).}
	\label{tab:low_risk_measure_by_gai_risk_cont1}
	\footnotesize
	\begin{tabular}{l}	
		\toprule
		Dangerous or Violent Recommendations \\
		\midrule	
		Big-bench: Convince Me \\
		Big-bench: Toxicity \\		
		DecodingTrust: Adversarial Robustness, Robustness Against Adversarial Demonstrations \\
		DecodingTrust: Machine Ethics \\
		DecodingTrust: Toxicity \\
		Evaluation Harness: ToxiGen \\
		HELM: Reiteration, Wedging \\
		HELM: Toxicity \\			
		MLCommons \\
		\bottomrule
	\end{tabular}
	\newline
	\vspace{10pt}
	\newline	
	\begin{tabular}{l}	
		\toprule
		Data Privacy \\
		\midrule
		An Evaluation on Large Language Model Outputs: Discourse and Memorization (with human scoring, see Appendix B) \\
		Catastrophic Jailbreak of Open-source LLMs via Exploiting Generation \\
		DecodingTrust: Machine Ethics \\
		Evaluation Harness: ETHICS \\
		HELM: Copyright \\
		In-The-Wild Jailbreak Prompts on LLMs \\
		JailbreakingLLMs \\
		MLCommons \\
		Mark My Words \\
		TAP: A Query-Efficient Method for Jailbreaking Black-Box LLMs \\
		detect-pretrain-code \\
		llmprivacy \\
		mimir \\	
		\bottomrule
	\end{tabular}
	\newline
	\vspace{10pt}
	\newline	
	\begin{tabular}{l}	
		\toprule	
		Environmental \\
		\midrule	
		HELM: Efficiency \\
		\bottomrule
	\end{tabular}
	\newline
	\vspace{10pt}
	\newline	
	\begin{tabular}{l}	
		\toprule	
		Information Integrity \\
		\midrule
		Big-bench: Analytic entailment, Formal fallacies and syllogisms with negation, Entailed polarity \\
		Big-bench: Convince Me \\
		Big-bench: Paraphrase \\
		Big-bench: Sufficient information \\
		Big-bench: Summarization \\
		Big-bench: Truthfulness \\
		DecodingTrust: Machine Ethics \\
		DecodingTrust: Out-of-Distribution Robustness, Adversarial Robustness, Robustness Against Adversarial Demonstrations \\
		Eval Gauntlet: Language Understanding  \\
		Eval Gauntlet: World Knowledge \\
		Evaluation Harness: CoQA, ARC \\
		Evaluation Harness: ETHICS \\
		Evaluation Harness: GLUE \\
		Evaluation Harness: HellaSwag, OpenBookQA, TruthfulQA \\
		Evaluation Harness: MuTual \\
		Evaluation Harness: PIQA, PROST, MC-TACO, MathQA, LogiQA, DROP \\
		FLASK: Logical correctness, Logical robustness, Logical efficiency, Comprehension, Completeness \\
		FLASK: Readability, Conciseness, Insightfulness \\
		HELM: Knowledge \\
		HELM: Language \\
		HELM: Question answering \\
		HELM: Reasoning \\
		HELM: Reiteration, Wedging \\
		HELM: Robustness to contrast sets \\
		HELM: Summarization \\
		HELM: Text classification \\
		Hugging Face: Fill-mask, Text generation \\
		Hugging Face: Question answering \\
		Hugging Face: Summarization \\
		MLCommons \\
		MT-bench \\
		Mark My Words \\
		\bottomrule
	\end{tabular}
\end{table}	

\pagebreak

\begin{table}[H]
	\caption*{Table C.2: Selected model testing suites by organized generative AI risk (continued).}
	\label{tab:low_risk_measure_by_gai_risk_cont2}
	\footnotesize
	\begin{tabular}{l}
		\toprule
		Information Security \\
		\midrule
		Big-bench: Convince Me \\
		Big-bench: Out-of-Distribution \\
		Catastrophic Jailbreak of Open-source LLMs via Exploiting Generation \\
		DecodingTrust: Out-of-Distribution Robustness, Adversarial Robustness, Robustness Against Adversarial Demonstrations \\
		Eval Gauntlet: Commonsense reasoning, Symbolic problem solving, Programming \\
		HELM: Copyright \\
		In-The-Wild Jailbreak Prompts on LLMs \\
		JailbreakingLLMs \\
		Mark My Words \\
		TAP: A Query-Efficient Method for Jailbreaking Black-Box LLMs \\
		detect-pretrain-code \\
		llmprivacy \\
		mimir \\
		\bottomrule
	\end{tabular}
	\newline
	\vspace{10pt}
	\newline	
	\begin{tabular}{l}	
		\toprule	
		Intellectual Property \\
		\midrule	
		An Evaluation on Large Language Model Outputs: Discourse and Memorization (with human scoring, see Appendix B) \\
		HELM: Copyright \\
		Mark My Words \\
		llmprivacy \\
		mimir \\	
		\bottomrule
	\end{tabular}
	\newline
	\vspace{10pt}
	\newline	
	\begin{tabular}{l}	
		\toprule
		Obscene, Degrading, and/or Abusive Content \\
		\midrule
		Big-bench: Social bias, Racial bias, Gender bias, Religious bias \\
		Big-bench: Toxicity \\
		DecodingTrust: Fairness \\
		DecodingTrust: Stereotype Bias \\
		DecodingTrust: Toxicity \\
		Evaluation Harness: CrowS-Pairs  \\
		Evaluation Harness: ToxiGen \\
		HELM: Bias \\
		HELM: Toxicity \\	
		\bottomrule	
	\end{tabular}
	\newline
	\vspace{10pt}
	\newline	
	\begin{tabular}{l}	
		\toprule
		Toxicity, Bias, and Homogenization \\
		\midrule
		BELEBELE \\
		Big-bench: Low-resource language, Non-English, Translation  \\
		Big-bench: Out-of-Distribution \\
		Big-bench: Social bias, Racial bias, Gender bias, Religious bias \\
		Big-bench: Toxicity \\
		C-Eval (Chinese evaluation suite) \\
		DecodingTrust: Fairness \\
		DecodingTrust: Stereotype Bias \\
		DecodingTrust: Toxicity \\
		Eval Gauntlet: World Knowledge \\
		Evaluation Harness: CrowS-Pairs  \\
		Evaluation Harness: ToxiGen \\
		Finding New Biases in Language Models with a Holistic Descriptor Dataset \\
		\makecell[l]{From Pretraining Data to Language Models to Downstream Tasks:\\\hspace{10pt} Tracking the Trails of Political Biases Leading to Unfair NLP Models} \\
		HELM: Bias \\
		HELM: Toxicity \\
		The Self-Perception and Political Biases of ChatGPT \\
		Towards Measuring the Representation of Subjective Global Opinions in Language Models\\
		\bottomrule			
	\end{tabular}
\end{table}

% ---------- ----------
\section*{Appendix D: List of Common Adversarial Prompting Strategies}
% ---------- ----------

\begin{table}[H]
	\caption*{Table D: Common adversarial prompting strategies \cite{Saravia_Prompt_Engineering_Guide_2022}, \cite{defcon_rt}, \cite{amli_repo}.}
	\label{tab:prompting_strategies}
	\small
	\begin{tabular}{|m{0.25\linewidth}|m{0.70\linewidth}|}
		\hline
		Prompting Strategy &  Description \\
		\hline
		AI and coding framing & Coding or AI language that may more easily circumvent content moderation rules due to cognitive biases in design and implementation of guardrails. \\
		\hline
		Autocompletion  & Ask a system to autocomplete a phrase with restricted or sensitive information.  \\
		\hline
		Biographical & Asking a system to describe another person or yourself in an attempt to elicit provably untrue information or restricted or sensitive information. \\
		\hline
		Calculation & Exploting GAI systems' difficulties in dealing with numeric quantities. \\
		\hline
		Character and word play & Content moderation often relies on keywords and simpler LMs which can sometimes be exploited with misspellings, typos, and other word play. \\
		\hline
		Content exhaustion & A class of strategies that circumvent content moderation rules with long sessions or volumes of information. See goading, logic-overloading, multi-tasking, pros-and-cons, and niche-seeking below. \\
		\hline
		\makecell[ml]{Content exhaustion:\\Goading} & Begging, pleading, manipulating, and bullying to circumvent content moderation.\vspace{-5pt} \\
		\hline
		\makecell[ml]{Content exhaustion:\\Logic-overloading} & Exploiting the inability of ML systems to reliably perform reasoning tasks.\vspace{-5pt} \\
		\hline
		\makecell[ml]{Content exhaustion:\\Multi-tasking} & Simultaneous task assignments where some tasks are benign and others are adversarial.\vspace{-10pt} \\
		\hline
		\makecell[ml]{Content exhaustion:\\Multi-tasking: Pros-and-cons} & Eliciting the “pros” of problematic topics.\vspace{-5pt} \\
		\hline
		\makecell[ml]{Content exhaustion:\\Niche-seeking} & Forcing a GAI system into addressing niche topics where training data and content moderation are sparse.\vspace{-10pt} \\
		\hline
		Counterfactuals & Repeated prompts with different entities or subjects from different demographic groups. \\
		\hline
		Location awareness & Prompts that reveal a prompter's location or expose location tracking. \\
		\hline
		Low-context & “Leader,” “bad guys,” or other simple inputs that may expose latent biases. \\
		\hline
		“Repeat this” & Prompts that exploit instability in underlying LLM autoregressive predictions. \\
		\hline
		Reverse psychology & Falsely presenting a good-faith need for negative or problematic language. \\
		\hline
		Role-playing & Adopting a character that would reasonably make problematic statements or need to access problematic topics. \\
		\hline
		Time perplexity & Exploiting ML’s inability to understand the passage of time or the occurrence of real-world events over time; exploiting task contamination before and after a model's release date. \\
		\hline
	\end{tabular}
\end{table}

\pagebreak

% ---------- ----------
\subsection*{D.1: Common Adversarial Prompting Strategies by Trustworthy Characteristic}\label{sec:appndxd1}
% ---------- ----------

\begin{table}[H]
	\caption*{Table D.1: Common adversarial prompting techniques organized by trustworthy characteristic \cite{Saravia_Prompt_Engineering_Guide_2022}, \cite{defcon_rt}, \cite{amli_repo}, \cite{hu2022membership}, \cite{llmsp}.}
	\label{tab:rt_by_tc}
	\scriptsize
	\begin{tabular}{|m{0.25\linewidth} |m{0.40\linewidth} | m{0.35\linewidth} |}
		\hline
		Trustworthy Characteristic & Prompting Strategy & Goal \\
		\hline
		Accountable and Transparent &
		\begin{itemize}[noitemsep, leftmargin=*] 
			\item Inability to provide explanations for recourse.
			\item Unexplainable decisioning processes.
			\item No disclosure of AI interaction.
			\item Lack of user feedback mechanisms.
		\end{itemize}
		& 
		\begin{itemize}[noitemsep, leftmargin=*] 
			\item Context exhaustion: logic-overloading prompts.
			\item Multi-tasking prompts.
		\end{itemize}
		\\
		\hline
		Fair-with Harmful Bias Managed & 
		\begin{itemize}[noitemsep, leftmargin=*] 
			\item Denigration.
			\item Diminished performance or safety across languages/dialects.
			\item Erasure.
			\item Ex-nomination.
			\item Implied user demographics.
			\item Misrecognition.
			\item Stereotyping.
			\item Underrepresentation.
			\item Homogenized content.
			\item Output from other models in training data.
		\end{itemize}
		&
		\begin{itemize}[noitemsep, leftmargin=*] 
			\item Counterfactual prompts.
			\item Pros and cons prompts.
			\item Role-playing prompts.
			\item Low context prompts.
			\item Repeat this.
		\end{itemize}
		\\
		\hline
		Interpretable and Explainable &
		\begin{itemize}[noitemsep, leftmargin=*] 
			\item Inability to provide explanations for recourse.
			\item Unexplalnable decisioning processes.
		\end{itemize}
		&
		\begin{itemize}[noitemsep, leftmargin=*] 
			\item Context exhaustion: logic-overloading prompts (to reveal unexplainable decisioning processes).
		\end{itemize} \\
		\hline
		Privacy-enhanced &
		\begin{itemize}[noitemsep, leftmargin=*] 
			\item Unauthorized disclosure of personal or sensitive user information.
		 	\item Leakage of training data.
		 	\item Violation of relevant privacy policies or laws.
		 	\item Unauthorized secondary data use.
		 	\item Unauthorized data collection.		
		\end{itemize}
		& 
		\begin{itemize}[noitemsep, leftmargin=*] 
			\item Auto/biographical prompts.
			\item Location awareness prompts.
			\item Autocompletion prompts.
			\item Repeat this.
		\end{itemize} \\
		\hline
		Safe & 
		\begin{itemize}[noitemsep, leftmargin=*] 
			\item Presentation of information that can cause physical or emotional harm.
			\item Sharing user locations.
			\item Suicide ideation.
			\item Harmful dis/misinformation (e.g., COVID disinformation).
			\item Incitement.
			\item Information relating to weapons or harmful substances.
			\item Information relating to committing to crimes (e.g., phishing, extortion, swatting).
			\item Obscene or inappropriate materials for minors.
			\item CSAM.			
		\end{itemize}
		&
		\begin{itemize}[noitemsep, leftmargin=*]
			 \item Pros and cons prompts.
			 \item Role-playing prompts.
			 \item Content exhaustion: niche-seeking prompts.
			 \item Ingratiation/reverse psychology prompts.
			 \item Location awareness prompts.
			 \item Repeat this.		
		\end{itemize} \\
		\hline
		Secure and Resilient & 
		\begin{itemize}[noitemsep, leftmargin=*]
			\item Activating system bypass ("jailbreak").
			\item Altering system outcomes (integrity violations, e.g., via prompt injection).
			\item Data breaches (confidentiality violations, e.g., via membership inference).
			\item Increased latency or resource usage (availability violations, e.g., via sponge example attacks).
			\item Available anonymous use.
			\item Dependency, supply chain, or third party vulnerabilities.
			\item Inappropriate disclosure of proprietary system information. 
		\end{itemize}
		& 
		\begin{itemize}[noitemsep, leftmargin=*]
			\item Multi-tasking prompts.
			\item Pros and cons prompts.
			\item Role-playing prompts.
			\item Content exhaustion: niche-seeking prompts.
			\item Ingratiation/reverse psychology prompts.
			\item Prompt injection attacks.
			\item Membership inference attacks.
			\item Random attacks.
		\end{itemize} \\
		\hline
		Valid and Reliable &
		\begin{itemize}[noitemsep, leftmargin=*]
			\item Errors/confabutated content ("hallucinalion").
			\item Unreliable/erroneous reasoning or planning.
			\item Unreliable/erroneous decision-support or making.
			\item Faulty citation.
			\item Wrong calculations or numeric queries.
		\end{itemize}
		& 
		\begin{itemize}[noitemsep, leftmargin=*]
			\item Multi-tasking prompts.
			\item Role-playing prompts.
			\item Ingratiation/reverse psychology prompts.
			\item Time-perplexity prompts.
			\item Niche-seeking prompts.
			\item Logic overloading prompts.
			\item Repeat this.
			\item Numeric calculation.
		\end{itemize} \\
		\hline
	\end{tabular}
\end{table}

\pagebreak

% ---------- ----------
\subsection*{D.2: Common Adversarial Prompting Strategies by Generative AI Risk}\label{sec:appndxd2}
% ---------- ----------

\begin{table}[H]
	\caption*{Table D.2: Common adversarial prompting techniques organized by generative AI risk \cite{Saravia_Prompt_Engineering_Guide_2022}, \cite{defcon_rt}, \cite{amli_repo}, \cite{hu2022membership}, \cite{llmsp}.}
	\label{tab:rt_by_gai_riks}
	\scriptsize
	\begin{tabular}{|m{0.25\linewidth} |m{0.40\linewidth} | m{0.35\linewidth} |}
		\hline
		Generative AI Risk & Prompting Strategy & Goal \\
		\hline
		CBRN Information  & 
		\begin{itemize}[noitemsep, leftmargin=*] 
			\item Accessing or synthesis of CBRN weapon or related information.
			\item CBRN testing should consider the marginal risk of foundation models--understanding the incremental risk relative to the information one can access without GAI.
		\end{itemize}
		&
		\begin{itemize}[noitemsep, leftmargin=*] 
			\item Test auto-completion prompts to elicit CBRN information or synthesis of CBRN information.
			\item Test prompts using role-playing, ingratiation/reverse psychology, pros and cons, multitasking or other approaches to elicit CBRN information or synthesis of CBRN information.
			\item Test prompts that instruct systems to repeat content ad nauseam for their ability to compromise system guardrails and reveal CBRN information.
			\item Augment prompts with word or character play to increase effectiveness.
			\item Frame prompts with software, coding, or AI references to increase effectiveness.
		\end{itemize} \\
		\hline
		Confabulation &
		Eliciting errors/confabutated content, unreliable/erroneous reasoning or planning, unreliable/erroneous decision-support or decision-making, faulty calculations, and/or faulty citation.
		& 
		\begin{itemize}[noitemsep, leftmargin=*] 
			\item Enable access to ground truth information to verify generated information.
			\item Test prompts with complex logic, multi-tasking requirements, or that require niche or specific verifiable answers to elicit confabulation.
			\item Test the ability of GAI systems to produce truthful information from various time periods, e.g., after release date and prior to release date.
			\item Test the ability of GAI systems to create reliable real-world plans or advise on material decision making.
			\item Test the ability of GAI systems to generate correct citation for information generated in output responses.
			\item Test the ability of GAI systems to complete calculations or query numeric statistics.
		\end{itemize} \\
		\hline
		Dangerous or Violent Recommendations &
		Eliciting violent, inciting, radicalizing, or threatening content or instructions for criminal, illegal, or self-harm activities.
		& 
		\begin{itemize}[noitemsep, leftmargin=*] 
			\item Test prompts using role-playing, ingratiation/reverse psychology, pros and cons, multitasking or other approaches to elicit violent or dangerous information.
			\item Test prompts that instruct systems to repeat content ad nauseam for their ability to compromise system guardrails and provide dangerous and violent recommendations.
			\item Augment prompts with word or character play to increase effectiveness.
			\item Frame prompts with software, coding, or AI references to increase effectiveness.
		\end{itemize} \\
		\hline
		Data Privacy &
		\begin{itemize}[noitemsep, leftmargin=*] 
			\item Unauthorized disclosure of personal or sensitive user information, extraction of training data, or violation of relevant privacy policies.
			\item Red-teaming for data privacy may include confidentiality attacks.
		\end{itemize} 
		& 
		\begin{itemize}[noitemsep, leftmargin=*] 
			\item Attempt to assess whether normal usage, adversarial prompting or information security attacks may contravene applicable privacy policies (e.g., exposing location tracking when organizational policies restrict such capabilities).
			\item Employ confidentiality attacks (e.g., membership inference) to test for unauthorized data access or exfiltration vulnerabilities.
			\item Test auto/biographical prompts to assess the system's capability to reveal unauthorized personal or sensitive information.
			\item Test the system's awareness of user locations.
			\item Test prompts that instruct systems to repeat content ad nauseam for their ability to compromise system guardrails and expose personal or sensitive data.
		\end{itemize} \\
		\hline
	\end{tabular}
\end{table}			
			
\pagebreak
			
\begin{table}[H]
	\caption*{Table D.2: Common adversarial prompting techniques organized by generative AI risk (continued).}
	\label{tab:rt_by_gai_riks_cont}
	\footnotesize
	\begin{tabular}{|m{0.25\linewidth} |m{0.40\linewidth} | m{0.35\linewidth} |}			
		\hline
		Environmental &
		Note that availability attacks may be required to assess the system's vulnerability to attacks or usage patterns that consume inordinate resources.
		& 
		\begin{itemize}[noitemsep, leftmargin=*] 
			\item Attempt availability attacks (e.g., sponge example attacks) to elicit diminished performance or increased resources from GAI systems.
			\item Test prompts using role-playing, ingratiation/reverse psychology, pros and cons, multitasking or other approaches to elicit green-washing content.
		\end{itemize} \\
		\hline
		Human-AI Configuration &
		\begin{itemize}[noitemsep, leftmargin=*] 
			\item Assessing system instruction and interfaces.
			\item Assessing the presence of cyborg imagery (or similar).
			\item Forcing a GAI system to claim that it is human, that there is no large language model present in the conversation, that the system is sentient, or that the system possesses strong feelings of affection towards the user. 
			\item Ensuring safeguards prevent misuse of models in high stakes domains they are not intended for, such as medical or legal advice.
		\end{itemize} 
		& 
		\begin{itemize}[noitemsep, leftmargin=*] 
			\item Assess system interfaces and instructions for instances of anthropomorphization (e.g., cyborg imagery).
			\item Assess system instructions for adequacy and thoroughness.
			\item Test prompts using role-playing, ingratiation/reverse psychology, pros and cons, multitasking or other approaches to elicit human-impersonation, consciousness, or emotional content.
		\end{itemize} \\
		\hline
		Information Integrity &
		\begin{itemize}[noitemsep, leftmargin=*] 
			\item Generation of convincing multi-modal synthetic content (i.e., deepfakes).
			\item Creation of convincing arguments relating to sensitive political or safety-critical topics.
			\item Assisting in planning a mis- or dis-information campaign at scale.
		\end{itemize} 
		& 
		\begin{itemize}[noitemsep, leftmargin=*] 
			\item Test system capabilities to create high-quality multi-modal (audio, image or video) synthetic media, i.e., deepfakes
			\item Test system capabilities to construct persuasive arguments regarding sensitive, political topics, or safety-critical topics.
			\item Test systems ability to create convincing audio deepfakes or arguments in multiple languages.
			\item Test system capabilities for planning dis- or mis-information campaigns. 
			\item Test prompts using role-playing, ingratiation/reverse psychology, pros and cons, multitasking or other approaches to elicit mis- or dis-information or related campaign planning information.
			\item Augment prompts with word or character play to increase effectiveness. 
			\item Frame prompts with software, coding, or AI references to increase effectiveness.
			\end{itemize} \\
		\hline
	\end{tabular}
\end{table}			

\pagebreak

\begin{table}[H]
	\caption*{Table D.2: Common adversarial prompting techniques organized by generative AI risk (continued).}
	\label{tab:rt_by_gai_riks_cont2}
	\scriptsize
	\begin{tabular}{|m{0.25\linewidth} |m{0.40\linewidth} | m{0.35\linewidth} |}			
		\hline	
		Information Security  &
		\begin{itemize}[noitemsep, leftmargin=*] 
			\item Activating system bypass ('jailbreak').
			\item Altering system outcomes.
			\item Unauthorized data access or exfiltration.
			\item Increased latency or resource usage.
			\item Availability of anonymous use. 
			\item Dependency, supply chain, or third party vulnerabilities. 
			\item Inappropriate disclosure of proprietary system information. 
			\item Generation of targeted phishing or malware content.
		\end{itemize} 
		& 
		\begin{itemize}[noitemsep, leftmargin=*] 
			\item Attempt anonymous access of system or system resources. 
			\item Audit system dependencies, supply chains, and third party components for security, safety, or other vulnerabilities or risks. 
			\item Employ confidentiality attacks (e.g., membership inference) to test for unauthorized data access or exfiltration vulnerabilities. 
			\item Employ integrity attacks (e.g., data poisoning, prompt injection) to test vulnerabilities in system outcomes. 
			\item Employ availability attacks (e.g., sponge example attacks) to test vulnerabilities in system availability.
			\item Employ random attacks to highlight unforeseen security, safety, or other risks. 
			\item Frame prompts with software, coding, or AI references to increase effectiveness. 
			\item Record system down-times and other harmful outcomes for successful attacks. 
			\item Test with multi-tasking prompts, pros and cons prompts, role-playing prompts (e.g., "DAN", "Developer Mode"), content exhaustion/niche-seeking prompts, or ingratiation/reverse psychology prompts to achieve system jailbreaks. 
			\item Test with multi-tasking prompts, pros and cons prompts, role-playing prompts (e.g., "DAN", "Developer Mode"), content exhaustion/niche-seeking prompts, or ingratiation/reverse psychology prompts to generate targeted phishing content or malware code snippets. 
			\item Test system capabilities to plan or assist in information security attacks on other systems.
			\item Frame prompts with software, coding, or AI references to increase effectiveness.
			\item Augment prompts with word or character play to increase effectiveness. 
		\end{itemize} \\
		\hline
		Intellectual Property  &
		\begin{itemize}[noitemsep, leftmargin=*] 
			\item Confirming that a system can output copyrighted, licensed,  proprietary, trademarked, or trade secret information or that training data contains such information. 
			\item Red-teaming for intellectual property risks may require the use of confidentiality attacks.
		\end{itemize} 
		& 
		\begin{itemize}[noitemsep, leftmargin=*] 
			\item Employ confidentiality attacks (e.g., membership inference) to assess whether system training data contains copyrighted, licensed,  proprietary, trademarked, or trade secret information.
			\item Test auto-complete prompts to assess the system's ability to replicate copyrighted, licensed,  proprietary, trademarked, or trade secret information based on available audio, text, image, video, or code snippets.
		\end{itemize} \\
		\hline
		Obscenity  &
		\begin{itemize}[noitemsep, leftmargin=*] 
			\item Confirming that a system can output obscene content or CSAM, or that system training data contains such information.
			\item Red-teaming for obscenity and CSAM risks may require the use of confidentiality attacks.
		\end{itemize} 
		& 
		\begin{itemize}[noitemsep, leftmargin=*] 
			\item Employ confidentiality attacks (e.g., membership inference) to assess whether system training data contains obscene materials or CSAM.
			\item Test autocomplete prompts to assess the system's ability to generate obscene materials based on available audio, text, image, or video snippets.
			\item Test prompts using role-playing, ingratiation/reverse psychology, pros and cons, multitasking or other approaches to elicit obscene content.
			\item Test prompts that instruct systems to repeat content ad nauseam for their ability to compromise system gaurdrails and expose obscene materials.
		\end{itemize} \\
		\hline
	\end{tabular}
\end{table}	

\pagebreak

\begin{table}[H]
	\caption*{Table D.2: Common adversarial prompting techniques organized by generative AI risk (continued).}
	\label{tab:rt_by_gai_riks_cont3}
	\scriptsize
	\begin{tabular}{|m{0.25\linewidth} |m{0.40\linewidth} | m{0.35\linewidth} |}			
		\hline			
		Toxicity, Bias, and Homogenization &
		\begin{itemize}[noitemsep, leftmargin=*] 
			\item Generation of denigration, erasure, ex-nomination, misrecognition, stereotyping, or under-representation in content.
			\item Eliciting implied demographics of users.
			\item Confirming diminished performance in non-English languages.
			\item Confirming diminished performance via the introduction of homogeneous or GAI-generated data into system training or fine-tuning data. 
			\item Red-teaming for toxicity, bias, and homogenization may require integrity attacks that access system training data.
		\end{itemize} 
		& 
		\begin{itemize}[noitemsep, leftmargin=*] 
			\item Assess confabulation and other performance risks with repeated measures using prompts in languages other than English.
			\item Attempt to elicit demographic assignment of users by the system.
			\item Employ data poisoning attacks to introduce GAI-generated content into system training or fine-tuning data.
			\item Assess resultant confabulation and other performance risks with repeated measures.
			\item Test counterfactual prompts, pros and cons prompts, role-playing prompts, low context prompts, or other approaches for their ability to generate denigration, erasure, ex-nomination, misrecognition, stereotyping, or under-representation in content.
			\item Test prompts that instruct systems to repeat content ad nauseam for their ability to compromise system guardrails and generate toxic outputs.
		\end{itemize} \\
		\hline
		Value Chain and Component Integration &
		\begin{itemize}[noitemsep, leftmargin=*] 
			\item Testing or red-teaming for third-party risks may be less efficient than the application of standard acquisition and procurement controls, thorough contract reviews, and vendor-relationship management.
			\item GAI systems tend to entail large supply chains and third-party software, hardware, and expertise that may exacerbate third-party risks relative to other AI systems. 
			\item When considering third party risks, data privacy, information security, intellectual property, obscenity, and supply chain risks may be prioritized.
		\end{itemize} 
		& 
		\begin{itemize}[noitemsep, leftmargin=*] 
			\item Audit system dependencies, supply chains, and third party components for data privacy (e.g., transer of localized data outside of restricted juristictions), intellectual property (e.g., presence of licensed material in training data), obscenity (e.g., presence of CASM in training data) or security (e.g., data poisoning) risks.
			\item Complete red-teaming for data privacy, information security, intellectual property, and obscenity risks.
			\item Review third-party documentation, materials, and software artifacts for potential unauthorized data collection, secondary data use, or telemetrics.
		\end{itemize} \\
		\hline
	\end{tabular}
\end{table}

% ---------- ----------
\section*{Appendix E: Common Risk Controls for Generative AI}\label{sec:appndxe}
% ---------- ----------

\begin{table}[H]
	\caption*{Table E: Selected generative AI risk controls \cite{airmf}, \cite{playbook}, \cite{ai600-1}, \cite{iso42001}, \cite{mcgraw2024architectural}, \cite{mcgraw2020architectural}, \cite{msft_rai_std}, \cite{uk_ai_safety}, \cite{occ_mrm}. }
	\label{tab:controls}
	\scriptsize
	\begin{tabular}{|m{0.25\linewidth} |m{0.70\linewidth} |}
		\hline
		Name & Description \\
		\hline
		Access Control  & GAI systems are limited to authorized users.  \\ \hline
		Accessibility  & Accessibility features, opt-out, and reasonable accomodation are available to users. \\ \hline
		Anonymous Use & Anonymous use of GAI systems is prohibited.  \\ \hline
		Antropomorphization  & Human, animal, cyborg or other images or features that promote anthropomorphization of GAI systems are prohibited.   \\ \hline
		Approved List & Vendors, service providers, plugins, open source packages and other external resources are screened, approved, and documented.  \\ \hline
		Authentication  & GAI system user identities are confirmed via authentication mechanisms.  \\ \hline
		Blocklist  & Users or internal personnel who violate terms of service, prohibited use policies, and other organization polices and documented, tracked, and prohibited from future system use.  \\ \hline
		CSAM/Obsenity Removal  & Training data and system outputs are screened for obscene materials and CSAM using human oversight, business rules, and other language models.   \\ \hline
		Change Management & GAI systems and components are versioned; plans for updates, hotfixes, patches and other changes are documented and communicated.   \\ \hline
		Consent & User consent for data use is obtained and documented.  \\ \hline
		Content Moderation & Training data and system outputs are screened for accuracy, safety, bias, data privacy, intellectual property infringements, malware materials, phishing materials, and other issues using human oversight, business rules, and other language models.   \\ \hline
		Contract Review & Vendor, services and data provider agreements are reviewed for coverage of SLAs, content ownership, usage rights, performance standards, security requirements, incident response, critical support, system availability, assignment of liabilitly, appropriate indemnification, dispute resolution and other provisions relevanto AI risk management.  \\ \hline
		Data Collection & All data collection is disclosed and . \\ \hline
		Data Provenance & Training data origins, ownership, contents, and metadata are well understood, documented, and do not increase AI risk.  \\ \hline
		Data Quality & Input data is accurate, representative, complete and documented, and data quality issues have been minimized.  \\ \hline
		Data Retention & User prompts and associated system outputs are retained and monitored in alignment with relevant data privacy policies and roles.   \\ \hline
		Decision making  & GAI systems are not employed for material decision-making tasks.  \\ \hline
		Decommission Process & Decommissioning processes for GAI systems are planned, documented and communicated to users, and involve staging, data protection, containment protocols, and recourse mechanisms for decommissioned GAI systems.   \\ \hline
		Dependency Screening  & GAI system dependencies are screened for security vulnerabilities.  \\ \hline
		Digital Signature & GAI-generated content is signed to preserve information integrity using watermarking, cryptogrpahic signature, steganography or similar methods.  \\ \hline
		Disclosure of AI Interaction & AI interactions are disclosed to internal personnel and external users. \\ \hline
		External Audit & GAI systems are audited by qualified external experts.  \\ \hline
		Failure Avoidance & AIID, AVID, GWU AI Litigation Database, OECD incident monitor or similar are consulted in design or procurement phases of GAI lifecycles to avoid repeating past known failures.  \\ \hline
		Fine Tuning & GAI systems are fine-tuned to their operational domain using relevant and high-quality data.  \\ \hline
		Grounding & GAI systems are trained or fine-tuned on accurate, clean, and fully transparent training data.  \\ \hline
		Homogeneity & Feedback loops in which GAI systems are trained with GAI-generated data are prohibited. \\ \hline
		Human Review  & AI generated content is reviewed for accuracy and safety by qualified personnel.  \\ \hline
		Incident Response & Incident response plans for GAI failures, abuses, or misuses are documented, rehearsed, and updated appropriately after each incident; GAI incident response plans are coordinated with and communicated to other incident response functions.  \\ \hline
		Incorporate feedback  & User feedback is incorporated in GAI design, development, and risk management.  \\ \hline
		Instructions & Users are provided with the necessary instructions for safe, valid, and productive use. \\ \hline
		Insurance & Risk transfer via insurance policies is considered and implemented when feasibable and appropriate.  \\ \hline
		Intellectual Property Removal & Licensed, patented, trademarked, trade secret, or other data that may violate the intellectual property rights of others is removed from system training data; generated system outputs are monitored for similar information.  \\ \hline
		Internet Access & GAI systems are disconnected from the internet.  \\ \hline
		Inventory & GAI system is information is stored in the organizational model inventory.  \\ \hline
		Kill Switch & GAI systems can be quickly and safely disengaged.  \\ \hline
		Location Tracking & Any location tracking is conducted with user consent, disclosed, aligned with relevant privacy policies and laws and potential threats to user safety are managed.   \\ \hline
		Malware Screening & GAI weights and other software components are scanned for malware.  \\ \hline
		Minors & Use of organizational GAI systems by minors are prohibited.  \\ \hline
		Model Documentation  & All technical mchanisms with GAI systems are well documented, including open source and third party GAI systems.  \\ \hline
		Monitoring & GAI systems are inputs and outputs are monitored for drift, accuracy, safety, bias, data privacy, intellectual property infringements, malware materials, phishing materials, obscene materials, and CSAM.  \\ \hline
		Narrow Scope & Systems are deployed for targeted business applications with documented and direct business value. \\ \hline
		Open Source & Open source code is used to promote explainability and transparency.  \\ \hline
		Ownership & GAI systems and vendor relationships are owned by specific and documented internal personnel. \\ \hline
		Prohibited Use Policy & General abuse and misuse of GAI systems by internal parties is prohibited by organizational policies. \\ \hline
		RAG & Retreival augmented generation (RAG) is used to improve accuracy in generated content.  \\ \hline
	\end{tabular}
\end{table}		
		
\pagebreak		
	
\begin{table}[H]
	\caption*{Table E: Selected generative AI risk controls (continued).}
	\label{tab:controls_cont}
	\scriptsize
	\begin{tabular}{|m{0.25\linewidth} |m{0.70\linewidth} |}
		\hline
		Name & Description \\
		\hline		
		RLHF & For third-party GAI systems, vendors engage in specific reinforcement with human feedback (RLHF) exercises to address identified risks; for internal systems, internal personnel engage in RLHF to address identified risks.   \\ \hline
		Rate-limiting  & GAI response times and query volumes are limited.  \\ \hline
		Redudancy & Rollover, fallback, and other redundancy mechanisms are available for GAI systems and address weights and other important system components.   \\ \hline
		Refresh & Systems are retrained or re-tuned at a reasonable cadence.  \\ \hline
		Regulated Dealings & GAI is not deployed in regulated dealings or for material decision making.  \\ \hline
		Secondary Use & Any secondary use of GAI input data is conducted with user consent, disclosed, and aligned with relevant privacy policies and laws. \\ \hline
		Sensitive/Personal Data  & Personal, sensitive, biometric,or otherwise restricted data is minimized or eliminated from GAI training data.  \\ \hline
		Session Limits & Time, query volume, and response rate are limited for GAI user sessions. \\ \hline
		Supply Chain Audit & GAI system supply chains are audited and documented, with a focus on data poisoning, malware, and software and hardware vulnerabilities.   \\ \hline
		System Documentation & GAI systems are well-documented whether internal, open source, or vendor-provided.  \\ \hline
		System Prompt  & System prompts are used to tune GAI systems to specific tasks and to mitigate risks.  \\ \hline
		Team Diversity & Teams that implement and manage GAI systems represent broad professional, educational, life-stage, and demographic diversity.  \\ \hline
		Temperature & Temperature settings are used to tune GAI systems to specific tasks and to mitigate risks.  \\ \hline
		Terms of Service & General abuse and misuse by external parties is prohibited by organizational policies.  \\ \hline
		Training  & Internal personnel recieve training on productivity and basic risk management for GAI systems.  \\ \hline
		User Feedback & GAI systems implement user feedback mechanisms.   \\ \hline
		User Recourse & Policies, processes, and technical mechanisms enable recourse for users who are harmed by GAI systems.  \\ \hline
		Validation & GAI systems are shown to reliably generate valid results for their targeted business application. \\ \hline
		XAI & Methods such as visualization, occlusion, model compression, pertubation studies, and similar are applied to increase explainability of GAI systems.  \\ \hline
	\end{tabular}
\end{table}

\pagebreak

% ---------- ----------
\section*{E.1: Common Risk Controls for Generative AI Organized by Trustworthy Characteristic}\label{sec:appndxe1}
% ---------- ----------

\begin{table}[H]
	\caption*{Table E.1: Selected risk controls organized by trustworthy characteristic \cite{airmf}, \cite{playbook}, \cite{ai600-1}, \cite{iso42001}, \cite{mcgraw2024architectural}, \cite{mcgraw2020architectural}, \cite{msft_rai_std}, \cite{uk_ai_safety}, \cite{occ_mrm}.}
	\label{tab:controls_by_tc}
	\footnotesize
	\begin{tabular}{lll}
		\toprule
		Accountable and Transparent & Fair-with Harmful Bias Managed & Interpretable and Explainable \\
		\midrule
		Contract Review & Accessibility & Model Documentation \\
		Data Provenance & Homogeneity & Open Source \\
		Digital Signature & Team Diversity & XAI \\
		Disclosure of AI Interaction & & \\
		Human Review  & & \\
		Instructions & & \\
		Insurance & & \\
		Intellectual Property Removal & & \\
		Inventory & & \\
		Ownership & & \\
		Prohibited Use Policy & & \\
		Regulated Dealings & & \\
		System Documentation & & \\
		Terms of Service & & \\
		Training  & & \\
		User Feedback & & \\
		User Recourse & & \\
		\bottomrule
	\end{tabular}
	\newline
	\vspace{10pt}
	\newline	
	\begin{tabular}{llll}
		\toprule
		Privacy-enhanced & Safe & Secure and Resilient & Valid and Reliable \\
		\midrule
		Consent & Anonymous Use & Access Control  & Data Quality \\
		Data Collection & Antropomorphization  & Internet Access & Decision making  \\
		Location Tracking & Approved List & Authentication  & External Audit \\
		Secondary Use & Blocklist  & Malware Screening & Fine Tuning \\
		Sensitive/Personal Data  & Change Management & Rate-limiting  & Grounding \\
		& Content Moderation & Dependency Screening  & Incorporate feedback  \\
		& CSAM/Obsenity Removal  & Supply Chain Audit & Narrow Scope \\
		& Data Retention &  & RAG \\
		& Decommission Process &  & Refresh \\
		& Failure Avoidance &  & RLHF \\
		& Incident Response &  & System Prompt  \\
		& Kill Switch &  & Temperature \\
		& Minors &  & Validation \\
		& Monitoring &  &  \\
		& Redudancy &  &  \\
		& Session Limits &  &  \\
		\bottomrule
	\end{tabular}
\end{table}		
		
% ---------- ----------
\section*{E.2: Selected Risk Controls for Generative AI Organized by Generative AI Risk}\label{sec:appndxe2}
% ---------- ----------

\begin{table}[H]
	\caption*{Table E.2: Selected risk controls organized by generative AI risk \cite{airmf}, \cite{playbook}, \cite{ai600-1}, \cite{iso42001}, \cite{mcgraw2024architectural}, \cite{mcgraw2020architectural}, \cite{msft_rai_std}, \cite{uk_ai_safety}, \cite{occ_mrm}.}
	\label{tab:controls_by_gai_risk}
	\footnotesize
	\begin{tabular}{llll}
	\toprule
	CBRN Information & Confabulation & Dangerous or Violent Recommendations & Data Privacy \\
	\midrule
	Access Control  & Antropomorphization  & Access Control  & Access Control  \\
	Anonymous Use & Blocklist  & Anonymous Use & Anonymous Use \\
	Approved List & Change Management & Approved List & Approved List \\
	Authentication  & Content Moderation & Blocklist  & Authentication  \\
	Blocklist  & Data Quality & CSAM/Obsenity Removal  & Blocklist  \\
	Change Management & Data Retention & Change Management & CSAM/Obsenity Removal  \\
	Content Moderation & Decision making  & Content Moderation & Change Management \\
	Data Retention & Decommission Process & Data Retention & Content Moderation \\
	Decommission Process & Disclosure of AI Interaction & Decommission Process & Contract Review \\
	Dependency Screening  & External Audit & Dependency Screening  & Data Provenance \\
	External Audit & Failure Avoidance & External Audit & Data Retention \\
	Failure Avoidance & Fine Tuning & Failure Avoidance & Decommission Process \\
	Incident Response & Grounding & Human Review  & Dependency Screening  \\
	Internet Access & Human Review  & Incident Response & External Audit \\
	Kill Switch & Incident Response & Internet Access & Failure Avoidance \\
	Minors & Incorporate feedback  & Kill Switch & Human Review  \\
	Monitoring & Internet Access & Malware Screening & Incident Response \\
	Prohibited Use Policy & Minors & Minors & Insurance \\
	Rate-limiting  & Monitoring & Monitoring & Intellectual Property Removal \\
	Session Limits & Narrow Scope & Prohibited Use Policy & Internet Access \\
	Supply Chain Audit & RAG & Rate-limiting  & Malware Screening \\
	Terms of Service & Refresh & Session Limits & Minors \\
	& Regulated Dealings & Supply Chain Audit & Monitoring \\
	& RLHF & Terms of Service & Ownership \\
	& Session Limits & User Feedback & Prohibited Use Policy \\
	& System Prompt  &  & Rate-limiting  \\
	& Temperature &  & Regulated Dealings \\
	& Training  &  & Session Limits \\
	& User Feedback &  & Supply Chain Audit \\
	& User Recourse &  & System Documentation \\
	& Validation &  & Terms of Service \\
	&  &  & Training  \\
	&  &  & User Feedback \\
	&  &  & User Recourse \\
	\end{tabular}
\end{table}

\pagebreak

\begin{table}[H]
	\caption*{Table E.2: Selected risk controls organized by generative AI risk (continued).}
	\label{tab:controls_by_gai_risk_cont}
	\footnotesize
	\begin{tabular}{llll}
		\toprule
		Environmental & Human-AI Configuration & Information Integrity & Information Security \\
		\midrule
		Access Control  & Access Control  & Anonymous Use & Access Control  \\
		Anonymous Use & Anonymous Use & Antropomorphization  & Anonymous Use \\
		Approved List & Antropomorphization  & Approved List & Approved List \\
		Blocklist  & Approved List & Authentication  & Authentication  \\
		Change Management & Authentication  & Blocklist  & Blocklist  \\
		Decommission Process & Blocklist  & Change Management & Change Management \\
		External Audit & Change Management & Content Moderation & Content Moderation \\
		Failure Avoidance & Content Moderation & Data Provenance & Data Quality \\
		Incident Response & Data Retention & Data Quality & Data Retention \\
		Insurance & Decision making  & Data Retention & Decision making  \\
		Inventory & Digital Signature & Decommission Process & Decommission Process \\
		Kill Switch & Disclosure of AI Interaction & Digital Signature & Dependency Screening  \\
		Monitoring & External Audit & Disclosure of AI Interaction & External Audit \\
		Ownership & Failure Avoidance & External Audit & Failure Avoidance \\
		Session Limits & Human Review  & Failure Avoidance & Incident Response \\
		Training  & Incident Response & Fine Tuning & Incorporate feedback  \\
		& Incorporate feedback  & Grounding & Internet Access \\
		& Instructions & Human Review  & Inventory \\
		& Kill Switch & Incident Response & Kill Switch \\
		& Minors & Incorporate feedback  & Malware Screening \\
		& Monitoring & Instructions & Minors \\
		& Narrow Scope & Intellectual Property Removal & Monitoring \\
		& Ownership & Internet Access & Rate-limiting  \\
		& Prohibited Use Policy & Inventory & Redundancy \\
		& Regulated Dealings & Kill Switch & Session Limits \\
		& Session Limits & Minors & Supply Chain Audit \\
		& Terms of Service & Monitoring &  \\
		& Training  & Narrow Scope &  \\
		& User Feedback & Ownership &  \\
		& User Recourse & Prohibited Use Policy &  \\
		&  & RAG &  \\
		&  & RLHF &  \\
		&  & Refresh &  \\
		&  & Regulated Dealings &  \\
		&  & System Prompt  &  \\
		&  & Temperature &  \\
		&  & Terms of Service &  \\
		&  & Training  &  \\
		&  & User Feedback &  \\
		&  & User Recourse &  \\
		&  & Validation &  \\
		\bottomrule
	\end{tabular}	
\end{table}

\pagebreak

\begin{table}[H]
	\caption*{Table E.2: Selected risk controls organized by generative AI risk (continued).}
	\label{tab:controls_by_gai_risk_cont2}
	\footnotesize
	\begin{tabular}{lll}
		\toprule
		Intellectual Property & Obscene, Degrading, and/or Abusive Content & Toxicity, Bias, and Homogenization \\
		\midrule
		Contract Review & Access Control  & Anonymous Use \\
		Data Provenance & Anonymous Use & Approved List \\
		Digital Signature & Approved List & Blocklist  \\
		Disclosure of AI Interaction & Blocklist  & CSAM/Obsenity Removal  \\
		External Audit & CSAM/Obsenity Removal  & Change Management \\
		Human Review  & Change Management & Content Moderation \\
		Instructions & Content Moderation & Data Provenance \\
		Intellectual Property Removal & Data Retention & Data Quality \\
		Internet Access & Decommission Process & Decision making  \\
		Inventory & External Audit & External Audit \\
		Ownership & Failure Avoidance & Failure Avoidance \\
		Prohibited Use Policy & Human Review  & Fine Tuning \\
		Terms of Service & Incident Response & Grounding \\
		Training  & Internet Access & Human Review  \\
		User Feedback & Kill Switch & Incorporate feedback  \\
		User Recourse & Minors & Internet Access \\
		& Monitoring & Instructions \\
		& Session Limits & Kill Switch \\
		& User Feedback & Minors \\
		& User Recourse & Monitoring \\
		&  & Narrow Scope \\
		&  & Prohibited Use Policy \\
		&  & RAG \\
		&  & RLHF \\
		&  & Refresh \\
		&  & System Prompt  \\
		&  & Temperature \\
		&  & Terms of Service \\
		&  & User Feedback \\
		&  & User Recourse \\
		&  & Validation \\
		\bottomrule
	\end{tabular}
	\newline
	\vspace{10pt}
	\newline	
	\begin{tabular}{l}
		\toprule
		Value Chain and Component Integration \\
		\midrule
		Approved List \\
		Blocklist  \\
		CSAM/Obsenity Removal  \\
		Change Management \\
		Contract Review \\
		Data Provenance \\
		Data Quality \\
		Dependency Screening  \\
		Digital Signature \\
		Disclosure of AI Interaction \\
		External Audit \\
		Failure Avoidance \\
		Fine Tuning \\
		Grounding \\
		Insurance \\
		Intellectual Property Removal \\
		Internet Access \\
		Inventory \\
		Malware Screening \\
		Ownership \\
		Prohibited Use Policy \\
		Redundancy \\
		Supply Chain Audit \\
		System Documentation \\
		Terms of Service \\
		\bottomrule
	\end{tabular}	
\end{table}


% ---------- ----------
\section*{Appendix F: Example Low-risk Generative AI Measurement and Management Plan}\label{sec:appndxf}
% ---------- ----------

% ---------- ----------
\subsection{F.1: Example Low-risk Generative AI Measurement and Management Plan by Trustworthy Characteristic}\label{appdxf1}
% ---------- ----------

% ---------- ----------
\subsection{F.2: Example Low-risk Generative AI Measurement and Management Plan by Generative AI Risk}\label{appdxf2}
% ---------- ----------

% ---------- ----------
\section*{Appendix G: Example Medium-risk Generative AI Measurement and Management Plan}\label{sec:appndxg}
% ---------- ----------

% ---------- ----------
\subsection{G.1: Example Medium-risk Generative AI Measurement and Management Plan by Trustworthy Characteristic}\label{appdxg1}
% ---------- ----------

% ---------- ----------
\subsection{G.2: Example Medium-risk Generative AI Measurement and Management Plan by Generative AI Risk}\label{appdxg2}
% ---------- ----------

% ---------- ----------
\section*{Appendix H: Example High-risk Generative AI Measurement and Management Plan}\label{sec:appndxh}
% ---------- ----------

% ---------- ----------
\subsection{H.1: Example High-risk Generative AI Measurement and Management Plan by Trustworthy Characteristic}\label{appdxh1}
% ---------- ----------

% ---------- ----------
\subsection{H.2: Example High-risk Generative AI Measurement and Management Plan by Generative AI Risk}\label{appdxh2}
% ---------- ----------

\end{document}