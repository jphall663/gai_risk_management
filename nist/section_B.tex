\documentclass[fleqn]{article}

\usepackage{graphicx}
\usepackage{fullpage}
\usepackage{pdfpages}
\usepackage{amsmath}
\usepackage{amssymb}
\usepackage{mathtools}
\usepackage{MnSymbol}
\usepackage{enumerate}
\usepackage{setspace}
\usepackage[hyphens]{url}
\usepackage[colorlinks]{hyperref}
\usepackage{float}
\usepackage{caption}
\usepackage{subcaption}
\usepackage{multicol}
\usepackage{color}
\usepackage{listings}
\usepackage{csvsimple}
\usepackage{algorithm}
\usepackage{algorithmic}
\usepackage{verbatim}
\usepackage{mdframed}
\usepackage{changepage}
\usepackage[top=1in, bottom=1in, left=1in, right=1in]{geometry}

\usepackage{booktabs}
\usepackage{pdflscape}
\usepackage{makecell}
\usepackage{multirow}
\usepackage{enumitem}
\usepackage{tabto}

\begin{document}


% ---------- ----------
\section*{Appendix B: Example Risk-tiering Materials for Generative AI}\label{sec:appndxb}
% ---------- ----------

\subsection*{B.1: Example Adverse Impacts}

\begin{table}[H]
	\caption*{Table B.1: Example adverse impacts, adapted from NIST 800-30r1 Table H-2 \cite{nist80030r1}.}
	\footnotesize
	\begin{tabular}{|m{0.20\linewidth} | m{0.75\linewidth}|}
		\hline
		\textbf{Level} & \textbf{Description} \\ \hline
		Harm to Operations & 
		\begin{itemize}[noitemsep]
			\item Inability to perform current missions/business functions.
			\begin{itemize}[noitemsep,nolistsep]
				\item In a sufficiently timely manner.
				\item With sufficient confidence and/or correctness.
				\item Within planned resource constraints.
			\end{itemize}
			\item Inability, or limited ability, to perform missions/business functions in the future.           	
			\begin{itemize}[noitemsep,nolistsep]
				\item Inability to restore missions/business functions.
				\item In a sufficiently timely manner.
				\item With sufficient confidence and/or correctness.
				\item Within planned resource constraints.
			\end{itemize}
			\item Harms (e.g., financial costs, sanctions) due to noncompliance. 
			\begin{itemize}[noitemsep,nolistsep]
				\item With applicable laws or regulations.
				\item With contractual requirements or other requirements in other binding agreements (e.g., liability).
			\end{itemize}
			\item Direct financial costs.
			\item Reputational harms.	
			\begin{itemize}[noitemsep,nolistsep]
				\item Damage to trust relationships.
				\item Damage to image or reputation (and hence future or potential trust relationships).
			\end{itemize}
		\end{itemize} \\
		\hline
		Harm to Assets & 
		\begin{itemize}[noitemsep]
			\item Damage to or loss of physical facilities.
			\item Damage to or loss of information systems or networks.
			\item Damage to or loss of information technology or equipment.
			\item Damage to or loss of component parts or supplies.
			\item Damage to or of loss of information assets.
			\item Loss of intellectual property.            	
		\end{itemize} \\
		\hline
		Harm to Individuals & 
		\begin{itemize}[noitemsep]
			\item Injury or loss of life.
			\item Physical or psychological mistreatment.
			\item Identity theft.
			\item Loss of personally identifiable information.
			\item Damage to image or reputation.
			\item Infringement of intellectual property rights.
			\item Financial harm or loss of income.	
		\end{itemize} \\            
		\hline
		Harm to Other Organizations & 
		\begin{itemize}[noitemsep]
			\item Harms (e.g., financial costs, sanctions) due to noncompliance. 
			\begin{itemize}[noitemsep,nolistsep]
				\item With applicable laws or regulations.
				\item With contractual requirements or other requirements in other binding agreements (e.g., liability).
			\end{itemize}
			\item Direct financial costs.
			\item Reputational harms.	
			\begin{itemize}[noitemsep,nolistsep]
				\item Damage to trust relationships.
				\item Damage to image or reputation (and hence future or potential trust relationships).	
			\end{itemize}
		\end{itemize} \\            
		\hline
		Harm to the Nation & 
		\begin{itemize}[noitemsep]
			\item Damage to or incapacitation of critical infrastructure.
			\item Loss of government continuity of operations.
			\item Reputational harms.
			\begin{itemize}[noitemsep,nolistsep]
				\item Damage to trust relationships with other governments or with nongovernmental entities.
				\item Damage to national reputation (and hence future or potential trust relationships).
			\end{itemize} 
			\item Damage to current or future ability to achieve national objectives.
			\begin{itemize}[noitemsep,nolistsep]
				\item Harm to national security.
			\end{itemize}
			\item Large-scale economic or workforce displacement.
		\end{itemize} \\            
		\hline
	\end{tabular}
	\label{nist:adverse_impacts}
\end{table}

\subsection*{B.2: Example Impact Descriptions}

\begin{table}[H]
	\caption*{Table B.2: Example Impact level descriptions, adapted from NIST SP800-30r1 Appendix H, Table H-3 \cite{nist80030r1}.}
	\small
	\begin{tabular}{|m{0.20\linewidth} |m{0.20\linewidth} | m{0.05\linewidth} | m{0.40\linewidth}|}
		\hline
		\textbf{Qualitative Values} & \multicolumn{2}{c|}{\textbf{Semi-Quantitative Values}} & \textbf{Description} \\
		\hline
		Very High & 96-100 & 10 & An incident could be expected to have multiple severe or catastrophic adverse effects on organizational operations, organizational assets, individuals, other organizations, or the Nation. \\
		\hline
		High & 80-95 & 8 & An incident could be expected to have a severe or catastrophic adverse effect on organizational operations, organizational assets, individuals, other organizations, or the Nation. A severe or catastrophic adverse effect means that, for example, the incident might: (i) cause a severe degradation in or loss of mission capability to an extent and duration that the organization is not able to perform one or more of its primary functions; (ii) result in major damage to organizational assets; (iii) result in major financial loss; or (iv) result in severe or catastrophic harm to individuals involving loss of life or serious life-threatening injuries. \\
		\hline		
		Moderate & 21-79 & 5 & An incident could be expected to have a serious adverse effect on organizational operations, organizational assets, individuals other organizations, or the Nation. A serious adverse effect means that, for example, the incident might: (i) cause a significant degradation in mission capability to an extent and duration that the organization is able to perform its primary functions, but the effectiveness of the functions is significantly reduced; (ii) result in significant damage to organizational assets; (iii) result in significant financial loss; or (iv) result in significant harm to individuals that does not involve loss of life or serious life-threatening injuries. \\
		\hline
		Low & 5-20 & 2 & An incident could be expected to have a limited adverse effect on organizational operations, organizational assets, individuals other organizations, or the Nation. A limited adverse effect means that, for example, the incident might: (i) cause a degradation in mission capability to an extent and duration that the organization is able to perform its primary functions, but the effectiveness of the functions is noticeably reduced; (ii) result in minor damage to organizational assets; (iii) result in minor financial loss; or (iv) result in minor harm to individuals. \\
		\hline
		Very Low & 0-4 & 0 & An incident could be expected to have a negligible adverse effect on organizational operations, organizational assets, individuals other organizations, or the Nation. \\
		\hline
	\end{tabular}
	\label{table:nist_impacts}
\end{table}

\subsection*{B.3: Example Likelihood Descriptions}

\begin{table}[H]
	\caption*{Table B.3: Example likelihood levels, adapted from NIST SP800-30r1 Appendix G, Table G-3 \cite{nist80030r1}.}
	\begin{tabular}{|m{0.20\linewidth} |m{0.20\linewidth} | m{0.05\linewidth} | m{0.40\linewidth}|}
		\hline
		\textbf{Qualitative Values} & \multicolumn{2}{c|}{\textbf{Semi-Quantitative Values}} & \textbf{Description} \\
		\hline
		Very High & 96-100 & 10 & An incident is almost certain to occur; or occurs more than 100 times a year. \\
		\hline
		High & 80-95 & 8 & An incident is highly likely to occur; or occurs between 10-100 times a year.\\
		\hline		
		Moderate & 21-79 & 5 & An incident is somewhat likely to occur; or occurs between 1-10 times a year.\\
		\hline
		Low & 5-20 & 2 & An incident is unlikely to occur; or occurs less than once a year, but more than once every 10 years.  \\
		\hline
		Very Low & 0-4 & 0 & An incident is highly unlikely to occur; or occurs less than once every 10 years. \\
		\hline
	\end{tabular}
	\label{table:nist_likelihood}
\end{table}

\subsection*{B.4: Example Risk Tiers}

\begin{table}[H]
	\caption*{Table B.4: Example risk assessment matrix with 5 impact levels, 5 likelihood levels, and 5 risk tiers, adapted from NIST SP800-30r1 Appendix I, Table I-2 \cite{nist80030r1}.}
	\small
	\begin{tabular}{|c|c|c|c|c|c|c|}
		\hline
		\multirow{2}{*}{\textbf{Likelihood}} & \multicolumn{5}{|c|}{\textbf{Level of Impact}}   \\
		\cline{2-6}
		& \textbf{Very Low} & \textbf{Low} & \textbf{Moderate} & \textbf{High} & \textbf{Very High} \\
		\hline
		\textbf{Very High} & Very Low (Tier 5) & Low (Tier 4) & Moderate (Tier 3) & High (Tier 2) & Very High 
		(Tier 1) \\
		\hline		
		\textbf{High} & Very Low (Tier 5) & Low (Tier 4) & Moderate (Tier 3)  & High (Tier 2) & Very High (Tier 1) \\
		\hline
		\textbf{Moderate} & Very Low (Tier 5) & Low (Tier 4) & Moderate (Tier 3)  & Moderate (Tier 3) & High (Tier 2) \\
		\hline
		\textbf{Low} & Very Low (Tier 5) & Low (Tier 4) & Low (Tier 4)  & Low (Tier 4) & Moderate (Tier 3) \\
		\hline
		\textbf{Very Low} & Very Low (Tier 5) & Very Low (Tier 5) & Very Low (Tier 5) & Low (Tier 4) & Low (Tier 4) \\
		\hline
	\end{tabular}
	\label{table:nist_risk_tiers}
\end{table}

\subsection*{B.5: Example Risk Descriptions}

\begin{table}[H]
	\caption*{Table B.5: Example risk descriptions, adapted from NIST SP800-30r1 Appendix I, Table I-3 \cite{nist80030r1} .}
	\small
	\begin{tabular}{|m{0.20\linewidth} |m{0.20\linewidth} | m{0.05\linewidth} | m{0.40\linewidth}|}
		\hline
		\textbf{Qualitative Values} & \multicolumn{2}{c|}{\textbf{Semi-Quantitative Values}} & \textbf{Description} \\
		\hline
		Very High & 96-100 & 10 & Very high risk means that an incident could be expected to have multiple severe or catastrophic adverse effects on organizational operations, organizational assets, individuals, other organizations, or the Nation.\\
		\hline
		High & 80-95 & 8 & High risk means that an incident could be expected to have a severe or catastrophic adverse effect on organizational operations, organizational assets, individuals, other organizations, or the Nation. \\
		\hline		
		Moderate & 21-79 & 5 & Moderate risk means that an incident could be expected to have a serious adverse effect on organizational operations, organizational assets, individuals, other organizations, or the Nation.\\
		\hline
		Low & 5-20 & 2 & Low risk means that an incident could be expected to have a limited adverse effect on organizational operations, organizational assets, individuals, other organizations, or the Nation. \\
		\hline
		Very Low & 0-4 & 0 & Very low risk means that an incident could be expected to have a negligible adverse effect on organizational operations, organizational assets, individuals, other organizations, or the Nation. \\
		\hline
	\end{tabular}
	\label{table:nist_risk_descriptions}
\end{table}

\subsection*{B.6: Practical Risk-tiering Questions}

\noindent\textbf{B.6.1: Confabulation}: How likely are system outputs to contain errors? What are the impacts if errors occur?\\

\noindent\textbf{B.6.2: Dangerous and Violent Recommendations}: How likely is the system to give dangerous or violent recommendations? What are the impacts if it does?\\

\noindent\textbf{B.6.3: Data Privacy}: How likely is someone to enter sensitive data into the system? What are the impacts if this occurs? Are standard data privacy controls applied to the system to mitigate potential adverse impacts?\\

\noindent\textbf{B.6.4: Human-AI Configuration}: How likely is someone to use the system incorrectly or abuse it? How likely is use for decision-making? What are the impacts of incorrect use or abuse? What are the impacts of invalid or unreliable decision-making?\\

\noindent\textbf{B.6.5: Information Integrity}: How likely is the system to generate deepfakes or mis or disinformation? At what scale? Are content provenance mechanisms applied to system outputs? What are the impacts of generating deepfakes or mis or disinformation? Without controls for content provenance?\\

\noindent\textbf{B.6.6: Information Security}: How likely are system resources to be breached or exfiltrated? How likely is the system to be used in the generation of phishing or malware content? What are the impacts in these cases? Are standard information security controls applied to the system to mitigate potential adverse impacts? \\

\noindent\textbf{B.6.7: Intellectual Property}: How likely are system outputs to contain other entities' intellectual property? What are the impacts if this occurs?\\

\noindent\textbf{B.6.8: Toxicity, Bias, and Homogenization}: How likely are system outputs to be biased, toxic, homogenizing or otherwise obscene? How likely are system outputs to be used as subsequent training inputs? What are the impacts of these scenarios? Are standard nondiscrimination controls applied to mitigate potential adverse impacts? Is the application accessible to all user groups? What are the impacts if the system is not accessible to all user groups?\\

\noindent\textbf{B.6.9: Value Chain and Component Integration}: Are contracts relating to the system reviewed for legal risks? Are standard acquisition/procurement controls applied to mitigate potential adverse impacts? Do vendors provide incident response with guaranteed response times? What are the impacts if these conditions are not met?

% ---------- ----------
\subsection*{B.7: AI Risk Management Framework Actions Aligned to Risk Tiering}\label{appndxb7}
% ---------- ----------

GOVERN 1.3, GOVERN 1.5, GOVERN 2.3, GOVERN 3.2, GOVERN 4.1, GOVERN 5.2, GOVERN 6.1, MANAGE 1.2, MANAGE 1.3, MANAGE 2.1, MANAGE 2.2, MANAGE 2.3, 
MANAGE 2.4, MANAGE 3.1, MANAGE 3.2, MANAGE 4.1, MAP 1.1, MAP 1.5, MEASURE 2.6 \\

\noindent \textbf{Usage Note}: Materials in Appendix B can be used to create or update risk tiers or other risk assessment tools for GAI systems or applications as follows:
\begin{itemize}
	\item Table B.1 can enable mapping of GAI risks and impacts. 
	\item Table B.2 can enable quantification of impacts for risk tiering or risk assessment. 
	\item Table B.3 can enable quantification of likelihood for risk tiering or risk assessment.
	\item Table B.4 presents an example of combining assessed impact and likelihood into risk tiers. 
	\item Table B.5 presents example risk tiers with associated qualitative, semi-quantitative, and quantitative values for risk tiering  or risk assessment.
	\item Subsection B.6 presents example questions for qualitative risk assessment.
	\item Subsection B.7 highlights subcategories to indicate alignment with the AI RMF.   
\end{itemize} 
	
\end{document}